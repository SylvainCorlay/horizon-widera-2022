\begin{abstract}
  \TODO{now little discussion of what we will do, no mention of training}

  \begin{draft}
  Clearer problem statement: shift emphasis from work description to problem/impact

  Note from review:
doesn't provide a clear justification for the work, nor says much in terms of impact (eg, `scale', match with expected outcomes...). It mostly talks about what the project would do, as technical/research work, but doesn't say much about most of the whys. Why should the EC fund your project, and fund it now – ie, why is reproducibility with existing tools an important issue that must be fixed now to achieve the expected outcomes and Destination impacts? Here you could reuse some of the things you say eg in 1.1.4 Motivation.

- Also, I’m not sure it clearly says what is the problem you are trying to solve (it talks about the work to be done, which is not quite the same). Also, make clear that the scope of the work goes beyond Jupyter.

- You should make it clear here that this is about *computational* reproducibility. You say so in the 2st par of 1.1.1 Ambition, but it bears repeating here. Remember that that the abstract is used to select reviewers, and that the call topic is about reproducibility in general, not just about computational tools.
  \end{draft}

\TODO{In this abstract we want to summarize the vision of the project;
  - Context: Open Science; what is Jupyter; Jupyter is big
  - Who are we
  - What is our goal
  - What is our strategy / concept
Points to hit:
- Open Science should be practical, not just available
- Jupyter is part of the solution
- *brief* highlight of how/why Jupyter and Binder make sense:
  - Jupyter is widely adopted
  - notebook encapsulate computation
  - Binder builds on Jupyter to enable shareable reproducible environments
  - Jupyter is web-based, enabling building services
- What we plan to do
  - improve Jupyter/Binder toward open science
  - operate Jupyter-based services on EOSC
  - Open Science training (skip?)
- Who we are
  - Core Jupyter experts
  - Domain experts motivating/validating Jupyter improvements
}


% \begin{verbatim}
% Call: Increasing the service offer of the EOSC Portal

% Title: Open Science Publication of Research Environments (OSPRE)

% Points to hit:

% - Open Science should be practical, not just available
% - Jupyter is part of the solution
% - *brief* highlight of how/why Jupyter and Binder make sense:
%   - Jupyter is widely adopted
%   - notebook encapsulate computation
%   - Binder builds on Jupyter to enable shareable reproducible environments
%   - Jupyter is web-based, enabling building services
% - What we plan to do
%   - improve Binder toward open science
%   - operate Jupyter-based services on EOSC
%   - Open Science training (skip?)
% - Who we are
%   - Core Jupyter experts
%   - Domain experts motivating/validating Jupyter improvements
% \end{verbatim}



% % (a3) prompt: Services supporting scholarly communication and open access (4M): based on existing initiatives across Europe (institutional and thematic repositories, aggregators, etc.), the services should empower researchers and research communities and initiatives with the necessary tools and functionalities for systematic publishing, analysing and re-using of scientific results beyond publications (data, software and other artefacts), as well as supporting long-term preservation and curation. The services should also enable scientific workflows with adequate metrics and monitoring mechanisms supporting career development and the monitoring of funding and research impact. Support to a catch-all repository for open research should be provided.
% 
% %  To truly achieve the societal goals of Open Science,
% %  we must make progress beyond the `mere availability' of scientific results,
% %  to the practical usability and exploitation of such data once it is made available,
% %  an area where there is much room for improvement.
% %  The Jupyter project and its ecosystem show great promise
% %  as tools for bridging this gap; for making Open Science
% %  useful and accessible to all,
% %  from researchers to educators to public citizens.
% %  The Jupyter Notebook and Jupyter ecosystem are of increasing
% %  importance in computational science, data science, academia,
% %  industry, governments, and service providers,
% %  used by millions worldwide.
% %  Jupyter notebooks have great potential to push Open Science
% %  forward because they provide a complete description of a
% %  computational study that can be turned into a publication
% %  or produce part of a publication, such as a figure,
% %  making complex tasks reproducible.
% %  The Jupyter-based Binder project adds a means to execute notebooks
% %  in specified computational environments, an aspect of reproducibility
% %  not yet widely supported.
% %  In \TheProject, we will extend the capabilities of the Jupyter
% %  tools and ecosystem to add functionality that we view as having great
% %  importance for EOSC and Open Science more
% %  widely and operate services on EOSC as a demonstration.
% %
% %  Many \TheProject partners have longstanding experience and
% %  leadership roles in the Jupyter ecosystem,
% %  and in deploying services built on Jupyter to many users across the globe.
% %  Complementary to this core expertise,
% %  we integrate partners focussing on the application of these tools from a wide range of disciplines,
% %  both to demonstrate and ensure that our developments serve
% %  real-world Open Science use cases.
% %
% 
%   To truly achieve the societal goals of Open Science, we must make progress
%   beyond the `mere availability' of scientific results as Open Access, to the
%   practical usability and exploitation of such artefacts once they are made
%   available, an area where there is much room for improvement. The Jupyter
%   ecosystem shows great promise as a collection of tools for bridging this gap;
%   for making Open Science useful and accessible to all, empowering researchers,
%   educators, and public citizens. Jupyter is of increasing importance in
%   computational science, data science, academia, industry, governments, and
%   service providers, and used by millions worldwide. Jupyter notebooks have
%   great potential to push Open Science beyond publications because they
%   encapsulate a computational study that may be part of a publication, such as
%   the creation of a figure, a major part of making complex tasks reproducible.
%   The Jupyter-based Binder project adds a means to execute notebooks in
%   specified computational environments, an aspect of reproducibility not yet
%   widely supported, and of great falue to re-using scientific results.
% 
%   Tools such as Jupyter and Binder increase the value of all existing Open
%   Access initiatives by adding the axis of interactive computability, empowering
%   researchers to produce derivative and validating (or refuting) work. Services
%   such as Binder also xpose public metrics and monitoring, supporting the
%   monitoring of research impact and career development for any users of the
%   system.
% 
%   We will (i) extend the capabilities of the Jupyter tools and ecosystem to add
%   functionality that we view as essential to and providing great value for EOSC
%   and Open Science, focused on accessibility, interactive publications, and
%   reproducibility. Based on this framework of improved Jupyter tools, it will be
%   possible to build Open Science Publication of Research Environments
%   (\TheProject), and (ii) build a range of diverse innovative open services on
%   EOSC as part of this project, both to demonstrate and ensure that our
%   developments serve real-world Open Science use cases.
% 
%   Many \TheProject partners have longstanding experience and leadership roles in
%   the Jupyter ecosystem, and in deploying services built on Jupyter to many
%   users across the globe. Complementary to this core expertise, we integrate
%   partners focusing on the application of these tools to a wide range of
%   scientific disciplines and communities, for which EOSC-hosted demonstrator
%   services are developed.

\begin{abstract}
  \TODO{Comment on Binder being a generic tool -- not limited to Jupyter}

  \TODO{Comment on Best practice guidelines and Training we deliver}
  
To increase the reproducibility of scientific results, we need find technical
solutions that enable reproducibility, and which at the same time are practical
and acceptable to the researchers who (could) use them.

The Jupyter notebook and ecosystem offers high potential for reproducibility:
(i) it is used widely, (ii) the notebook is a document that includes the
computation, and (iii) the notebook is machine executable.

Reproducibility enabled through Jupyter notebooks can have high impact due to
the wide spread use of notebooks: the notebook provides an environment for
scientific data exploration and thinking that is embraced and adopted by many.

To be able to reproduce results from notebooks, the software environment in
which the notebook should be executed needs to be created. This project is
about automating and improving ways to do this by improving and extending the
Binder project.

The Binder project automates the creation of software environments in which
notebooks can be executed, based on software specification standards that
already exist.

In this project, we will (i) investigate which fraction of existing Jupyter
notebooks are reproducible. Such reproducibility either comes from the efforts
of pioneers in the field to make their repositories 'Binder-enabled' using the
already existing Binder software. Or the reproducibility of the software environment is
provided because the authors of the notebook deposited software specifications
in a standard way with the notebook, which the binder tool correctly interprets.

We will (ii) extend the robustness of the applicability of the
Binder tools. For example, authors may not specify which version of
software they have used. Using the commit dates in the repository, an
improved Binder can attempt to identify a working version. We will
(iii) extend Binder's capabilities to broaden the applicability and
increase the positive impact on reproducibility in science. 

We will evaluate our efforts with selected and new use cases, such as
data publishing and use of Binder in HPC contexts, and invite the
community of researchers to contribute to the project. 

Many \TheProject individuals have longstanding experience and leadership roles in the
development of the Jupyter ecosystem, and in deploying services built on
Jupyter to many users across the globe. Complementary to this core expertise, we
integrate partners focusing on the application of these tools to a range of
scientific disciplines and communities.

All outcomes from this project -- including improved and new software, best
practice guides, application examples and training materials -- will be made
available as open source. 
\end{abstract}

%%% Local Variables:
%%% mode: latex
%%% TeX-master: "proposal"
%%% End:


\TODO{Is it appropriate to mention who we are in the abstract? Or spend these words on something else, such as training/best practices?
}
\TODO{ANS: Yes, if we have the words, but lower priority than hitting all outcomes
}

% Project members have longstanding experience and leadership roles in the
% Jupyter ecosystem, and in deploying services built on
% Jupyter to millions of users across the globe.
% Complementary to this core expertise,
% we integrate partners focusing on the application of these tools to a range of scientific disciplines and communities.

\end{abstract}

%%% Local Variables:
%%% mode: latex
%%% TeX-master: "proposal"
%%% End:
