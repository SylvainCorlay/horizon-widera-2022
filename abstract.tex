\begin{abstract}
  \TODO{now little discussion of what we will do, no mention of training}

  \begin{draft}
  Clearer problem statement: shift emphasis from work description to problem/impact

  Note from review:
doesn't provide a clear justification for the work, nor says much in terms of impact (eg, `scale', match with expected outcomes...). It mostly talks about what the project would do, as technical/research work, but doesn't say much about most of the whys. Why should the EC fund your project, and fund it now – ie, why is reproducibility with existing tools an important issue that must be fixed now to achieve the expected outcomes and Destination impacts? Here you could reuse some of the things you say eg in 1.1.4 Motivation.

- Also, I’m not sure it clearly says what is the problem you are trying to solve (it talks about the work to be done, which is not quite the same). Also, make clear that the scope of the work goes beyond Jupyter.

- You should make it clear here that this is about *computational* reproducibility. You say so in the 2st par of 1.1.1 Ambition, but it bears repeating here. Remember that that the abstract is used to select reviewers, and that the call topic is about reproducibility in general, not just about computational tools.
  \end{draft}

The societal goals of Open Science -- particularly improved validation and reuse of research outputs --
can only be achieved if reproducibility is an integral part of the process.
\emph{Merely available} code is of drastically less value than code
that can be run easily by anyone.
As institutions and policies increasingly adopt Open and Reproducible policies,
researchers urgently need practical tools to fulfill these requirements.
Further, policy makers can benefit from tools for evaluation and enforcement of reproducibility policies.
In \TheProject, we focus on \emph{computational} reproducibility.

To reproduce computational results, the software environment in
which to execute the code must be created.
This project automates and improves ways to do this by developing and extending the
Binder project.

Binder aims to \myemph{automate existing practices}
for reproducible computational environments,
solving a key piece of reproducibility without dictating the execution workflow itself.
Binder is a subproject of Jupyter,
providing open source tools such as the widely used Jupyter notebook.

Binder has demonstrated potential in the Jupyter community by serving over ten million user sessions in 2021 at mybinder.org,
building over fifty thousand unique computational environments.
We aim to bring this potential to researchers not yet served by the tools,
by extending the robustness and the applicability of the
Binder tools,
and delivering clear communication and training about best practices for reproducibility.

We motivate and validate our efforts with selected use cases, such as
data publishing and in High Performance Computing (HPC) contexts where Binder tools are not currently practical,
and engage with the global community of researchers to contribute to the open source project.

\TODO{Is it appropriate to mention who we are in the abstract? Or spend these words on something else, such as training/best practices?
}
\TODO{ANS: Yes, if we have the words, but lower priority than hitting all outcomes
}

% Project members have longstanding experience and leadership roles in the
% Jupyter ecosystem, and in deploying services built on
% Jupyter to millions of users across the globe.
% Complementary to this core expertise,
% we integrate partners focusing on the application of these tools to a range of scientific disciplines and communities.

\end{abstract}

%%% Local Variables:
%%% mode: latex
%%% TeX-master: "proposal"
%%% End:
