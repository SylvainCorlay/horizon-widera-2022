\textbf{Horizon Europe Programme}

\textbf{Standard Application Form (HE RIA and IA)}

\textbf{Project proposal -- Technical description (Part B) }

\textbf{Version 1.2}

\textbf{25 May 2021}

\textbf{\uline{Structure of the Proposal }}

The proposal contains two parts:

- \textbf{Part A} of the proposal \textbf{is generated by the IT system.
It is based on the information entered by the participants through the
submission system in the Funding \& Tenders Portal.} The participants
can update the information in the submission system at any time before
final submission.

- \textbf{Part B} of the proposal is the narrative part that includes
three sections that each correspond to an evaluation criterion. Part B
needs to be uploaded as a PDF document following the templates
downloaded by the applicants in the submission system for the specific
call or topic. The templates for a specific call may slightly differ
from the example provided in this document.

The electronic submission system is an online wizard that guides you
step-by-step through the preparation of your proposal. The submission
process consists of 6 steps:

- Step 1: Logging in the Portal

- Step 2: Select the call, topic and type of action in the Portal

- Step 3: Create a draft proposal: Title, acronym, summary, main
organisation and contact details

- Step 4: Manage your parties and contact details: add your partner
organisations and contact details.

- Step 5: Edit and complete web forms for proposal part A and upload
proposal part B

- Step 6: Submit the proposal

\begin{longtable}[]{@{}
  >{\raggedright\arraybackslash}p{(\columnwidth - 4\tabcolsep) * \real{0.3333}}
  >{\raggedright\arraybackslash}p{(\columnwidth - 4\tabcolsep) * \real{0.3333}}
  >{\raggedright\arraybackslash}p{(\columnwidth - 4\tabcolsep) * \real{0.3333}}@{}}
\toprule
\endhead
\subsection{\texorpdfstring{\protect\hypertarget{_Toc448834151}{}{HISTORY
OF CHANGES}}{HISTORY OF CHANGES}} & & \\
& & \\
\subsection{\texorpdfstring{\protect\hypertarget{_Toc448834152}{}{Version}}{Version}}
&
\subsection{\texorpdfstring{\protect\hypertarget{_Toc448834153}{}{Publication
date}}{Publication date}} &
\subsection{\texorpdfstring{\protect\hypertarget{_Toc448834154}{}{Changes}}{Changes}} \\
& & \\
1.0 & \subsection{10.03.2021} & \\
& & \\
\begin{itemize}
\item
  Initial version
\end{itemize}

1.1 & \subsection{19.04.2021} & \\
\begin{itemize}
\item
  Formatting and alignment
\item
  Clarification of the indicative number of pages in section 2.2 is for
  sections 2.2 and 2.3
\item
  Added the name of the award criterion in section 3
\end{itemize}

1.2 & \subsection{25.05.2021} & \\
& & \\
\bottomrule
\end{longtable}

\begin{itemize}
\item
  Addition of a table in section 3.1 about in-kind contributions
\end{itemize}

\section{Proposal template Part B: technical description}

\textbf{\emph{(for full proposals: single stage submission procedure and
2}\emph{\textsuperscript{nd}}\emph{ stage of a two-stage submission
procedure)}}

This template is to be used in a single-stage submission procedure or at
the 2\textsuperscript{nd} stage of a two-stage submission procedure.

The structure of this template must be followed when preparing your
proposal. It has been designed to ensure that the important aspects of
your planned work are presented in a way that will enable the experts to
make an effective assessment against the evaluation criteria. Sections
1, 2 and 3 each correspond to an evaluation criterion.

Please be aware that proposals will be evaluated as they were submitted,
rather than on their potential if certain changes were to be made. This
means that only proposals that successfully address all the required
aspects will have a chance of being funded. There will be no possibility
for significant changes to content, budget and consortium composition
during grant preparation.

\textbf{ Page limit}: \uline{The title, list of participants and
sections 1, 2 and 3, together, should not be longer than 45 pages.} All
tables, figures, references and any other element pertaining to these
sections must be included as an integral part of these sections and are
thus counted against this page limit. The number of pages included in
each section of this template is only \textbf{indicative}.

The page limit will be applied automatically. \textbf{At the end of this
document you can see the structure of the actual proposal that you need
to submit, please remove all instruction pages that are watermarked. }

If you attempt to upload a proposal longer than the specified limit
before the deadline, you will receive an automatic warning and will be
advised to shorten and re-upload the proposal. After the deadline,
excess pages (in over-long proposals/applications) will be automatically
made invisible, and will not be taken into consideration by the experts.
The proposal is a self-contained document. Experts will be instructed to
ignore hyperlinks to information that is specifically designed to expand
the proposal, thus circumventing the page limit.

Please, do not consider the page limit as a target! It is in your
interest to keep your text as concise as possible, since experts rarely
view unnecessarily long proposals in a positive light.

The following formatting conditions apply.

The reference font for the body text of proposals is Times New Roman
(Windows platforms), Times/Times New Roman (Apple platforms) or Nimbus
Roman No. 9 L (Linux distributions).

The use of a different font for the body text is not advised and is
subject to the cumulative conditions that the font is legible and that
its use does not significantly shorten the representation of the
proposal in number of pages compared to using the reference font (for
example with a view to bypass the page limit).

The minimum font size allowed is 11 points. Standard character spacing
and a minimum of single line spacing is to be used. This applies to the
body text, including text in tables.

Text elements other than the body text, such as headers, foot/end notes,
captions, formula's, may deviate, but must be legible.

The page size is A4, and all margins (top, bottom, left, right) should
be at least 15 mm (not including any footers or headers).

\emph{Fill in the title of your proposal below.}

\textbf{\textsc{Title of the Proposal}}

\emph{The consortium members are listed in part A of the proposal
(application forms). A summary list should also be provided in the table
below.}

\textbf{List of participants}

\begin{longtable}[]{@{}
  >{\raggedright\arraybackslash}p{(\columnwidth - 4\tabcolsep) * \real{0.3333}}
  >{\raggedright\arraybackslash}p{(\columnwidth - 4\tabcolsep) * \real{0.3333}}
  >{\raggedright\arraybackslash}p{(\columnwidth - 4\tabcolsep) * \real{0.3333}}@{}}
\toprule
\endhead
\subparagraph{Participant No. *} &
\subparagraph{\texorpdfstring{\textbf{Participant
organisation}shapeType202fFlipH0fFlipV0rotation-2949120fLockRotation0fLockAspectRatio0fLockPosition0fLockAgainstSelect0fLockVerticies0fLockText0fLockAdjustHandles0fLockAgainstGrouping0lTxid393216fRotateText0fFitShapeToText1fFilled0lineJoinStyle2fArrowheadsOK0fLine0fLockShapeType1wzNameText
Box
28posrelh0posrelv0metroBlobdhgt251653632fLayoutInCell0fAllowOverlap1fBehindDocument1fHidden0fLayoutInCell0Instructions,
please
remove}{Participant organisationshapeType202fFlipH0fFlipV0rotation-2949120fLockRotation0fLockAspectRatio0fLockPosition0fLockAgainstSelect0fLockVerticies0fLockText0fLockAdjustHandles0fLockAgainstGrouping0lTxid393216fRotateText0fFitShapeToText1fFilled0lineJoinStyle2fArrowheadsOK0fLine0fLockShapeType1wzNameText Box 28posrelh0posrelv0metroBlobdhgt251653632fLayoutInCell0fAllowOverlap1fBehindDocument1fHidden0fLayoutInCell0Instructions, please remove}}

\subparagraph{\texorpdfstring{\textbf{Participant
organisation}Instructions, please
remove}{Participant organisationInstructions, please remove}}

\subparagraph{Participant organisation name} & \subparagraph{Country} \\
& & \\
\subparagraph{1 (Coordinator)} & & \\
& & \\
\subparagraph{2} & & \\
& & \\
\subparagraph{3} & & \\
& & \\
\bottomrule
\end{longtable}

* Please use the same participant numbering and name as that used in the
administrative proposal forms.

\textbf{1. Excellence}

\begin{itemize}
\item
  \emph{The following aspects will be taken into account only to the
  extent that the proposed work is within the scope of the work
  programme
  topic.}shapeType202fFlipH0fFlipV0rotation-2949120fLockRotation0fLockAspectRatio0fLockPosition0fLockAgainstSelect0fLockVerticies0fLockText0fLockAdjustHandles0fLockAgainstGrouping0lTxid458752fRotateText0fFitShapeToText1fFilled0lineJoinStyle2fArrowheadsOK0fLine0fLockShapeType1wzNameText
  Box
  29posrelh0posrelv0metroBlobdhgt251658240fLayoutInCell0fAllowOverlap1fBehindDocument1fHidden0fLayoutInCell0Instructions,
  please remove
\item
  \emph{The following aspects will be taken into account only to the
  extent that the proposed work is within the scope of the work
  programme topic.}Instructions, please remove
\item
  \emph{The following aspects will be taken into account only to the
  extent that the proposed work is within the scope of the work
  programme topic.}
\end{itemize}

\textbf{1.1 Objectives and ambition }\emph{{[}e.g. 4 pages{]}}

\begin{itemize}
\item
  Briefly describe the objectives of your proposed work. Why are they
  pertinent to the work programme topic? Are they measurable and
  verifiable? Are they realistically achievable?
\item
  Describe how your project goes beyond the state-of-the-art, and the
  extent the proposed work is ambitious. Indicate any exceptional
  ground-breaking R\&I, novel concepts and approaches, new products,
  services or business and organisational models. Where relevant,
  illustrate the advance by referring to products and services already
  available on the market. Refer to any patent or publication search
  carried out.
\item
  Describe where the proposed work is positioned in terms of R\&I
  maturity (i.e. where it is situated in the spectrum from `idea to
  application', or from `lab to market'). Where applicable, provide an
  indication of the Technology Readiness Level, if possible
  distinguishing the start and by the end of the project.
\end{itemize}

\begin{itemize}
\item
  \emph{Please bear in mind that advances beyond the state of the art
  must be interpreted in the light of the positioning of the project.
  Expectations will not be the same for RIAs at lower TRL, compared with
  Innovation Actions at high TRLs. }
\end{itemize}

\textbf{1.2 Methodology }\emph{{[}e.g. 15 pages{]}}

\begin{itemize}
\item
  Describe and explain the overall methodology, including the concepts,
  models and assumptions that underpin your work. Explain how this will
  enable you to deliver your project's objectives. Refer to any
  important challenges you may have identified in the chosen methodology
  and how you intend to overcome them. \emph{{[}e.g. 10 pages{]}}
\end{itemize}

\begin{itemize}
\item
  \emph{This section should be presented as a narrative. The detailed
  tasks and work packages are described below under `Implementation'. }
\item
  \emph{Where relevant, include how the project methodology complies
  with the `do no significant harm' principle as per Article 17 of on
  the establishment of a framework to facilitate sustainable investment
  (i.e. the so-called 'EU Taxonomy Regulation'). This means that the
  methodology is designed in a way it is not significantly harming any
  of the six environmental objectives of the EU Taxonomy Regulation.}
\end{itemize}

\begin{itemize}
\item
  \emph{Note: This section is mandatory except for topics which have
  been identified in the work programme as not requiring the integration
  of the
  }shapeType202fFlipH0fFlipV0rotation-2949120fLockRotation0fLockAspectRatio0fLockPosition0fLockAgainstSelect0fLockVerticies0fLockText0fLockAdjustHandles0fLockAgainstGrouping0lTxid524288fRotateText0fFitShapeToText1fFilled0lineJoinStyle2fArrowheadsOK0fLine0fLockShapeType1wzNameText
  Box
  30posrelh0posrelv0metroBlobdhgt251655680fLayoutInCell0fAllowOverlap1fBehindDocument1fHidden0fLayoutInCell0Instructions,
  please remove
\item
  \emph{Note: This section is mandatory except for topics which have
  been identified in the work programme as not requiring the integration
  of the }Instructions, please remove
\item
  \emph{Note: This section is mandatory except for topics which have
  been identified in the work programme as not requiring the integration
  of the gender dimension into R\&I content.}
\item
  \emph{Remember that that this question relates to the \uline{content}
  of the planned research and innovation activities, and not to gender
  balance in the teams in charge of carrying out the project.}
\item
  \emph{Sex and gender analysis refers to biological characteristics and
  social/cultural factors respectively. For guidance on methods of sex /
  gender analysis and the issues to be taken into account, please refer
  to }
\end{itemize}

\begin{itemize}
\item
  Describe how appropriate open science practices are implemented as an
  integral part of the proposed methodology. Show how the choice of
  practices and their implementation are adapted to the nature of your
  work, in a way that will increase the chances of the project
  delivering on its objectives \emph{{[}e.g. 1 page{]}}. If you believe
  that none of these practices are appropriate for your project, please
  provide a justification here.
\end{itemize}

\begin{itemize}
\item
  \emph{Open science is an approach based on open cooperative work and
  systematic sharing of knowledge and tools as early and widely as
  possible in the process. Open science practices include early and open
  sharing of research (for example through preregistration, registered
  reports, pre-prints, or crowd-sourcing); research output management;
  measures to ensure reproducibility of research outputs; providing open
  access to research outputs (such as publications, data, software,
  models, algorithms, and workflows); participation in open peer-review;
  and involving all relevant knowledge actors including citizens, civil
  society and end users in the co-creation of R\&I agendas and contents
  (such as citizen science).}
\item
  \emph{Please note that this question does not refer to outreach
  actions that may be planned as part of communication, dissemination
  and exploitation activities. These aspects should instead be described
  below under `Impact'.}
\end{itemize}

\begin{itemize}
\item
  Research\textbf{ data management and management of other research
  outputs: }Applicants generating/collecting data and/or other research
  outputs (except for publications) during the project must provide
  maximum 1 page on how the data/ research outputs will be managed in
  line with the FAIR principles (Findable, Accessible, Interoperable,
  Reusable), addressing the following (the description should be
  specific to your project): \emph{{[}1 page{]}}
\end{itemize}

\textbf{Types of data/research outputs} (e.g. experimental,
observational, images, text, numerical) and their estimated size; if
applicable, combination with, and provenance of, existing data.

\textbf{Findability of data/research outputs:} Types of~persistent and
unique~identifiers (e.g. digital object identifiers)~and~trusted
repositories~that will be used.

\textbf{Accessibility of data/research outputs:}~IPR considerations and
timeline for open access (if open access not provided, explain why);
provisions for access to restricted data for verification purposes.

\textbf{Interoperability of data/research outputs:} Standards, formats
and vocabularies for data and metadata.

\textbf{Reusability of data/research outputs}:~ Licenses for data
sharing and re-use (e.g. Creative Commons, Open Data
Commons);~availability of tools/software/models for data generation and
validation/interpretation /re-use.

\textbf{Curation and storage/preservation costs}; person/team
responsible for data management and quality assurance.

\begin{itemize}
\item
  \emph{Proposals selected for funding under Horizon Europe will need to
  develop a~detailed data management plan (DMP) for making their
  data/research outputs findable, accessible, interoperable and reusable
  (FAIR) as a deliverable by month 6 and revised towards the end of a
  project's lifetime. }
\item
  \emph{For guidance on open science practices and research data
  management, please refer to the relevant section of the on the Funding
  \& Tenders Portal.}
\end{itemize}

\textbf{2. Impact}

\begin{longtable}[]{@{}l@{}}
\toprule
\endhead
\textbf{\emph{Impact }\emph{--}\emph{ aspects to be taken into
account.}} \\
 \\
\bottomrule
\end{longtable}

\begin{itemize}
\item
  Credibility of the pathways to achieve the expected outcomes and
  impacts specified in the work programme, and the likely scale and
  significance of the contributions due to the project.
\item
  Suitability and quality of the measures to maximise expected outcomes
  and impacts, as set out in the dissemination and exploitation plan,
  including communication activities.
\end{itemize}

\emph{The results of your project should make a contribution to the
expected outcomes set out for the work programme topic over the medium
term, and to the wider expected impacts set out in the `destination'
over the longer term. }

\emph{In this section you should show how your project could contribute
to the outcomes and impacts described in the work programme, the likely
scale and significance of this contribution, and the measures to
maximise these impacts.}\textbf{ }

\textbf{2.1 Project's pathways towards impact \emph{{[}}}\emph{e.g. 4
pages{]}}

\begin{itemize}
\item
  Provide a \textbf{narrative} explaining how the project's results are
  expected to make a difference in terms of impact, beyond the immediate
  scope and duration of the project. The narrative should include the
  components below, tailored to your project.
\end{itemize}

\begin{enumerate}
\def\labelenumi{\alph{enumi}.}
\item
  Describe the unique contribution your project results would make
  towards (1) the \textbf{outcomes} specified in this topic, and (2) the
  \textbf{wider impacts}, in the longer term, specified in the
  respective destinations in the work programme.
\end{enumerate}

\begin{itemize}
\item
  \emph{Be specific, referring to the effects of your project, and not
  R\&I in general in this field. }
\item
  \emph{State the target groups that would benefit. Even if target
  groups are mentioned in general terms in the work programme, you
  should be specific here, breaking target groups into particular
  interest groups or segments of society relevant to this project.}
\item
  \emph{The outcomes and impacts of your project may:}

  \begin{itemize}
  \item
    \emph{Scientific, e.g. contributing to specific scientific advances,
    across and within disciplines, creating new knowledge, reinforcing
    scientific equipment and instruments, computing systems (i.e.
    research
    infrastructur}shapeType202fFlipH0fFlipV0rotation-2949120fLockRotation0fLockAspectRatio0fLockPosition0fLockAgainstSelect0fLockVerticies0fLockText0fLockAdjustHandles0fLockAgainstGrouping0lTxid655360fRotateText0fFitShapeToText1fFilled0lineJoinStyle2fArrowheadsOK0fLine0fLockShapeType1wzNameText
    Box
    32posrelh0posrelv0metroBlobdhgt251659776fLayoutInCell0fAllowOverlap1fBehindDocument1fHidden0fLayoutInCell0Instructions,
    please remove
  \item
    \emph{Scientific, e.g. contributing to specific scientific advances,
    across and within disciplines, creating new knowledge, reinforcing
    scientific equipment and instruments, computing systems (i.e.
    research infrastructur}Instructions, please remove
  \item
    \emph{Scientific, e.g. contributing to specific scientific advances,
    across and within disciplines, creating new knowledge, reinforcing
    scientific equipment and instruments, computing systems (i.e.
    research infrastructures);}
  \item
    \emph{Economic/technological, e.g. bringing new products, services,
    business processes to the market, increasing efficiency, decreasing
    costs, increasing profits, contributing to standards' setting, etc.
    }
  \item
    \emph{Societal , e.g. decreasing CO\textsubscript{2} emissions,
    decreasing avoidable mortality, improving policies and decision
    making, raising consumer awareness.}
  \end{itemize}
\end{itemize}

\emph{Only include such outcomes and impacts where your project would
make a significant and direct contribution. Avoid describing very
tenuous links to wider impacts.} \emph{However, include any potential
negative environmental outcome or impact of the project including when
expected results are brought at scale (such as at commercial level).
Where relevant, explain how the potential harm can be managed.}

\begin{enumerate}
\def\labelenumi{\alph{enumi}.}
\item
  Describe any requirements and potential barriers - arising from
  factors beyond the scope and duration of the project - that may
  determine whether the desired outcomes and impacts are achieved. These
  may include, for example, other R\&I work within and beyond Horizon
  Europe; regulatory environment; targeted markets; user behaviour.
  Indicate if these factors might evolve over time. Describe any
  mitigating measures you propose, within or beyond your project, that
  could be needed should your assumptions prove to be wrong, or to
  address identified barriers.
\end{enumerate}

\begin{itemize}
\item
  \emph{Note that this does not include the critical risks inherent to
  the management of the project itself , which should be described below
  under `Implementation'.}
\end{itemize}

\begin{enumerate}
\def\labelenumi{\alph{enumi}.}
\item
  Give an indication of the scale and significance of the project's
  contribution to the expected outcomes and impacts, should the project
  be successful. Provide quantified estimates where possible and
  meaningful.
\end{enumerate}

\begin{itemize}
\item
  `\emph{\uline{Scale}\uline{'} refers to how widespread the outcomes
  and impacts are likely to be. For example, in terms of the size of the
  target group, or the proportion of that group, that should benefit
  over time; \uline{`}\uline{Significance}\uline{'} refers to the
  importance, or value, of those benefits. For example, number of
  additional healthy life years; efficiency savings in energy supply.}
\end{itemize}

\begin{itemize}
\item
  \emph{Explain your baselines, benchmarks and assumptions used for
  those estimates. Wherever possible, quantify your estimation of the
  effects that you expect from your project. Explain assumptions that
  you make, referring for example to any relevant studies or statistics.
  Where appropriate, try to use only one methodology for calculating
  your estimates: not different methodologies for each partner, region
  or country (the extrapolation should preferably be prepared by one
  partner).}
\item
  \emph{Your estimate must relate to this project only - the effect of
  other initiatives should not be taken into account.}
\end{itemize}

\textbf{2.2 Measures to maximise impact - Dissemination, exploitation
and communication }\emph{{[}e.g. 5 pages, including section 2.3{]}}

\begin{itemize}
\item
  Describe the planned measures to maximise the impact of your project
  by providing a first version of your `\uline{plan for the
  dissemination and exploitation including communication activities'}.
  Describe the dissemination, exploitation and communication measures
  that are planned, and the target group(s) addressed (e.g. scientific
  community, end users, financial actors, public at large).
\end{itemize}

\begin{itemize}
\item
  \emph{Please remember that this plan is an admissibility condition,
  unless the work programme topic explicitly states otherwise. In case
  your proposal is selected for funding, a more detailed `plan for
  dissemination and exploitation including communication activities'
  will need to be provided as a mandatory project deliverable within 6
  months after signature date. This plan shall be periodically updated
  in alignment with the project's progress. }
\item
  \emph{\uline{Communication}}\footnote{footnote text,Schriftart: 9
    pt,Schriftart: 10 pt,Schriftart: 8 pt,WB-Fuß For further guidance on
    communicating EU research and innovation for project participants,
    please refer to the on the Funding \& Tenders Portal}\emph{ measures
  should promote the project throughout the full lifespan of the
  project. The aim is to inform and reach out to society and show the
  activities performed, and the use and the benefits the project will
  have for citizens. Activities must be strategically planned, with
  clear objectives, start at the outset and continue through the
  lifetime of the project. The description of the communication
  activities needs to state the main
  m}shapeType202fFlipH0fFlipV0rotation-2949120fLockRotation0fLockAspectRatio0fLockPosition0fLockAgainstSelect0fLockVerticies0fLockText0fLockAdjustHandles0fLockAgainstGrouping0lTxid720896fRotateText0fFitShapeToText1fFilled0lineJoinStyle2fArrowheadsOK0fLine0fLockShapeType1wzNameText
  Box
  33posrelh0posrelv0metroBlobdhgt251660800fLayoutInCell0fAllowOverlap1fBehindDocument1fHidden0fLayoutInCell0Instructions,
  please remove
\item
  \emph{\uline{Communication}}\footnote{footnote text,Schriftart: 9
    pt,Schriftart: 10 pt,Schriftart: 8 pt,WB-Fuß For further guidance on
    communicating EU research and innovation for project participants,
    please refer to the on the Funding \& Tenders Portal}\emph{ measures
  should promote the project throughout the full lifespan of the
  project. The aim is to inform and reach out to society and show the
  activities performed, and the use and the benefits the project will
  have for citizens. Activities must be strategically planned, with
  clear objectives, start at the outset and continue through the
  lifetime of the project. The description of the communication
  activities needs to state the main m}Instructions, please remove
\item
  \emph{\uline{Communication}}\footnote{footnote text,Schriftart: 9
    pt,Schriftart: 10 pt,Schriftart: 8 pt,WB-Fuß For further guidance on
    communicating EU research and innovation for project participants,
    please refer to the on the Funding \& Tenders Portal}\emph{ measures
  should promote the project throughout the full lifespan of the
  project. The aim is to inform and reach out to society and show the
  activities performed, and the use and the benefits the project will
  have for citizens. Activities must be strategically planned, with
  clear objectives, start at the outset and continue through the
  lifetime of the project. The description of the communication
  activities needs to state the main messages as well as the tools and
  channels that will be used to reach out to each of the chosen target
  groups.}
\item
  \emph{All measures should be proportionate to the scale of the
  project, and should contain concrete actions to be implemented both
  during and after the end of the project, e.g. standardisation
  activities. Your plan should give due consideration to the possible
  follow-up of your project, once it is finished. In the justification,
  explain why each measure chosen is best suited to reach the target
  group addressed. Where relevant, and for innovation actions, in
  particular, describe the measures for a plausible path to
  commercialise the innovations.}
\item
  \emph{If exploitation is expected primarily in non-associated third
  countries, justify by explaining how that exploitation is still in the
  Union's interest.}
\item
  \emph{Describe possible feedback to policy measures generated by the
  project that will contribute to designing, monitoring, reviewing and
  rectifying (if necessary) existing policy and programmatic measures or
  shaping and supporting the implementation of new policy initiatives
  and decisions.}
\end{itemize}

\begin{itemize}
\item
  Outline your strategy for the management of intellectual property,
  foreseen protection measures, such as patents, design rights,
  copyright, trade secrets, etc., and how these would be used to support
  exploitation.
\end{itemize}

\begin{itemize}
\item
  \emph{If your project is selected, you will need an appropriate
  consortium agreement to manage (amongst other things) the ownership
  and access to key knowledge (IPR, research data etc.). Where relevant,
  these will allow you, collectively and individually, to pursue market
  opportunities arising from the project.}
\item
  \emph{If your project is selected, you must indicate the owner(s) of
  the results (results ownership list) in the final periodic report.}
\end{itemize}

\textbf{2.3 Summary }

Provide a summary of this section by presenting in the canvas below the
key elements of your project impact pathway and of the measures to
maximise its impact.

\textbf{KEY ELEMENT OF THE IMPACT SECTION}

\begin{longtable}[]{@{}
  >{\raggedright\arraybackslash}p{(\columnwidth - 0\tabcolsep) * \real{1.0000}}@{}}
\toprule
\endhead
A \textbf{major electronic company} (Samsung or Apple)
\textbf{exploits/uses the new product} in their manufacturing. \\
 \\
\textbf{IMPACTS} \\
 \\
\emph{What are the expected wider scientific, economic and societal
effects of the project contributing to the expected impacts outlined in
the respective destination in the work programme?}

Example 1

\textbf{Scientific: } New breakthrough scientific discovery on passenger
forecast modelling.

\textbf{Economic:} Increased airport efficiency

Size: 15\% increase of maximum passenger capacity in European airports,
leading to a 28\% reduction in infrastructure expansion costs.

Example 2

\textbf{Scientific:} New breakthrough scientific discovery on
transparent electronics.

\textbf{Economic/Technological:} A new market for touch enabled
electronic devices.

\textbf{Societal:} Lower climate impact of electronics manufacturing
(including through material sourcing and waste management). \\
 \\
\bottomrule
\end{longtable}

\begin{longtable}[]{@{}
  >{\raggedright\arraybackslash}p{(\columnwidth - 0\tabcolsep) * \real{1.0000}}@{}}
\toprule
\endhead
\begin{enumerate}
\def\labelenumi{\arabic{enumi}.}
\setcounter{enumi}{2}
\item
  \textbf{Quality and efficiency of the implementation}
\end{enumerate}

\textbf{\emph{Quality and efficiency of the implementation}\emph{
}\emph{--}\emph{ aspects to be taken into account}} \\
 \\
\bottomrule
\end{longtable}

\begin{itemize}
\item
  \emph{Quality and effectiveness of the work plan, assessment of risks,
  and appropriateness of the effort assigned to work packages, and the
  resources overall}
\item
  \emph{Capacity and role of each participant, and extent to which the
  consortium as a whole brings together the necessary expertise.}
\end{itemize}

\textbf{3.1 Work plan and resources }\emph{{[}e.g. 14 pages -- including
tables{]}}

Please provide the following:

\begin{itemize}
\item
  brief presentation of the overall structure of the work plan;
\item
  timing of the different work packages and their components (Gantt
  chart or similar);
\item
  graphical presentation of the components showing how they inter-relate
  (Pert chart or similar).
\item
  detailed work description,
  i.e.:shapeType202fFlipH0fFlipV0rotation-2949120fLockRotation0fLockAspectRatio0fLockPosition0fLockAgainstSelect0fLockVerticies0fLockText0fLockAdjustHandles0fLockAgainstGrouping0lTxid917504fRotateText0fFitShapeToText1fFilled0lineJoinStyle2fArrowheadsOK0fLine0fLockShapeType1wzNameText
  Box
  36posrelh0posrelv0metroBlobdhgt251661824fLayoutInCell0fAllowOverlap1fBehindDocument1fHidden0fLayoutInCell0Instructions,
  please remove
\item
  detailed work description, i.e.:Instructions, please remove
\item
  detailed work description, i.e.:

  \begin{itemize}
  \item
    a list of work packages (table 3.1a);
  \item
    a description of each work package (table 3.1b);
  \item
    a list of deliverables (table 3.1c);
  \end{itemize}
\end{itemize}

\begin{itemize}
\item
  \emph{Give full details. Base your account on the logical structure of
  the project and the stages in which it is to be carried out.}
  \emph{The number of work packages should be proportionate to the scale
  and complexity of the project.}
\item
  \emph{You should give enough detail in each work package to justify
  the proposed resources to be allocated and also quantified information
  so that progress can be monitored, including by the Commission}
\item
  \emph{Resources assigned to work packages should be in line with their
  objectives and deliverables. You are advised to include a distinct
  work package on `project management', and to give due visibility in
  the work plan to `data management' `dissemination and exploitation'
  and `communication activities', either with distinct tasks or distinct
  work packages. }
\item
  \emph{You will be required to update the `plan for the dissemination
  and exploitation of results including communication activities', and a
  `data management plan', (this does not apply to topics where a plan
  was not required.) This should include a record of activities related
  to dissemination and exploitation that have been undertaken and those
  still planned. }
\item
  \emph{Please make sure the information in this section matches the
  costs as stated in the budget table in section 3 of the application
  forms, and the number of person months, shown in the detailed work
  package descriptions.}
\end{itemize}

\begin{itemize}
\item
  a list of milestones (table 3.1d);
\item
  a list of critical risks, relating to project implementation, that the
  stated project's objectives may not be achieved. Detail any risk
  mitigation measures. You will be able to update the list of critical
  risks and mitigation measures as the project progresses (table 3.1e);
\item
  a table showing number of person months required (table 3.1f);
\item
  a table showing description and justification of subcontracting costs
  for each participant (table 3.1g);
\item
  a table showing justifications for `purchase costs' (table 3.1h) for
  participants where those costs exceed 15\% of the personnel costs
  (according to the budget table in proposal part A);
\item
  if applicable, a table showing justifications for `other costs
  categories' (table 3.1i);
\item
  if applicable, a table showing in-kind contributions from third
  parties (table 3.1j)
\end{itemize}

\textbf{3.2 Capacity of participants and consortium as a whole
}\emph{{[}e.g. 3 pages{]} }

\emph{The individual members of the consortium are described in a
separate section under Part A. There is no need to repeat that
information here. }

\begin{itemize}
\item
  Describe the consortium. How does it match the
\item
  Describe the consortium. How does it match the project's objectives,
  and bring together the necessary disciplinary and inter-disciplinary
  knowledge. Show how this includes expertise in social sciences and
  humanities, open science practices, and gender aspects of R\&I, as
  appropriate. Include in the description affiliated entities and
  associated partners, if any.
\item
  Show how the partners will have access to critical infrastructure
  needed to carry out the project activities.
\item
  Describe how the members complement one another (and cover the value
  chain, where appropriate)
\item
  In what way does each of them contribute to the project? Show that
  each has a valid role, and adequate resources in the project to fulfil
  that role.
\item
  If applicable, describe the industrial/commercial involvement in the
  project to ensure exploitation of the results and explain why this is
  consistent with and will help to achieve the specific measures which
  are proposed for exploitation of the results of the project (see
  section 2.2).
\item
  \textbf{Other countries and international organisations}: If one or
  more of the participants requesting EU funding is based in a country
  or is an international organisation that is not automatically eligible
  for such funding (entities from Member States of the EU, from
  Associated Countries and from one of the countries in the exhaustive
  list included in the Work Programme General Annexes B are
  automatically eligible for EU funding), explain why the participation
  of the entity in question is essential to successfully carry out the
  project.
\end{itemize}

\textbf{Tables for section 3.1}

\textbf{Table 3.1a: List of work packages}

\textbf{Table 3.1b: Work package description }

\textbf{For each work package: }

\textbf{Table 3.1c: List of Deliverables}\footnote{\begin{longtable}[]{@{}l@{}}
  \toprule
  \endhead
  \textbf{Description of work} (where appropriate, broken down into
  tasks), lead partner and role of participants \\
  \textbf{Deliverables} (brief description and month of delivery) \\
   \\
  \bottomrule
  \end{longtable}

  footnote text,Schriftart: 9 pt,Schriftart: 10 pt,Schriftart: 8
  pt,WB-Fuß You must include a data management plan (DMP) and a `plan
  for dissemination and exploitation including communication activities
  as distinct deliverables within the first 6 months of the project. The
  DMP will evolve during the lifetime of the project in order to present
  the status of the project's reflections on data management. A template
  for such a plan is available in the on the Funding \& Tenders Portal.}\textbf{
}

Only include deliverables that you consider essential for effective
project monitoring.

\begin{longtable}[]{@{}
  >{\raggedright\arraybackslash}p{(\columnwidth - 12\tabcolsep) * \real{0.1429}}
  >{\raggedright\arraybackslash}p{(\columnwidth - 12\tabcolsep) * \real{0.1429}}
  >{\raggedright\arraybackslash}p{(\columnwidth - 12\tabcolsep) * \real{0.1429}}
  >{\raggedright\arraybackslash}p{(\columnwidth - 12\tabcolsep) * \real{0.1429}}
  >{\raggedright\arraybackslash}p{(\columnwidth - 12\tabcolsep) * \real{0.1429}}
  >{\raggedright\arraybackslash}p{(\columnwidth - 12\tabcolsep) * \real{0.1429}}
  >{\raggedright\arraybackslash}p{(\columnwidth - 12\tabcolsep) * \real{0.1429}}@{}}
\toprule
\endhead
& & & & & & \\
& & & & & & \\
& & & & & & \\
& & & & & & \\
\textbf{KEY }

Deliverable numbers in order of delivery dates. Please use the numbering
convention \textless WP number\textgreater.\textless number of
deliverable within that WP\textgreater.

For example, deliverable 4.2 would be the second deliverable from work
package 4.

\textbf{Type: }

Use one of the following codes:

R: Document, report (excluding the periodic and final reports)

DEM: Demonstrator, pilot, prototype, plan designs

DEC: Websites, patents filing, press \& media actions, videos, etc.

DATA: Data sets, microdata, etc.

DMP: Data management plan

ETHICS: Deliverables related to ethics issues.

SECURITY: Deliverables related to security issues

OTHER: Software, technical diagram, algorithms, models, etc.

\textbf{Dissemination level: }

Use one of the following codes:

PU -- Public, fully open, e.g. web (Deliverables flagged as public will
be automatically published in CORDIS project's page)

SEN -- Sensitive, limited under the conditions of the Grant Agreement

Classified R-UE/EU-R -- EU RESTRICTED under the Commission Decision
No2015/444

Classified C-UE/EU-C -- EU CONFIDENTIAL under the Commission Decision
No2015/444

Classified S-UE/EU-S -- EU SECRET under the Commission Decision
No2015/444

\textbf{Delivery date}

Measured in months from the project start date (month 1) & & & & & & \\
& & & & & & \\
\bottomrule
\end{longtable}

\textbf{Table 3.1d: List of milestones }

\textbf{Table 3.1e: Critical risks for implementation }

\begin{longtable}[]{@{}
  >{\raggedright\arraybackslash}p{(\columnwidth - 4\tabcolsep) * \real{0.3333}}
  >{\raggedright\arraybackslash}p{(\columnwidth - 4\tabcolsep) * \real{0.3333}}
  >{\raggedright\arraybackslash}p{(\columnwidth - 4\tabcolsep) * \real{0.3333}}@{}}
\toprule
\endhead
\textbf{Description of risk (indicate level of (i) likelihood, and (ii)
severity: Low/Medium/High)} & \textbf{Work package(s) involved} &
\textbf{Proposed risk-mitigation measures} \\
& & \\
& & \\
& & \\
& & \\
& & \\
& & \\
\textbf{Definition critical risk: }

A critical risk is a plausible event or issue that could have a high
adverse impact on the ability of the project to achieve its objectives.

\textbf{Level of likelihood to occur: Low/medium/high}

The likelihood is the estimated probability that the risk will
materialise even after taking account of the mitigating measures put in
place.

\textbf{Level of severity: Low/medium/high}

The relative seriousness of the risk and the significance of its effect.
& & \\
& & \\
\bottomrule
\end{longtable}

\textbf{Table 3.1f: Summary of staff effort}

\emph{Please indicate the number of person/months over the whole
duration of the planned work, for each work package, for each
participant. Identify the work-package leader for each WP by showing the
relevant person-month figure in bold.}

\begin{longtable}[]{@{}
  >{\raggedright\arraybackslash}p{(\columnwidth - 8\tabcolsep) * \real{0.2000}}
  >{\raggedright\arraybackslash}p{(\columnwidth - 8\tabcolsep) * \real{0.2000}}
  >{\raggedright\arraybackslash}p{(\columnwidth - 8\tabcolsep) * \real{0.2000}}
  >{\raggedright\arraybackslash}p{(\columnwidth - 8\tabcolsep) * \real{0.2000}}
  >{\raggedright\arraybackslash}p{(\columnwidth - 8\tabcolsep) * \real{0.2000}}@{}}
\toprule
\endhead
& \textbf{WPn} & \textbf{WPn+1} & \textbf{WPn+2} & \textbf{Total
Person-}

\textbf{Months per Participant} \\
& & & & \\
\textbf{Participant Number/Short Name} & & & & \\
& & & & \\
\textbf{Participant Number/}

\textbf{Short Name} & & & & \\
& & & & \\
\textbf{Participant Number/}

\textbf{Short Name} & & & & \\
& & & & \\
\textbf{Total Person Months} & & & & \\
& & & & \\
\bottomrule
\end{longtable}

\textbf{Table 3.1g: `Subcontracting costs' items }

For each participant describe and justify the tasks to be subcontracted
(please note that core tasks of the project should not be
sub-contracted).

\begin{longtable}[]{@{}lll@{}}
\toprule
\endhead
\textbf{Description of tasks and justification} & & \\
& & \\
\textbf{Subcontracting } & & \\
& & \\
\bottomrule
\end{longtable}

\textbf{Table 3.1h: `Purchase costs' items (travel and subsistence,
equipment and other goods, works and services) }

Please complete the table below for each participant if the purchase
costs (i.e. the sum of the costs for 'travel and subsistence',
`equipment', and `other goods, works and services') exceeds 15\% of the
personnel costs for that participant (according to the budget table in
proposal part A). The record must list cost items in order of costs and
starting with the largest cost item, up to the level that the remaining
costs are below 15\% of personnel costs.

\begin{longtable}[]{@{}lll@{}}
\toprule
\endhead
\textbf{Participant Number/Short Name} & & \\
& & \\
& \textbf{Cost (€)} & \textbf{Justification} \\
& & \\
\textbf{Travel and subsistence } & & \\
\textbf{Equipment } & & \\
\textbf{Other goods, works and services} & & \\
& & \\
\textbf{Remaining purchase costs (\textless{}15\% of pers. Costs)} &
& \\
& & \\
\textbf{Total} & & \\
& & \\
\bottomrule
\end{longtable}

\textbf{Table 3.1i: `Other costs categories' items (e.g. internally
invoiced goods and services)}

Please complete the table below for each participants that would like to
declare costs under other costs categories (e.g. internally invoiced
goods and services), irrespective of the percentage of personnel costs.

\begin{longtable}[]{@{}lll@{}}
\toprule
\endhead
\textbf{Participant Number/Short Name} & & \\
& & \\
& \textbf{Cost (€)} & \textbf{Justification} \\
& & \\
\textbf{Internally invoiced goods and services} & & \\
\textbf{\ldots{}} & & \\
& & \\
\bottomrule
\end{longtable}

\textbf{Table 3.1j: `In-kind contributions' provided by third parties}

Please complete the table below for each participants that will make use
of in-kind contributions (non-financial resources made available free of
charge by third parties). In kind contributions provided by third
parties free of charge are declared by the participants as eligible
direct costs in the corresponding cost category (e.g. personnel costs or
purchase costs for equipment).

\textbf{STANDARD MODULAR EXTENSION OF PROPOSAL TEMPLATE:}

\begin{enumerate}
\def\labelenumi{\arabic{enumi}.}
\item
  \textbf{FINANCIAL SUPPORT TO THIRD PARTIES}

  \begin{itemize}
  \item
    \textbf{PART A: No additions}
  \item
    \textbf{PART B: Add an additional annex with information on
    financial support to third parties}
  \end{itemize}
\end{enumerate}

\textbf{Financial support to third parties}

\emph{For more information on terms and conditions: see Work Programme
General Annexes section B and Horizon Europe Model Grant Agreement
Articles 6.2.D.1 and 9.4 }

\emph{{[}OPTION financial support in the form of a grant:}\textbf{ }\\
\textbf{Financial support in the form of a grant awarded after a call
for proposals}

Where this possibility is indicated under the relevant topic in the Work
Programme and in the relevant calls for proposals, provide a description
of the use of financial support to third parties. This description must
address at least the following:

\begin{enumerate}
\def\labelenumi{\arabic{enumi}.}
\item
  clearly detail the objectives and the results to be obtained and
\item
  contain the following specifications (as a minimum):
\end{enumerate}

\begin{enumerate}
\def\labelenumi{\alph{enumi}.}
\item
  the maximum amount of financial support for each third party; this
  amount may not exceed 60 000 EUR, unless explicitly mentioned in the
  work programme topic
\item
  the criteria for calculating the exact amount of the financial support
\item
  the different types of activity that qualify for financial support, on
  the basis of a closed list
\item
  the persons or categories of persons that may receive financial
  support, and
\item
  the criteria for giving financial support
\end{enumerate}

Please check in the Work Programme and call for proposals if there are
other conditions that apply and, if so, include them in the
specifications or in any other element of the proposal as appropriate.

\emph{{]}}

\emph{{[}OPTION financial support in the form of a prize: }\\
\textbf{Financial support in the form of a prize}

Where this possibility is indicated under the relevant topic in the Work
Programme, provide a description of the use of financial support to
third parties. This description must address at least the following:

\begin{enumerate}
\def\labelenumi{\arabic{enumi}.}
\item
  clearly detail the objectives and the results to be obtained and
\item
  contain the following specifications (as a minimum):
\end{enumerate}

\begin{enumerate}
\def\labelenumi{\alph{enumi}.}
\item
  the eligibility and award criteria
\item
  the amount of the prize and
\item
  the payment arrangements.
\end{enumerate}

Please check in the Work Programme and the call for proposals if the are
other conditions that apply and, if so, include them in the
specifications or in any other element of the proposal as appropriate.

\emph{{]}}

\begin{enumerate}
\def\labelenumi{\arabic{enumi}.}
\item
  \textbf{CLINICAL TRIALS}

  \begin{itemize}
  \item
    \textbf{PART A: Additional question}
  \item
    \textbf{PART B: Add an additional annex with information on clinical
    trials }
  \end{itemize}
\end{enumerate}

\begin{enumerate}
\def\labelenumi{\arabic{enumi}.}
\item
  \textbf{CALLS FLAGGED AS SECURITY SENSITIVE}

  \begin{itemize}
  \item
    \textbf{PART A: No additions}
  \item
    \textbf{Part B: Add an additional
    }shapeType202fFlipH0fFlipV0rotation-2949120fLockRotation0fLockAspectRatio0fLockPosition0fLockAgainstSelect0fLockVerticies0fLockText0fLockAdjustHandles0fLockAgainstGrouping0lTxid1507328fRotateText0fFitShapeToText1fFilled0lineJoinStyle2fArrowheadsOK0fLine0fLockShapeType1wzNameText
    Box
    47posrelh0posrelv0metroBlobdhgt251668992fLayoutInCell0fAllowOverlap1fBehindDocument1fHidden0fLayoutInCell0Instructions,
    please remove
  \item
    \textbf{Part B: Add an additional }Instructions, please remove
  \item
    \textbf{Part B: Add an additional annex with information on security
    Proposal template Part B: technical description}
  \end{itemize}
\end{enumerate}

\textbf{\textsc{Title of the Proposal}}

\textbf{List of participants}

\begin{longtable}[]{@{}
  >{\raggedright\arraybackslash}p{(\columnwidth - 4\tabcolsep) * \real{0.3333}}
  >{\raggedright\arraybackslash}p{(\columnwidth - 4\tabcolsep) * \real{0.3333}}
  >{\raggedright\arraybackslash}p{(\columnwidth - 4\tabcolsep) * \real{0.3333}}@{}}
\toprule
\endhead
\subparagraph{Participant No. *} & \subparagraph{Participant
organisation name} & \subparagraph{Country} \\
& & \\
\subparagraph{1 (Coordinator)} & & \\
& & \\
\subparagraph{2} & & \\
& & \\
\subparagraph{3} & & \\
\subparagraph{\ldots{}} & & \\
& & \\
\bottomrule
\end{longtable}

\textbf{1. Excellence}

\textbf{1.1 Objectives and ambition }

Insert here text for your proposal

\textbf{1.2 Methodology }

Insert here text for your proposal

\textbf{2. Impact}

\textbf{2.1 Project's pathways towards impact}

Insert here text for your proposal

\textbf{2.2 Measures to maximise impact - Dissemination, exploitation
and communication}

Insert here text for your proposal

\textbf{2.3 Summary }

\textbf{KEY ELEMENT OF THE IMPACT SECTION}

\begin{longtable}[]{@{}
  >{\raggedright\arraybackslash}p{(\columnwidth - 0\tabcolsep) * \real{1.0000}}@{}}
\toprule
\endhead
\textbf{SPECIFIC NEEDS} \\
 \\
\emph{What are the specific needs that triggered this project?}

Insert here text for your proposal \\
 \\
\textbf{D \& E \& C MEASURES} \\
 \\
What dissemination, exploitation and communication measures will you
apply to the results?

Insert here text for your proposal \\
 \\
\textbf{EXPECTED RESULTS} \\
 \\
What do you expect to generate by the end of the project?

Insert here text for your proposal \\
 \\
\textbf{TARGET GROUPS} \\
 \\
\emph{Who will use or further up-take the results of the project? Who
will benefit from the results of the project? }

Insert here text for your proposal \\
 \\
\textbf{OUTCOMES} \\
 \\
\emph{What change do you expect to see after successful dissemination
and exploitation of project results to the target group(s)?}

Insert here text for your proposal \\
 \\
\textbf{IMPACTS} \\
 \\
\emph{What are the expected wider scientific, economic and societal
effects of the project contributing to the expected impacts outlined in
the respective destination in the work programme?}

Insert here text for your proposal \\
 \\
\bottomrule
\end{longtable}

\textbf{3. Quality and efficiency of the implementation}

\textbf{3.1 Work plan and resources }

Insert here text for your proposal

\textbf{3.2 Capacity of participants and consortium as a whole}

Insert here text for your proposal

\textbf{Tables for section 3.1}

\textbf{Table 3.1a: List of work packages}

\begin{longtable}[]{@{}lllllll@{}}
\toprule
\endhead
\textbf{Work package No} & \textbf{Work Package Title} & \textbf{Lead
Participant No} & \textbf{Lead Participant Short Name} &
\textbf{Person-Months} & \textbf{Start Month} & \textbf{End month} \\
& & & & & & \\
& & & & & & \\
& & & & & & \\
& & & & & & \\
& & & & & & \\
& & & & & & \\
& & & & Total person- months & & \\
& & & & & & \\
\bottomrule
\end{longtable}

\textbf{Table 3.1b: Work package description }

\textbf{For each work package: }

\begin{longtable}[]{@{}llllllll@{}}
\toprule
\endhead
\textbf{Work package number } & & \textbf{Lead beneficiary} & & & & & \\
& & & & & & & \\
\textbf{Work package title} & & & & & & & \\
& & & & & & & \\
\textbf{Participant number} & & & & & & & \\
\textbf{Short name of participant} & & & & & & & \\
\textbf{Person months per participant:} & & & & & & & \\
& & & & & & & \\
\textbf{Start month} & & \textbf{End month} & & & & & \\
& & & & & & & \\
\textbf{Objectives} & & & & & & & \\
\textbf{Description of work} (where appropriate, broken down into
tasks), lead partner and role of participants & & & & & & & \\
\textbf{Deliverables} (brief description and month of delivery) & & & &
& & & \\
& & & & & & & \\
\bottomrule
\end{longtable}

\textbf{Table 3.1c: List of Deliverables }

\begin{longtable}[]{@{}
  >{\raggedright\arraybackslash}p{(\columnwidth - 12\tabcolsep) * \real{0.1429}}
  >{\raggedright\arraybackslash}p{(\columnwidth - 12\tabcolsep) * \real{0.1429}}
  >{\raggedright\arraybackslash}p{(\columnwidth - 12\tabcolsep) * \real{0.1429}}
  >{\raggedright\arraybackslash}p{(\columnwidth - 12\tabcolsep) * \real{0.1429}}
  >{\raggedright\arraybackslash}p{(\columnwidth - 12\tabcolsep) * \real{0.1429}}
  >{\raggedright\arraybackslash}p{(\columnwidth - 12\tabcolsep) * \real{0.1429}}
  >{\raggedright\arraybackslash}p{(\columnwidth - 12\tabcolsep) * \real{0.1429}}@{}}
\toprule
\endhead
\textbf{Deliverable (number)} & \textbf{Deliverable name} & \textbf{Work
package number } & \textbf{Short name of lead participant } &
\textbf{Type} & \textbf{Dissemination level} & \textbf{Delivery date}

\textbf{(in months)} \\
& & & & & & \\
& & & & & & \\
& & & & & & \\
& & & & & & \\
& & & & & & \\
& & & & & & \\
& & & & & & \\
\bottomrule
\end{longtable}

\textbf{Table 3.1d: List of milestones }

\begin{longtable}[]{@{}lllll@{}}
\toprule
\endhead
\textbf{Milestone number} & \textbf{Milestone name} & \textbf{Related
work package(s)} & \textbf{Due date (in month)} & \textbf{Means of
verification} \\
& & & & \\
& & & & \\
& & & & \\
& & & & \\
& & & & \\
& & & & \\
\bottomrule
\end{longtable}

\textbf{Table 3.1e: Critical risks for implementation }

\begin{longtable}[]{@{}lll@{}}
\toprule
\endhead
\textbf{Description of risk (indicate level of (i) likelihood, and (ii)
severity: Low/Medium/High)} & \textbf{Work package(s) involved} &
\textbf{Proposed risk-mitigation measures} \\
& & \\
& & \\
& & \\
& & \\
& & \\
& & \\
\bottomrule
\end{longtable}

\textbf{Table 3.1f: Summary of staff effort}

\begin{longtable}[]{@{}
  >{\raggedright\arraybackslash}p{(\columnwidth - 8\tabcolsep) * \real{0.2000}}
  >{\raggedright\arraybackslash}p{(\columnwidth - 8\tabcolsep) * \real{0.2000}}
  >{\raggedright\arraybackslash}p{(\columnwidth - 8\tabcolsep) * \real{0.2000}}
  >{\raggedright\arraybackslash}p{(\columnwidth - 8\tabcolsep) * \real{0.2000}}
  >{\raggedright\arraybackslash}p{(\columnwidth - 8\tabcolsep) * \real{0.2000}}@{}}
\toprule
\endhead
& \textbf{WPn} & \textbf{WPn+1} & \textbf{WPn+2} & \textbf{Total
Person-}

\textbf{Months per Participant} \\
& & & & \\
\textbf{Participant Number/Short Name} & & & & \\
& & & & \\
\textbf{Participant Number/}

\textbf{Short Name} & & & & \\
& & & & \\
\textbf{Participant Number/}

\textbf{Short Name} & & & & \\
& & & & \\
\textbf{Total Person Months} & & & & \\
& & & & \\
\bottomrule
\end{longtable}

\textbf{Table 3.1g: `Subcontracting costs' items }

\begin{longtable}[]{@{}lll@{}}
\toprule
\endhead
\textbf{Participant Number/Short Name} & & \\
& & \\
& \textbf{Cost (€)} & \textbf{Description of tasks and justification} \\
& & \\
\textbf{Subcontracting } & & \\
& & \\
\bottomrule
\end{longtable}

\textbf{Table 3.1h: `Purchase costs' items (travel and subsistence,
equipment and other goods, works and services) }

\begin{longtable}[]{@{}lll@{}}
\toprule
\endhead
\textbf{Participant Number/Short Name} & & \\
& & \\
& \textbf{Cost (€)} & \textbf{Justification} \\
& & \\
\textbf{Travel and subsistence } & & \\
\textbf{Equipment } & & \\
\textbf{Other goods, works and services} & & \\
& & \\
\textbf{Remaining purchase costs (\textless15\% of pers. Costs)} & & \\
& & \\
\textbf{Total} & & \\
& & \\
\bottomrule
\end{longtable}

\textbf{Table 3.1i: `Other costs categories' items (e.g. internally
invoiced goods and services)}

\begin{longtable}[]{@{}lll@{}}
\toprule
\endhead
\textbf{Participant Number/Short Name} & & \\
& & \\
& \textbf{Cost (€)} & \textbf{Justification} \\
& & \\
\textbf{Internally invoiced goods and services} & & \\
\textbf{\ldots{}} & & \\
& & \\
\bottomrule
\end{longtable}

\textbf{Table 3.1j: `In-kind contributions' provided by third parties}

\begin{longtable}[]{@{}
  >{\raggedright\arraybackslash}p{(\columnwidth - 6\tabcolsep) * \real{0.2500}}
  >{\raggedright\arraybackslash}p{(\columnwidth - 6\tabcolsep) * \real{0.2500}}
  >{\raggedright\arraybackslash}p{(\columnwidth - 6\tabcolsep) * \real{0.2500}}
  >{\raggedright\arraybackslash}p{(\columnwidth - 6\tabcolsep) * \real{0.2500}}@{}}
\toprule
\endhead
\textbf{Participant Number/Short Name} & & & \\
& & & \\
\textbf{Third party name} & \textbf{Category} & \textbf{Cost (€)} &
\textbf{Justification} \\
& & & \\
& \textbf{Select between}

Seconded personnel

Travel and subsistence

Equipment

Other goods, works and services

Internally invoiced goods and services & & \\
& & & \\
& & & \\
\bottomrule
\end{longtable}
