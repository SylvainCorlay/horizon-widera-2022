\eucommentary{Milestones means control points in the project that help to chart progress. Milestones may
correspond to the completion of a key deliverable, allowing the next phase of the work to begin.
They may also be needed at intermediary points so that, if problems have arisen, corrective
measures can be taken. A milestone may be a critical decision point in the project where, for
example, the consortium must decide which of several technologies to adopt for further
development.
}
\begin{draft}
\begin{verbatim}
* TODO
- [ ] we refer to \TheProject SERVICES a lot
- [ ] would it be better to refer to TOOLS instead?
- [ ] we should probably list the services / tools somewhere for clarity
\end{verbatim}
\end{draft}


% \milestonetable



\begin{milestones}
  \milestone[
    id=startup,
    month=12,
    verif={
      Completed all corresponding deliverables
      and preparation for deployment of prototype services is underway
      }
    ]
  {Project startup, requirements gathering, and exploratory prototypes}
  {

  By milestone 1, we will have established the infrastructure for the project
  and begun exploratory prototyping development and deployment of services,
  engaging with the existing communities. We will have a network of advisors
  through the Community Engagement Panel and are coordinating our plans for
  \TheProject with those of wider open science communities and the Jupyter
  project. We will have preliminary study results to guide and evaluate
  improvements to reproducibility with \TheProject tools. }

  \milestone[
    id=prototype,
    month=24,
    verif={
      Developed functional prototypes for all important topics.
      Early users are able to access and test prototype services
    }
    ]
  {Prototype demonstrator services}
  {

  By this point, prototype demonstrator services will be useful and accessible
  to a broad range of users, and we will have begun to experiment with early-adopter
  users and local demonstrators to guide further development of \TheProject,
  ensuring that development serves the reproducibility needs of the global science community.

  There will be improvements to reproducibility.
  }

  \milestone[
    id=final,
    month=36,
    verif={Refactored and stabilised prototypes.}
    ]
  {Project conclusion}
  {
  At the end of the project,
  we will have engaged with the science communities to evaluate demonstrator services, and
  identified which services and tools shall be sustained beyond the life of the project.
  We will have made those robust and maintainable from a software engineering point of view.

  We will have training materials and have run workshops to train users in Reproducibility and Open Science,
  making use of tools and services developed through \TheProject.

  We will have developed a sustainability plan for how future maintenance and
  development of \TheProject tools be achieved under community support and
  leadership.

  \TheProject tools are more reliable and useful to a broader audience than at
  the start of the project.

  }


\end{milestones}
