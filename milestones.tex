\eucommentary{Milestones means control points in the project that help to chart progress. Milestones may
correspond to the completion of a key deliverable, allowing the next phase of the work to begin.
They may also be needed at intermediary points so that, if problems have arisen, corrective
measures can be taken. A milestone may be a critical decision point in the project where, for
example, the consortium must decide which of several technologies to adopt for further
development.}

\milestonetable



\paragraph{General Milestones}

\begin{milestones}
  \milestone[
    id=startup,
    month=12,
    verif={
      Completed all corresponding deliverables
      and preparation for deployment of prototype services is underway
      }
    ]
  {Startup, requirements, and prototype generic Jupyter service}
  {
  By milestone 1, we will have established the infrastructure
  for the project and begun prototyping development and deployment of services,
  engaging with the existing communities,
  coordinating plans for \TheProject with those of the wider Jupyter and open science communities.
  We should have preliminary study results to guide and evaluate improvements to reproducibility
  with \TheProject tools.
  }

  \milestone[
    id=prototype,
    month=24,
    verif={
      Completed all corresponding deliverables and early users are able to access and test prototype services
    }
    ]
  {Prototype demonstrator services}
  {
  By this point, prototype demonstrator services should be useful and accessible
  to a broad range of users, and begun to experiment with early-adopter
  users and local demonstrators to guide further development of \TheProject,
  ensuring that development serves the needs of the community.
  Improvements to reproducibility
  }

  \milestone[
    id=final,
    month=36,
    verif={Completed all corresponding deliverables and reported progress at the final project review}
    ]
  {Project conclusion}
  {
  At the end of the project,
  we will have engaged with the community to evaluate demonstrator services,
  identified which services and tools shall be sustained beyond the life of the project,
  and developed a sustainability plan for how this may be achieved under community support and leadership.
  We will have training materials and have run workshops to train users in Open Science,
  making use of operational \TheProject services.
  \TheProject tools are more reliable and useful to a broader audience than the start of the project.
  }


\end{milestones}
