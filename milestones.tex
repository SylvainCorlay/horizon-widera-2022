\subsubsection{Milestones}\label{sec:milestones}

\eucommentary{Milestones means control points in the project that help to chart progress. Milestones may
correspond to the completion of a key deliverable, allowing the next phase of the work to begin.
They may also be needed at intermediary points so that, if problems have arisen, corrective
measures can be taken. A milestone may be a critical decision point in the project where, for
example, the consortium must decide which of several technologies to adopt for further
development.
}

\begin{draft}
\begin{verbatim}
TODO:
- [x] sort milestones
- [ ] check dates
- [ ] omit descriptions? Template doesn't have any
- [x] involved WP: both input and output, or just input?
\end{verbatim}
\end{draft}

\begin{milestones}
  \milestone[
    id=conda-time,
    month=6,
    wps={reproducibility},
    verif={Feature available in conda/mamba software},
    ]
  {Select conda packages by date}
  {
  The conda/mamba package manager shall be able to
  select packages for installation based on a given date.
  Necessary for repo2docker to best take time into account
  when creating a software environment.
  }

  \milestone[
    id=study,
    month=12,
    wps={reproducibility,education},
    verif={Report produced},
    ]
  {Reproducibility study and evaluation tool}
  {
  We will have preliminary study results and an associated tool,
  regarding the reproducibility of repositories with repo2docker.
  These results will inform future development of the tools in WP2,
  as well as best practices resources and education in WP5.
  }

  \milestone[
    id=prototype,
    month=12,
    wps={applications,impact},
    verif={
      Deployed first functional prototypes of science demonstrators.
      Early users are able to access and test prototype services
    }
    ]
  {Prototype demonstrator services}
  {
  By this point, prototype demonstrator services will be useful and accessible
  to a broad range of users, and we will have begun to experiment with early-adopter
  users and local demonstrators to guide further development in WP3,
  ensuring that development serves the reproducibility needs of the global science community.
  }

  \milestone[
    id=docs-online,
    month=12,
    wps={education},
    verif={Resources available from project website},
    ]
  {Draft best practices documentation}
  {
  Draft version of documentation for best practices is online.
  Required starting point for education tasks in WP5.
  }

  \milestone[
    id=rm-docker,
    month=15,
    wps={impact,applications},
    verif={Feature available in repo2docker software},
    ]
  {Support for alternative container technologies in repo2docker for suitability in HPC}
  {
  The Docker container runtime is not suitable in all cases.
  In order to proceed with some demonstrators in WP4,
  we must ensure compatibility with container runtimes supported by our HPC providers,
  such as Singularity.
  }

  \milestone[
    id=repo2docker-time,
    month=18,
    wps={reproducibility},
    verif={Feature available in repo2docker software},
    ]
  {repo2docker takes publication time into account}
  {
  By taking publication time into account,
  repo2docker will reproduce environments with higher fidelity,
  especially when environments are not fully or strictly specified.
  }

  \milestone[
    id=data-publishing,
    month=18,
    wps={impact,applications},
    verif={Demonstrated example deployment},
    ]
  {Practical support for authenticated data publishing}
  {
  It shall be practical to deploy BinderHub with performant, authenticated access to large datasets,
  required for some advanced science demonstrators in WP4.
  }

  \milestone[
    id=repo2docker-improved,
    month=24,
    wps={reproducibility},
    verif={Delivered in repo2docker software; Repeat study, comparing baseline results form start of project},
    ]
  {repo2docker produces robust computational environments}
  {
  Taking input from earlier study and tests,
  repo2docker has been improved to produce environments more reliably and robustly,
  as verified by a comparison study with the baseline at the beginning of the project.
  }

\end{milestones}

\milestonetable
