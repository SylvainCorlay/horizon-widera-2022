\TOWRITE{ALL}{Proofread WP 1 Management pass 1}
\begin{draft}
\TOWRITE{PS (Work Package Lead)}{For WP leaders, please check the following (remove items
  once completed)}
\begin{verbatim}
* TODO items (27 March 2022)
- [ ] check links to local deliverables
- [ ] Need data management plan
- [X] do we need an IPR management plan?
      - I don't think so (HF). The IPR management needs to be part of the
        consortium agreement.
- [ ] do we need a Innovation Management Task? It is commented out for now. As
      we are open source all the way through, we don't need to worry about IPR so
      much, and we also don't need to establish a common understanding.
- [ ] have only one data management plan: it should not be hard to decide to
make everything open source. (i.e. get rid of 'draft' plan at M1)
- [ ] update mile stone names so this WP can be completed
- [ ] should we have a \Binder command?
- [ ] Is 'Binder' the right name to refer to? (Should be consistent throughout.)
- [ ] We mention here that all outputs will be open source (technical
      management). This should maybe go elsewhere?
- [ ] should also check with collaborators that they are happy with this.
\end{verbatim}
\end{draft}

% (12+4)/36 = 0.4444444444444444 -> !.44
\begin{workpackage}[id=management,type=MGT,wphases=0-36!.44,
  title=Project Management,
  short=Management,
  lead=SRL,
  MPRM=1,
  QSRM=1,
  IFRRM=1,
  UIORM=1,
  SRLRM=12,
  swsites
]
\begin{wpobjectives}
  The main objective of WP1 is to establish and maintain an effective contract,
  project, and operational management approach ensuring:

 \begin{compactitem}
 \item Timely and successful implementation of the project; including
   administrative and legal coordination
    \item Technical management and quality assurance
    \item Risk and innovation management of the project as a whole; including
      data and IPR management
    \item Smooth communication and interaction with the EC and other interested
      parties
 \end{compactitem}
\end{wpobjectives}

\begin{wpdescription}
  The project will be managed by Simula, which has extensive experience in
  administering and leading EU funded and national projects. The coordinator
  together with the WP leaders, will be responsible for monitoring WP status,
  coordination of work plan updates and annual internal progress reports.
  \TOWRITE{}{Management details section has been removed}
  % The project management structure and roles of partners in the consortium are
  % presented in \ref{sect:mgt}.

\end{wpdescription}

\begin{tasklist}

\begin{task}[
  title=Administrative Management,
  id=admin,
  lead=SRL,
  PM=6,
  wphases={0-36!.166},
  partners={MP,QS,UIO,IFR},
]
The task includes the following activities:
\begin{compactenum}
\item Preparation, distribution and maintenance of all contractual documents
  (Consortium Agreement, Grant Agreement and all other legal frameworks)
\item Establishment of appropriate communication and collaborative environment
  for the consortium, as well as the EC and other relevant academic and industry
  stakeholders (the project website, intranet and communication procedures) to
  organise transfer of knowledge, present and promote project results
  (\localdelivref{infrastructure});
\item Organisation of project review and progress meetings;
\item Performing qualitative and quantitative risk analysis, planning risk
  mitigation and control
\item Progress and Financial Reporting to the EC;
\item Data and IPR Management will be managed in accordance with agreed rules
  stated in the Consortium Agreement and in accordance with the Data Management
  Plan (\localdelivref{data-management-plan}.
\end{compactenum}
\end{task}

\begin{task}[
  title=Technical Project Management,
  id=project-management,
  lead=SRL,
  PM=6,
  wphases={0-36!.166},
  partners={MP,QS,UIO,IFR}
  ]
  The task includes the following activities:
  \begin{compactenum}
\item The scientific and technical management to ensure coherent quality and
  soundness of the work and results.
\item Applying quality assurance measures across all partners for all tasks and
  deliverables.
\item Reporting of outcomes and quality assurance activities in technical
  reports and reviews.
\item The project coordinator, with the help of the work package leads, will
regularly review technological risks and recommend mitigation plans to minimise
or remove them. This will be reported on at each reporting period in the
project's technical report.
\item Set up and maintenance of technical management infrastructure required for
  a software project of this type, such as a web site, open source hosting of code and
  documentation, mailing lists, task trackers, automatic tests and continuous
  integration. We will feed back into existing open source repositories projects
  where they exist already, and make use of commonly used tools and services
  such as GitHub. All outputs will be published under an open source license.
\end{compactenum}
\end{task}

\begin{task}[
  title=Management of dissemination and communication activities,
  id=website,
  lead=SRL,
  PM=2,
  % >>> 2/36 = 0.05555555555555555
  wphases={0-36!.056},
  partners={}
]

This task comprises the management and administrative aspects of all forms of
direct dissemination and public communication activities such as press releases,
scientific and technical publications, seminars, talks, promotion through social
media, creation of advertisement materials such as flyers, posters, and
electronic feeds as well as their distribution. We will use standard community
building technology such as mailing lists, wikis and forums, to ensure
dissemination to and engagement with the user community.
\end{task}


% \begin{task}[
%   title=Innovation Management,
%   id=innovation-management,
%   lead=SRL,
%   PM=4,
%   wphases={0-36!.111},
%   partners={MP,QS,UIO,IFR}
% ]
% One of the most important criteria for success for the \TheProject project is to
% bring the project results into use. Therefore, exploitation routes will be
% sought whenever possible. In order to create a common understanding within the
% Consortium of how we can best shepherd an idea all the way from conception to
% realisation and exploitation, the Coordinator will be responsible for the
% preparation and realisation of an Innovation Plan. This plan will assure that
% research activities meet the required milestones and produce outputs fully
% aligned with the project objectives. All research activities will go through an
% initial process where the exploitation opportunity is identified along with the
% main stakeholders for the exploitation opportunity and an IP owner
% (\localdelivref{innovation-management-plan}).
% \end{task}

\begin{task}[
  title=Community Engagement Panel,
  id=community-engagement-panel,
  lead=SRL,
  PM=2,
  wphases={0-36!.056},
  partners={MP,QS,UIO,IFR},
  ]
  The Community Engagement Panel (CEP) is a forum to bring together
  representatives of different current and potential user communities of Binder tools.
  Through the community engagement panel we want to maximise the interaction between existing and new users.
  This will help shape the software features so that we can
  achieve the highest possible impact for reproducibility in science.

  Stakeholders for the topic of reproducibility that should be represented in the
  community engagement panel include researchers, research infrastructure
  providers, publishers, research councils, librarians, and educators.

The task includes the following activities:
\begin{compactenum}
\item Form the community engagement panel by inviting representative of relevant
  communities. Ensure that representatives from stakeholder communities include
  current and potential future Binder users.
\item Organise regular (online) community engagement panel meetings, soon after the
  beginning of the project, and subsequently at the end of years 1, 2, and 3.
\item Ensure that input and feedback from community engagement panel members are
  considered to direct the project to improve the usefulness of Binder tools
  and broaden the range of their applicability to maximise overall impact.
\item Encourage and foster voluntary collaboration and direct contributions to
  the project from the communities represented in the community engagement
  panel that go beyond the advisory role of the panel itself.
\end{compactenum}

We have already secured agreement from the following to be part of the panel,
and will extend this if funded:
\begin{itemize}
\item Suzanne Dumouchel, Head of European Cooperation at TGIR Huma-Num CNRS unit, a large infrastructure for
digital humanities and member of the EOSC Association Board of Directors. She is the partnerships coordinator
of OPERAS Research Infrastructure, devoted to scholarly communication in Social Sciences and Humanities and is a member of DARIAH ERIC Coordination Office,
dedicated to Digital Arts and Humanities. She is also the scientific coordinator of TRIPLE, H2020 project
(INFRAEOSC2). Strongly committed to the Open Science movement and to the promotion of research in Social Sciences
and Humanities (SSH), she is particularly active in the field of research infrastructures.
\item Andy Götz, Software Group Leader at European Radiation Synchrotron
  Facility (ESRF), coordinator of the EOSC project PaNOSC for making data from
  photon and neutron facilities FAIR, and chairman of the IT working Group of
  the ``League of European Accelerator-based Photon Sources'' (LEAPS). The LEAPS
  facilities wish to enable their users to create reproducible publications
  based on large data sets captured at the light sources.
\item Paula Andrea Martinez, Project Coordinator - Software Program, \href{https://ardc.edu.au/}{Australian Research Data Commons} (ARDC).
She is also the co-chair of the \href{https://www.rd-alliance.org/groups/fair-research-software-fair4rs-wg}{FAIR4RS RDA Working Group} and
Community Manager at \href{https://www.researchsoft.org/}{Research Software Alliance} (ReSA), and actively contributing
to increase the visibility of research software.
\item Aleksandra Nenadic, Training Lead of the Software Sustainability Institute,
based at the University of Manchester (UK). She is also an active member and
promoter of the Carpentries community and involved as an instructor,
instructor trainer, mentor, workshop organiser and regional coordinator
for the UK, driving and supporting new material creation using the
Carpentries collaborative and pedagogical lesson development principles.
\item Gergely Sipos, head of services, solutions and support department at the
  EGI Foundation. He is representing EGI, ``Advanced Computing for EOSC''
  (EGI-ACE) and the EOSC Compute Platform, which are working on a large-scale
  deployment of BinderHub as part of their services for researchers in Europe
  and beyond.
\item Violaine Louvet, head of GRICAD (Grenoble Alpe Research -
 Scientific Computing and Data Infrastructure), supported by
 CNRS, Grenoble Alpes University and INRIA. GRICAD is a Tier 2
 infrastructure and provides data and computing resources to all the
 science communities in Grenoble. In particular, GRICAD provides HPC,
 HTC, cloud and storage resources for all the disciplinary fields,
 from computer sciences to human sciences and health. GRICAD
 also offers a JupyterHub and a BinderHub platform. We are also very
 involved in helping the scientific communities and in training activities.
 \item Rollin Thomas, Big Data Architect at HPC expert at the National Energy
  Research Scientific Computing Center at Lawrence Berkeley National Laboratory
  (US). He represents the HPC community, and focuses on interactivity,
  real-time, and reproducibility in supercomputing for science.
\item Andreas Zeller, Professor of Software Engineering at Saarland University. He uses Notebooks
  to provide open-source text books to his students and the world-wide
  community of readers. He will represent Binder users in academia, who use it
  to deliver zero-install computational environments for educational
  purposes.
\end{itemize}

\end{task}
\end{tasklist}


\begin{wpdelivs}

\begin{wpdeliv}[due=2,miles=startup,id=infrastructure,dissem=PU,nature=DEC,lead=SRL]
  {Basic project infrastructure (web site, mailing lists, issue trackers, mailing lists, repositories)}
\end{wpdeliv}

% “Proposals selected for funding under Horizon Europe will need to develop a
% detailed data management plan (DMP) for making their data/research outputs
% findable, accessible, interoperable and reusable (FAIR) as a deliverable by
% month 6 and revised towards the end of a project’s lifetime.”
\begin{wpdeliv}[due=6,miles=startup,id=data-management-plan,dissem=PU,nature=R,lead=SRL]
  {Data Management Plan}
\end{wpdeliv}

\begin{wpdeliv}[due=36,miles=final,id=data-management-plan-revised,dissem=PU,nature=R,lead=SRL]
  {Revised Data Management Plan}
\end{wpdeliv}

% \begin{wpdeliv}[due=9,miles=startup,id=innovation-management-plan,dissem=CO,nature=R,lead=SRL]
%   {Innovation Management Plan}
% \end{wpdeliv}

\end{wpdelivs}
\end{workpackage}
%%% Local Variables:
%%% mode: latex
%%% TeX-master: "../proposal"
%%% End:

%  LocalWords:  workpackage wphases wpobjectives wpdescription pageref wpdelivs wpdeliv
%  LocalWords:  dissem mailinglists swrepository final-mgt-rep compactitem swsites ipr
%  LocalWords:  TOWRITE tasklist delivref
