\TOWRITE{ALL}{Proofread WP 1 Management pass 1}
\begin{draft}
\TOWRITE{PS (Work Package Lead)}{For WP leaders, please check the following (remove items
once completed)}
\begin{verbatim}
- [ ] have all the tasks in this Work Package a lead institution?
- [ ] have all deliverables in the WP a lead institution?
- [ ] do all tasks list all sites involved in them?
- [ ] does the table of sites and their PM efforts match lists of sites for each task?
      (each site from the table is listed in all relevant tasks, and no site is listed
      only in the table or only at some task)
\end{verbatim}
\end{draft}

\begin{workpackage}[
  id=impact,
  wphases=0-36,
  swsites,
  title=Broadening impact,
  short=Impact,
  lead=SRL,
  % EGIRM=4,
  % INSERMRM=4,
  % EPRM=21,
  % QSRM=20,
  % SILRM=4,
  SRLRM=30,
  % % UIORM=4,
  % UPSUDRM=2,
  % WTTRM=6,
  % XFELRM=54,
  % EPRM=20,
]
\begin{wpobjectives}
 \begin{compactitem}
   \item develop projects for creating open science services built out of Jupyter components and exploring new models for such services
   \item develop workflows for data science using Jupyter software
 \end{compactitem}
\end{wpobjectives}

\begin{wpdescription}

\begin{compactitem}
   \item BinderHub
   \item Relax kubernetes
   \item binder@home
   \item new buildpacks
   \item data access
 \end{compactitem}

% Open source software in general, and Jupyter in particular,
% is developed not as a monolithic application,
% but rather as a collection of related components,
% which can be assembled in numerous combinations to meet diverse needs.
% The Jupyter community is no different.
% Jupyter itself is composed of several projects,
% but there are even more projects that build on top of Jupyter to create
% things like cloud services or data pipelines.
% The goal of \TheProject is to facilitate open science through Jupyter,
% and this includes working with projects all around the Jupyter ecosystem.
% We will focus this work package on developing
% Jupyter ecosystem projects with an emphasis on open science.
%
% repo2docker is a project for creating
% reproducible environments in which Jupyter notebooks (and other user interfaces) can be run.
% It reads a number of common formats to list required software packages,
% and prepares a Docker container with those packages installed.
% BinderHub is software for operating a web service using repo2docker,
% which enables sharing of interactive and reproducible Jupyter (and Rstudio) environments on the web with a single link.
% We will develop repo2docker and BinderHub further to meet the needs of the open science community.
%
% In addition to the interactive aspects of Jupyter,
% notebooks can be used in a "workflows" style,
% where job systems run analyses and produce reports,
% either on a scheduled basis or triggered by events.
% There is a great deal of interest in using notebooks in this way,
% and much room for development of tools supporting workflows in data-driven open science.

\end{wpdescription}

\begin{tasklist}
% add tasks from task directory here
% template for a task
% each task should be added to exactly one workpackage
% in the workpackage task list
\begin{task}[
  title=Reducing technical constraints to enable broader usage,
  % task id for references
  id=constraints,
  % lead institution ID
  lead=SRL,
  PM=14,
  wphases={0-36},
  % partner institution ID(s)
  % don't include lead here
  partners={MP,QS}
]

In this task, we refactor and extend the existing code to be more flexible. In
particular, we want to address and remove the constraints that currently exist:

\begin{compactitem}
\item Dependency of existing Kubernetes system: at the moment, BinderHub can
  only start container environments within a Kubernetes installation. In this
  task, we will make it possible to start a container directly on the host
  machine. Setting up Kubernetis system is a complex task: while justified to
  exploit large computational resources through swarms of containers
  effectively, it is not necessary for single-machine execution of
  Binder-reproduced environments (such as anticipated for Binder@home in
  \taskref{applications}{binder-at-home}).
\item Support of only Docker containers to host created software environments.
  Docker is widespread and popular, and in particular has an attractive user
  interface for Windows and OSX. However, many HPC centres refuse to allow use
  of Docker containers on their systems for security reasons. In this task, we
  will make it possible to user container technologies, which are more
  acceptable to use on large HPC installations (such as singularity for
  example).
     % deliverable will be defined in the appropriate WorkPackage.tex
    % (\localdelivref{deliv-id})
  \end{compactitem}

  These steps are essential to enable the Binder@home
  (\taskref{applications}{binder-at-home}) and the Binder@HPC
  (\taskref{applications}{binder-at-hpc}) use cases.

%   The task includes the following activities
%   \begin{compactitem}
%   \item ...
%      % deliverable will be defined in the appropriate WorkPackage.tex
%     % (\localdelivref{deliv-id})
%   \end{compactitem}
\end{task}

% template for a task
% each task should be added to exactly one workpackage
% in the workpackage task list
\begin{task}[
  title=Support more use patterns,
  % task id for references
  id=patterns,
  % lead institution ID
  lead=SRL,
  PM=1,
  wphases={0-36},
  % partner institution ID(s)
  % don't include lead here
  partners={MP}
]
  The task includes the following activities
  \begin{compactitem}
  \item Data access (BinderHub deployment, possibly mostly docs)
  \item Restrict access to BinderHub through authentication
    - useful for many institutes who want to offer reproducibility services but
    restrict their usage
  \end{compactitem}
\end{task}

\end{tasklist}


\begin{wpdelivs}
\begin{wpdeliv}[
    % id for linking with \delivref or \localdelivref
    id=deliv,
    % lead institution
    lead=XXX,
    % month when deliverable is due (max 36)
    due=12,
    % associated milestone id (see milestones.tex)
    miles=startup,
    % ~always PU, DEC
    dissem=PU,
    nature=DEC,
]
  {
  One-line name of deliverable
  }
\end{wpdeliv}

\end{wpdelivs}
\end{workpackage}
%%% Local Variables:
%%% mode: latex
%%% TeX-master: "../proposal"
%%% End:

%  LocalWords:  workpackage wphases wpobjectives wpdescription pageref wpdelivs wpdeliv
%  LocalWords:  dissem mailinglists swrepository final-mgt-rep compactitem swsites ipr
%  LocalWords:  TOWRITE tasklist delivref
