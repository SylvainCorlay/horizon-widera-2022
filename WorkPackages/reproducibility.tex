
\begin{workpackage}[
  id=reproducibility,
  % wphases=0-36!1.03,
  wphases={0-6!1.14,6-24!0.73,24-36!0.1},
  title=Improving robustness of reproducibility tools,
  short=Improving robustness,
  lead=QS,
  SRLRM=23,
  UIORM=0,
  MPRM=2,
  QSRM=12,
  swsites,
]
\begin{wpobjectives}
  \begin{compactitem}
    \item to better understand and evaluate successful reproduction of computational environments
    \item to improve the practical reproducibility of environments constructed
      with \TheProject tools
    \item to support and maintain core Binder software infrastructure in order to keep it healthy
         and useful for open science and reproducibility
 \end{compactitem}
\end{wpobjectives}

\begin{wpdescription}

This Work package is focused on making \repotodocker{} do the things it does
already \emph{better}, \emph{more robustly} and \emph{more sustainably}.
(Orthogonal to those improvements, we plan to significantly extend the
\repotodocker{} use cases in \WPref{impact}.)

To be able to asses the impact of our planned improvements, we need to have a
metric. Task \localtaskref{repo2docker-checker} will create this
for us. In addition to the evaluation of the improvements in this proposal, this
can be used more generally as an indicator for reproducibility of software
environments.

One major improvement to the existing capabilities of \repotodocker is the
\emph{time-machine} functionality, and this is implemented in \localtaskref{repo2docker-timemachine}.

In task \localtaskref{performance-optimisation}, we will speed-up the execution time of
\repotodocker{} to improve the user experience when reproducing or re-using
existing software and data.

Open source software needs ongoing maintenance to adapt to changing requirements
and dependencies. We schedule a certain amount of time for this in task
\localtaskref{maintenance}.

All changes to the software will be made available online as open source already during development
(i.e. throughout the whole project), and new features will be made available
through software releases of the Binder tools. A final release will be made and
reported through the deliverable \localdelivref{repo2docker-release24}.

% the existing functionality of repodocker.
% Community-led open source software is critical to a sustainable future for open science.
% Commonly used tools make up a shared infrastructure,
% where investment in core components benefits the widest user community.
% \TheProject is centred around the Jupyter project,
% which is a collection of projects for interactive computing and
% communicating computational ideas.
%
% This work package is focused on developing and maintaining
% the core of Jupyter.
% In particular, we will help maintain these projects to meet the needs of the
% Jupyter community, with a focus on needs for open science.
% To serve the needs of \TheProject,
% Jupyter core infrastructure will need improvements
% to security, performance, and scalability,
% which will be provided in \localtaskref{maintenance}.
% In addition, we will develop new features in the core of Jupyter
% to bring it to a wider audience,
% and to improve its usefulness to those working toward open science practices,
% including via collaboration features (\localtaskref{collaboration})
% and accessibility (\localtaskref{accessibility}).

\end{wpdescription}

\begin{tasklist}

% template for a task
% each task should be added to exactly one workpackage
% in the workpackage task list
\begin{task}[
  title=Towards quantifiable progress for reproducible software environments,
  % task id for references
  id=repo2docker-checker,
  % lead institution ID
  lead=SRL,
  PM=10,
  % wphases={0-24!0.42},
  % partner institution ID(s)
  % don't include lead here
  partners={MP}
  ]
  The \repotodocker{} tool is a key component of the Binder software for
  reproducibility (see \ref{binder-how-does-it-work}). It can be used to create
  a software environment based on software dependency specification standards
  (see \ref{sec:repo2docker}) that are widely used.

  If the required software is specified -- for example through a
  \texttt{requirements.txt} file for Python dependencies -- then \repotodocker{}
  can create the software environment (currently limited to such environments in
  Docker images), within in which the main computation or data analysis can be
  reproduced.

  In this task, we will develop a tool -- with working name
  \softwarename{repo2docker-checker} -- that allows us to \emph{automatically}
  assess the reproducibility of software environments for software that is
  publicly available on GitHub, Bitbucket or GitLab repositories.

  For every repository, the \softwarename{repo2docker-checker} tool will report if an
  appropriate software environment could be produced, or if a problem occurred. Software
  environments in repositories may be reproducible because the authors already
  use Binder to offer their repository in an interactive Binder environment. Or
  the software environment may be reproducible because the authors have followed
  standard conventions understood by \repotodocker{}.

The task includes the following activities:
\begin{compactitem}
  \item Through manual inspection of selected repositories, identify common
    failure modes of building of the software environment (such as for example
    not specifying the Python version to use).
  \item Design and develop the \softwarename{repo2docker-checker}. A prototype
    exists.\footnote{https://github.com/minrk/repo2docker-checker}
  \item Where possible, identify for what reason the software build has failed.
  \item Develop a strategy and heuristic to evaluate success of the build
    process.
  \item Identify suitable software repositories for the study.
  \item Automate the software reproduction process for the available
    repositories.
  \item Automate the analysis of the results, so the study can be repeated later.
  \item Carry out the study to estimate the fraction of reproducible
    repositories. (This is one of our KPIs, see Section~\ref{sec:KPIs}.)
  \item Repeat the study after the robustness of \repotodocker{} has
    been improved (\localtaskref{repo2docker-timemachine}) to evaluate
    progress.
  \end{compactitem}

  The tool will be made available as open source
  (\localdelivref{deliv-id-repo2docker-checker-software}).

  % Some of the findings
  % here will contribute to the \TODO{deliverable XXX in WP5 - best practice for
  %   reproducible repositories with Binder.} At the end of the project, we will
  % provide a summary of improvements in reproducibility we have achieved through
  % changes in the Binder tools.
\end{task}

\begin{task}[
  title=repo2docker development,
  id=repo2docker-timemachine,
  lead=SRL,
  PM=12,
  %wphases={0-24!0.5},
  partners={QS}
]

This tasks improves the robustness of \repotodocker{}. We illustrate this with one specific example:
Often, a repository of scientific results may
specify which software library is required (such as the Python library
\softwarename{pandas}), but not which version.

A software environment creation tool -- such as \repotodocker{} -- can then
attempt to install the most recent version of \softwarename{pandas}. This is
usually the intention of the authors, and was correct \emph{at the time the repository
was created}. However, as time moves on, the interface, behaviour and dependence
on other packages of \softwarename{pandas} will change, and at some point an
automatic build of the software for the whole repository may fail because of
conflicting dependencies.

We have found through prior study\cite{repo2docker-checker2020} that these problems can be
overcome if a \softwarename{pandas} version can be chosen that was the
most recent at the time when the repository was created. A related
issue is that the Python version itself (such as 3.8, 3.9 or 3.10) may
not be specified at all.

We will teach \repotodocker{} to establish the date of publication
(or last modification) of the repository, to determine the appropriate version
of software libraries from that time, and to select libraries with those
versions if no specific version is specified.

In the context of Python packages, we can use the
\softwarename{pypi-timemachine}
package \footnote{\url{https://github.com/astrofrog/pypi-timemachine}},
and we will implement a similar fetaure in the context of conda packages.
\end{task}

% template for a task
% each task should be added to exactly one workpackage
% in the workpackage task list
\begin{task}[
  title=Performance optimisation,
  % task id for references
  id=performance-optimisation,
  % lead institution ID
  lead=QS,
  PM=9,
  wphases={0-24!0.375},
  % partner institution ID(s)
  % don't include lead here
  partners={SRL}
]
  The creation of reproducible software environments can take -- depending on
  the complexity and overall size -- some time: generally, an image can be built
  within few minutes, but there are tasks taking much longer.

  In this task, we will optimise the \repotodocker{} performance.

  \TODO{QS - add a little more detail here?}

  \TODO{Mention in which deliverable this will be reported?}
  \begin{compactitem}
  \item ...
    % deliverable will be defined in the appropriate WorkPackage.tex
    % (\localdelivref{deliv-id})
  \end{compactitem}
\end{task}

\begin{task}[
  title=Maintenance of open source reproducibility software,
  id=maintenance,
  lead=SRL,
  PM=6,
  wphases={0-24!.25},
  partners={QS}
]

Developing software that people will use requires maintenance of that
software, not just new development. Through  this proposal we will
contribute general support to open source reproducibility software
where this is helpful for \TheProject. Such contributions are expected 
to the Jupyter (and as subproject) Binder code base. They support not
just all participants in \TheProject but also millions of people
relying on Jupyter software.


% Maintenance of core software is often an implicit and un-paid cost, or one
% hidden in over-describing the resources required to deliver proposed
% developments. In \TheProject, we make it clear and explicit that we will spend a
% significant amount of time developing and maintaining the core Jupyter and
% JupyterHub e-Infrastructure to respond to the needs of \TheProject and others,
% and contribute towards the sustainability and health of the community.
% 
% We will provide support to the Jupyter e-Infrastructure software, ensuring that
% it meets the needs (\localdelivref{jupyter-contributions}) of \TheProject, and
% aid in the release process to ensure that stable releases of Jupyter software
% can be used in mature \TheProject services (\localdelivref{jupyter-releases}).
% 
%   \TheProject will need improvements to core Jupyter functionality, including areas of:
% 
%   \begin{compactenum}
%     \item ease of deployment
%     \item security
%     \item scalability of JupyterHub
%     \item performance
%   \end{compactenum}
% 
%   We will contribute improvements in these areas,
%   meeting the needs of \TheProject and benefiting the wider Jupyter
%   community.

\end{task}


\end{tasklist}


\begin{wpdelivs}
  % \begin{wpdeliv}[due=1,miles=startup,id=infrastructure,dissem=PU,nature=DEC,lead=SRL]
  %   {Some Deliverable}
  % \end{wpdeliv}

  % (\localdelivref{deliv-id})

  \begin{wpdeliv}[due=12,id=deliv-id-repo2docker-checker-software,dissem=PU,nature=OTHER,lead=SRL]
    {Release software tool for checking of reproducibility of software
      environments (\texttt{repo2docker-checker})}
  \end{wpdeliv}

  \begin{wpdeliv}[due=24,id=repo2docker-checker-study-report,dissem=PU,nature=R,lead=SRL]
    {Summary of reproducibility improvements achieved.}
  \end{wpdeliv}

  \begin{wpdeliv}[due=24,id=repo2docker-release24,dissem=PU,nature=R,lead=SRL]
    {Release of \repotodocker{} with improve robustness features.}
  \end{wpdeliv}

\end{wpdelivs}

\end{workpackage}
%%% Local Variables:
%%% mode: latex
%%% TeX-master: "../proposal"
%%% End:

%  LocalWords:  workpackage wphases wpobjectives wpdescription pageref wpdelivs wpdeliv
%  LocalWords:  dissem mailinglists swrepository final-mgt-rep compactitem swsites ipr
%  LocalWords:  TOWRITE tasklist delivref
