\TOWRITE{ALL}{Proofread WP 5 Management pass 1}
\begin{draft}
\TOWRITE{PS (Work Package Lead)}{For WP leaders, please check the following (remove items
once completed)}
\begin{verbatim}
- [ ] have all the tasks in this Work Package a lead institution?
- [ ] have all deliverables in the WP a lead institution?
- [ ] do all tasks list all sites involved in them?
- [ ] does the table of sites and their PM efforts match lists of sites for each task?
      (each site from the table is listed in all relevant tasks, and no site is listed
      only in the table or only at some task)

- [ ] Binder / repo2docker documentation: tutorials and best practice guides -
      have we got this covered?
\end{verbatim}
\end{draft}
% (41)/36 = 1.138888 -> !1.14
% 35/30 = 1.16
\begin{workpackage}[id=education, wphases={0-12!0.48,12-24!0.48,24-36!0.71},
  title={Dissemination, education and engagement},
  short={Dissemination, education, and engagement},
  lead=IFR,
  IFRRM=10,
  MPRM=6,
  SRLRM=7,
  QSRM=3,
  UIORM=9,
  swsites
]


\begin{wpobjectives}
  A key focus of this work package is to disseminate the results of this
  project, including the technical advances and guidance for best practice for
  reproducible science. This includes educating researchers about the value of
  open science, reproducibility and re-usability as well as the possibilities of
  integrating Binder tools in their workflows.

  Beyond this activity, which is directed from the project members to the wider community of
  scientists, we use this work package to engage with researchers and
  stakeholders and seek input from them to the project. Desired input includes
  requirements for practical reproducibility in the different domain as well as
  technical contributions -- for example through merge requests for Binder
  tools, or open source documentation of best practice for reproducible software
  environments. Strong collaboration with the Community Engagement Panel is expected to take place.

  Our dissemination, education and engagement objectives includes:
 \begin{compactitem}
   \item Ensure awareness of the results of the project in the user community,
     and in particular in those groups that act as educators and multipliers of
     knowledge (such as the Carpentries and research infrastructure organisations).
   \item Educate the community on the value of open science, and in particular
   \item train researchers in best practices for open and reproducible science.
   \item Produce and training and education material to disseminate the ability to
     publish reproducible computational science outputs using the tools we
     improve and develop.
   \item Address the shortage of researchers and research support staff trained
     in practical reproducibility.
   \item Provide documentation and tutorials which can serve as the technical
     components of reproducibility policies.
   \item Throughout these activities engage with users and stakeholders, to
     listen and understand their barriers or incentives towards more
     reproducible science, and the usability of the \TheProject outputs.
 \end{compactitem}
\end{wpobjectives}

% Potential sources of inspiration: ODK's WP2 work package about dissemination:
% PDF: p.36 of https://github.com/OpenDreamKit/OpenDreamKit/raw/master/Proposal/proposal-www.pdf
% Sources: https://github.com/OpenDreamKit/OpenDreamKit/blob/master/Proposal/WorkPackages/DisseminationCommunityBuilding.tex

\begin{wpdescription}

  Open science and reproducible science is entirely dependent on researchers
  adopting open practices. 

  We address this challenge in multiple ways:
  \begin{enumerate}
  \item The philosophy of the Binder tools is to respect existing standards and
    best practice (and not to invent additional syntax or requirements). It is
    thus possible to use the Binder tools (to recreate a software environment)
    even if the repository authors did not anticipate the use of Binder, or knew
    about their existence. In the best possible scenario, a scientific research
    output becomes automatically reproducible with Binder without the
    \emph{author having to know about Binder or having to invest additional
      effort} (beyond following best practice).

    \item In this work package, we produce education materials and carry out
      education activities to spread the knowledge about \emph{good practice for
        reproducibility and re-usability in science}, such as for example
      automation of all analysis steps, and complete documentation of the
      required software stack. Only one aspect of this training is to show how
      Binder can help with reproducibility.

    \item Throughout the activities of this project and the engagement with the
      wider science community through the Community Engagement Panel, we aim to understand 
      the underlying drivers for
      acceptance, rejection or lack-of-interest in adopting practices that lead
      to reproducible science results.
  \end{enumerate}

  % HF: I think training the wider (non-scientist) public about Binder is going
  % too far?
  %
  % Going further, it is also clear that open science is not just of value
  % to researchers: one of the largest benefits of open science is that it makes
  % science accessible to the broader public who may not be members of the
  % research community.
  %
  % To this end, in addition to training researchers, we will also train the
  % public in how to make use of the open science research and services
  % facilitated by \TheProject. This will be done through regular open
  % dissemination and training workshops, as well as by producing and maintaining
  % material for online courses and documentation.

  Science applications (\WPref{applications}) will also support the creation
   of tutorials and \emph{best practice guides for
    reproducibility} (\localtaskref{online-resources}), and
  offer interactive (online, hybrid and/or in-person) workshops (\localtaskref{workshops}) 
  to help disseminate the content more effectively.

  We will also participate in the well established academic dissemination
  activities, and events of the European e-infrastructure projects and other
  relevant structures. EGI is a member of our the community engagement panel
  (\taskref{management}{community-engagement-panel}) and the interaction with
  them will be useful to prioritise our resources in this very active field.
  
\end{wpdescription}

\begin{tasklist}
% template for a task
% each task should be added to exactly one workpackage
% in the workpackage task list
\begin{task}[
  title=Best practice guidelines for reproducible science,
  id=online-resources,
  lead=UIO,
  PM=10,
  wphases={0-36!.28},
  partners={SRL,MP,UIO}
]
  The aims of this task are to (i) provide online resources for Open Science and
  (ii) support \taskref{education}{workshops}.
  
  This task includes the following activities:
  \begin{compactitem}
  \item Collect and compose best practice guidelines for reproducible and
    re-usable science. Split the content into multiple topic areas so learners
    with different prior knowledge can skip the content they are familiar with
    already.
  \item Develop lesson materials on \emph{open science} best practices (version
    control, testing, automation of all steps, collaboration and peer review,
    documentation, software licensing and open source, use of Jupyter
    notebooks).
  \item Develop lesson materials on \emph{reproducible computational science},
    which focuses on combining the open science tools for reproducible science.
  \item Develop materials on \emph{using Binder tools to make science more
      reproducible and re-usable}. This includes addressing and describing the
    use cases from \WPref{applications}.
  \item Collaboration with the \href{https://coderefinery.org}{CodeRefinery}
    project for the development and maintainance of the
    \href{https://coderefinery.org/lessons/}{online lesson materials}. Following
    CodeRefinery's tradition, the aim will be to contribute the lessons to
    \href{https://software-carpentry.org/}{Software Carpentry} and
    \href{https://data-carpentry.org/}{Data Carpentry}.
  \item The training material will also be referenced on the Binder tools webpage.
  \end{compactitem}
  All material will be licensed under an open license such as
  \href{https://creativecommons.org/licenses/by-sa/4.0/}{CC BY-SA 4.0}
  (\delivref{education}{education-materials1}, \delivref{education}{education-materials1}).
\end{task}

% template for a task
% each task should be added to exactly one workpackage
% in the workpackage task list
\begin{task}[
  title=Training Workshops for more reproducible science,
  id=workshops,
  lead=UIO,
  PM=9,
  % wphases={12-36!.25},
  partners={SRL,MP,IFR}
]
This task is focused on taking the content from the
\taskref{education}{online-resources} (Best practice for
reproducible science guidelines) and disseminating it through various channels and to different target audiences.

% \begin{compactitem}

%    \item Defining and implementing a strategy to enable a shared vision of the Jupyter ecosystem across all the actors from developers, users to every stakeholder: the current misalignment hinders the full exploitation of open software practices where co-design is a de facto approach.
%
% For instance, the official Jupyter documentation (https://jupyter.org/documentation) solely reflects the view of developers where the Jupyter ecosystem is defined as a set of software packages (jupyter-core, jupyter-client, kernels, widgets (ipywidgets, ipyleaflet, etc.). The user vision is relegated to examplars (blogs, newsletters, etc.) which inevitably tend to be restrictive but often become de facto standards. This can lead to misconceptions and makes it more difficult for on-boarding novices and new communities.
%

% \item Triggering a cultural change to help under-represented groups to actively participate to the development of open source project to ensure the sustainability of the \TheProject services deployed on EOSC-HUB.
%

%\item Foster open innovation by collaborating with others from different background and activities (school, universities, industries, journalists, artists, etc.)
%  \end{compactitem}

To achieve these goals, the following actions/activities will take place:

  \begin{compactitem}

  %\item co-design hackathons: the co-design efforts between domain scientists, \TheProject developers and service providers will be carried out at any point in time of the project and will be registered in a co-design register to help for future engagement with new communities of users. To be fully effective,  co-design hackathons will be organized to set the stage, define rules for co-design interactions and more importantly align all actors into a common user-centred vision of \TheProject services and associated development towards a successful EOSC deployment.

%    \item Workshops on Findable, Accessible, Interoperable and Reusable (FAIR)
%      software and data to facilitate the adoption of open science and open
%      scholarship best practices (transparent, sharable and collaborative
%      Science): this would not be restricted to the Jupyter ecosystem and will
%      teach users how to make data, lab notes and other research processes freely
%      available, under terms that enable reuse (licensing), redistribution and
%      reproducibility of methods and/or results.

   \item Delivery of workshops on (i) open science, (ii) reproducible computational
     science, and (iii) the use of Binder tools to support this.

     The content is focused on key insights and tools need for more reproducible
     science, but will be contextualised and delivered in the wider field of
     Findable, Accessible, Interoperable and Reusable (FAIR) software and data.

   % \item Trainings on how to use \TheProject software and services to fully
   %   exploit \TheProject developments for repoducible science: develop training
   %   materials and organize training events for researchers and the public to
   %   enable open science and maximise the usefulness of \TheProject
   %   developments.

   \item \TheProject Admin trainings: we will offer training events for learning on how to
     deploy \TheProject services such as BinderHub. This will be relevant for a
     very small (but important) group of users, i.e. those that want to host
     their own BinderHub instance. We know from multiple research organisations
     that this desire exists.

   \item Where possible (for instance after consultation of the Community Engagement Panel), 
     we will schedule dissemination events to take place
     during conferences and community events, such as PyData, EuroSciPy,
     Supercomputing meetings.

   \item We will archive recordings of the training events to support the
     increasing desire of learners to make use of online streaming services
     (such as YouTube) to work through a learning programme at their own time
     and pace.

   \item We will offer in-person, remote and hybrid training.

   \item The work will be done in collaboration with
     \href{https://coderefinery.org}{CodeRefinery} project (the University of Oslo is a partner of the CodeRefinery project) 
     and will make available its network of instructors and helpers
     to co-organize, advertise and run online workshops on open science best practices. 

  \item We will detail our executed activities through the reporting at the end of
    each reporting period.
  \end{compactitem}
\end{task}

\begin{task}[
  title=Community support and engagement,
  id=community-support,
  lead=SRL,
  PM=13,
  wphases={0-36!.36},
  partners={MP,QS,UIO,IFR}
]
A project such as \TheProject{} has the ambition to develop a small set of tools
that will \emph{impact many researchers} and have the potential to be useful
\emph{across all scientific domains that need electronic data processing} as part of their
scientific research and publication process.

As such, we expect that the demand through support queries, documentation
clarification questions, and helpful feedback will be substantial. With this
task, we explicitly reserve some time for such activities.

We have an opportunity here to address multiple aims simultaneously.

The aims of this task are
\begin{compactitem}
\item to engage with community members (and potentially their computing support
  staff) to help them make best use of the Binder tools. This can range from
  helping to configure a BinderHub installation, to address usage questions of
  tools such as \repotodocker{} in domain-specific contexts;
\item to engage with community members to better understand diverse
  requirements, and use this information to make the Binder tools and
  reproducibility guidelines more useful for a wider diversity of scientific
  domains;
\item to engage with community members to train researchers and research
  software engineers in reproducibility practices and tools (to address a
  shortage of staff with such skills)
\item to engage with community members to invite them to contribute to the
  binder tools, the reproducibility guidelines and policy development, and other
  open source tools.
\end{compactitem}

We will achieve those aims through listening to feedback, queries and requests
for help from the community, and reserve time to respond. Depending on the
complexity of an issue, guidance by email, chat, video meeting or even an
in-person visit may be appropriate. (When demand exceeds the time budget, we
will need to prioritise which issues we can deal with first.)

We know from our experience with running and contributing to open source
projects that such engagement activities are effective in training interested
and often highly skilled scientists and research software engineers to become
contributors to open source projects. While they may have a primary interest in
improving an open source tool to suit their needs, this will likely benefit
others as well. Once somebody has contributed to a particular open source
software tool, they are more likely to make follow-up contributions - for
example to improve documentation.

\end{task}

\end{tasklist}

\begin{wpdelivs}
  \begin{wpdeliv}[due=24,id=best-practice-guide,dissem=PU,nature=R,lead=IFR]
  {Best practice guide for reproducible science with Binder.}
\end{wpdeliv}
\begin{wpdeliv}[due=36,id=education-materials2,dissem=PU,nature=R,lead=IFR]
  {All training sessions material completed, reviewed, and published online.}
\end{wpdeliv}
% \begin{wpdeliv}[due=36,id=report2,dissem=PU,nature=R,lead=UIO]
%   {Community building: Report on impact of development workshops, dissemination and training activities.}
% \end{wpdeliv}
\end{wpdelivs}

\end{workpackage}
%%% Local Variables:
%%% mode: latex
%%% TeX-master: "../proposal"
%%% End:

%  LocalWords:  workpackage wphases wpobjectives wpdescription pageref wpdelivs wpdeliv
%  LocalWords:  dissem mailinglists swrepository final-mgt-rep compactitem swsites ipr
%  LocalWords:  TOWRITE tasklist delivref
