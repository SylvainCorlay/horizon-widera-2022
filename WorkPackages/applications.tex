\TOWRITE{ALL}{Proofread WP 1 Management pass 1}
\begin{draft}
\TOWRITE{PS (Work Package Lead)}{For WP leaders, please check the following (remove items
once completed)}
\begin{verbatim}
- [ ] have all the tasks in this Work Package a lead institution?
- [ ] have all deliverables in the WP a lead institution?
- [ ] do all tasks list all sites involved in them?
- [ ] does the table of sites and their PM efforts match lists of sites for each task?
      (each site from the table is listed in all relevant tasks, and no site is listed
      only in the table or only at some task)
\end{verbatim}
\end{draft}

\begin{workpackage}[
  id=applications,
  wphases=0-36,
  swsites,
  title=Applications and use cases,
  short=Applications,
  lead=MP,
  % EGIRM=7,
  % CDSRM=12,
  % INSERMRM=24,
  % QSRM=6,
  % SILRM=12,
  SRLRM=9,
  % UIORM=12,
  % UPSUDRM=20,
  % WTTRM=3,
  % XFELRM=36,
  % EPRM=3,
]
\begin{wpobjectives}
  The objectives of this work package are
 \begin{compactitem}
   \item to guide the development of core tools by simultaneously
     developing and using applications in diverse fields with active
     scientists from these fields, and
   \item to demonstrate that the tools we develop are valuable to diverse
     fields of science, thus ensuring we develop e-infrastructure and
     services which can cater for a broad customer base of EOSC.
   \end{compactitem}
\end{wpobjectives}

\begin{wpdescription}

  Whilst the components issued from work packages  \WPref{core} and \WPref{ecosystem} will be
  made available as generic building blocks for EOSC services, this
  work package aims at building and deploying bespoke EOSC services
  targeting real-world cases.

  We have selected a number of applications in a variety of domains
  to demonstrate the broad impact of \TheProject, in particular in the
  areas of astronomy
  % (\localtaskref{astro}), education
  % (\localtaskref{teaching}), fluid dynamics
  % (\localtaskref{application-gpu}), geosciences
  % (\localtaskref{geoscience}), health
  % (\localtaskref{opendose-analysis}), mathematics
  % (\localtaskref{math}) and photon science and imaging
  % (\localtaskref{reproducibility-xfel}).
  The context and vision for each of the demonstrators is described in
  section \ref{sec:science-demonstrators-in-concept} on page
  \pageref{sec:science-demonstrators-in-concept}.

  Working closely with the core developers of the Jupyter ecosystem will make it possible to
  go way beyond what is normally available "out-of-the-box" and to offer better solutions,
  thereby guiding further development of the core features.

  \medskip
  Our demonstrators will typically undergo two-stages: (i)
  development and testing of the services locally at the developing partner
  site. (ii) Making the service available through the European Open
  Science Cloud (EOSC).

  All demonstrators will deliver base-line services by making the
  relevant notebooks executable in the \emph{European Binder Service} instance that this
  project will deploy on EOSC (\taskref{eosc}{eu-binder}). This will demonstrate
  the Jupyter service capabilities such as reproducibility, interactive
  widget use and visualisation, and show how these can
  enable new open science on EOSC.

  The particular workflows, data infrastructures and data policies for
  FAIR\footnote{Findable, Accessible, Interoperable and Reusable} sharing of data vary from one community and use-case to
  the other, or may not be fully defined yet. Therefore, this proposal
  does not enforce a specific way of handling data. Instead we
  will explore in the demonstrator tasks how existing data policies,
  infrastructure and workflows can be respected and integrated with
  authentication and authorisation, data management, and
  JupyterHub/Binder services on EOSC. EGI is a partner
  for all the tasks in this work package and will work with us to find the
  best integration solutions in the evolving EOSC
  infrastructure.

  In the EOSC-hub project EGI operates a Jupyter Hub service which is deployed
  in a scalable mode on EGI IaaS Cloud. This Jupyter Hub is already integrated
  with the EUDAT B2DROP and OneData data management services of EOSC, and will
  be integrated, in the next 12 months, with the EUDAT B2SHARE service.
  The integrations enable users to move data between Jupyter notebooks and storage
  sites of the EGI Federation (with Onedata), between Jupyter and storage sites
  of the EUDAT federation (B2SHARE), and between Jupyter and their personal cloud
  storage hosted in EUDAT (B2DROP). The WP4 use cases will evaluate these data management
  integrations and EGI will bring the respective technology from EOSC-hub into the
  services operated by \TheProject.

  For some of the demonstrators, authentication and authorization and/or
  data management are being advanced outside \TheProject.
  This is for instance the case for the photon science and astronomy
  demonstrators via \href{https://panosc-eu.github.io/}{PaNOSC} and
  \href{https://www.eso.org/public/announcements/ann18084/}{ESCAPE} projects, respectively.

\end{wpdescription}

\begin{tasklist}
% add tasks from task directory here
% template for a task
% each task should be added to exactly one workpackage
% in the workpackage task list
\begin{task}[
  title=Sample Task,
  % task id for references
  id=task-id,
  % lead institution ID
  lead=SRL,
  PM=1,
  wphases={0-36},
  % partner institution ID(s)
  % don't include lead here
  partners={XXX}
]
  The task includes the following activities
  \begin{compactitem}
  \item ...
     % deliverable will be defined in the appropriate WorkPackage.tex
    % (\localdelivref{deliv-id})
  \end{compactitem}
\end{task}

\end{tasklist}



\begin{wpdelivs}
%\TODO{update due date and startup!}
%\TODO{update milestone!}
\begin{wpdeliv}[
    % id for linking with \delivref or \localdelivref
    id=deliv,
    % lead institution
    lead=XXX,
    % month when deliverable is due (max 36)
    due=12,
    % associated milestone id (see milestones.tex)
    miles=startup,
    % ~always PU, DEC
    dissem=PU,
    nature=DEC,
]
  {
  One-line name of deliverable
  }
\end{wpdeliv}


\end{wpdelivs}
\end{workpackage}
%%% Local Variables:
%%% mode: latex
%%% TeX-master: "../proposal"
%%% End:

%  LocalWords:  workpackage wphases wpobjectives wpdescription pageref wpdelivs wpdeliv
%  LocalWords:  dissem mailinglists swrepository final-mgt-rep compactitem swsites ipr
%  LocalWords:  TOWRITE tasklist delivref
