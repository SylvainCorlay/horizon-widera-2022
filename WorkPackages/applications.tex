\TOWRITE{ALL}{Proofread WP 1 Management pass 1}
\begin{draft}
\TOWRITE{PS (Work Package Lead)}{For WP leaders, please check the following (remove items
once completed)}
\begin{verbatim}
- [ ] have all the tasks in this Work Package a lead institution?
- [ ] have all deliverables in the WP a lead institution?
- [ ] do all tasks list all sites involved in them?
- [ ] does the table of sites and their PM efforts match lists of sites for each task?
      (each site from the table is listed in all relevant tasks, and no site is listed
      only in the table or only at some task)
\end{verbatim}
\end{draft}

\begin{workpackage}[id=applications,wphases=0-48,swsites,
  title=Science demonstrators,
  short=Demonstrators,
  lead=MP,
  % EGIRM=7,
  % CDSRM=12,
  % INSERMRM=24,
  % QSRM=6,
  % SILRM=12,
  SRLRM=9,
  % UIORM=12,
  % UPSUDRM=20,
  % WTTRM=3,
  % XFELRM=36,
  % EPRM=3,
]
\begin{wpobjectives}
  The objectives of this work package are
 \begin{compactitem}
   \item to guide the development of core tools by simultaneously
     developing and using applications in diverse fields with active
     scientists from these fields, and
   \item to demonstrate that the tools we develop are valuable to diverse
     fields of science, thus ensuring we develop e-infrastructure and
     services which can cater for a broad customer base of EOSC.
   \end{compactitem}
\end{wpobjectives}

\begin{wpdescription}

  Whilst the components issued from work packages  \WPref{core} and \WPref{ecosystem} will be
  made available as generic building blocks for EOSC services, this
  work package aims at building and deploying bespoke EOSC services
  targeting real-world cases.

  We have selected a number of applications in a variety of domains
  to demonstrate the broad impact of \TheProject, in particular in the
  areas of astronomy
  (\localtaskref{astro}), education
  (\localtaskref{teaching}), fluid dynamics
  (\localtaskref{application-gpu}), geosciences
  (\localtaskref{geoscience}), health
  (\localtaskref{opendose-analysis}), mathematics
  (\localtaskref{math}) and photon science and imaging
  (\localtaskref{reproducibility-xfel}).
  The context and vision for each of the demonstrators is described in
  section \ref{sec:science-demonstrators-in-concept} on page
  \pageref{sec:science-demonstrators-in-concept}.

  Working closely with the core developers of the Jupyter ecosystem will make it possible to
  go way beyond what is normally available "out-of-the-box" and to offer better solutions,
  thereby guiding further development of the core features.

  \medskip
  Our demonstrators will typically undergo two-stages: (i)
  development and testing of the services locally at the developing partner
  site. (ii) Making the service available through the European Open
  Science Cloud (EOSC).

  All demonstrators will deliver base-line services by making the
  relevant notebooks executable in the \emph{European Binder Service} instance that this
  project will deploy on EOSC (\taskref{eosc}{eu-binder}). This will demonstrate
  the Jupyter service capabilities such as reproducibility, interactive
  widget use and visualisation, and show how these can
  enable new open science on EOSC.

  The particular workflows, data infrastructures and data policies for
  FAIR\footnote{Findable, Accessible, Interoperable and Reusable} sharing of data vary from one community and use-case to
  the other, or may not be fully defined yet. Therefore, this proposal
  does not enforce a specific way of handling data. Instead we
  will explore in the demonstrator tasks how existing data policies,
  infrastructure and workflows can be respected and integrated with
  authentication and authorisation, data management, and
  JupyterHub/Binder services on EOSC. EGI is a partner
  for all the tasks in this work package and will work with us to find the
  best integration solutions in the evolving EOSC
  infrastructure.

  In the EOSC-hub project EGI operates a Jupyter Hub service which is deployed
  in a scalable mode on EGI IaaS Cloud. This Jupyter Hub is already integrated
  with the EUDAT B2DROP and OneData data management services of EOSC, and will
  be integrated, in the next 12 months, with the EUDAT B2SHARE service.
  The integrations enable users to move data between Jupyter notebooks and storage
  sites of the EGI Federation (with Onedata), between Jupyter and storage sites
  of the EUDAT federation (B2SHARE), and between Jupyter and their personal cloud
  storage hosted in EUDAT (B2DROP). The WP4 use cases will evaluate these data management
  integrations and EGI will bring the respective technology from EOSC-hub into the
  services operated by \TheProject.

  For some of the demonstrators, authentication and authorization and/or
  data management are being advanced outside \TheProject.
  This is for instance the case for the photon science and astronomy
  demonstrators via \href{https://panosc-eu.github.io/}{PaNOSC} and
  \href{https://www.eso.org/public/announcements/ann18084/}{ESCAPE} projects, respectively.

\end{wpdescription}

\begin{tasklist}
% add tasks from task directory here
% % template for a task
% each task should be added to exactly one workpackage
% in the workpackage task list
\begin{task}[
  title=Sample Task,
  % task id for references
  id=task-id,
  % lead institution ID
  lead=SRL,
  PM=1,
  wphases={0-36},
  % partner institution ID(s)
  % don't include lead here
  partners={XXX}
]
  The task includes the following activities
  \begin{compactitem}
  \item ...
     % deliverable will be defined in the appropriate WorkPackage.tex
    % (\localdelivref{deliv-id})
  \end{compactitem}
\end{task}

\begin{task}[
  title=Co-design and technical support,
  id=codesign-support,
  lead=SRL,
  PM=15,
  wphases={0-36!.3},
  partners={SRL,MP,UIO,QS}
]

This task coordinates and supports the work of the other \WPref{applications}
tasks. It will help us exploit synergies and coordinate the
gathering and formulation of requirements and the
preparation of the deliverables. It will also support demonstrator efforts to
speed-up the development process and ease the deployment of innovative
services. Finally, it will drive the co-design cycle between
\WPref{applications} and all the technical work packages
(\WPref{core}, \WPref{ecosystem} and \WPref{eosc}).

\begin{compactitem}
\item Assess co-design efforts and distribute workload across the core
  developers
\item Regularly feedback information to \WPref{education} to adapt
  trainings and dissemination
\item Offer technical support to the demonstrators throughout all the
  steps until final deployment of the services
\item Liase with \taskref{eosc}{eosc} for authentication,
  authorisation, data management and further EOSC integration.
\end{compactitem}

\end{task}

\input{tasks/application-astro}
\input{tasks/teaching}
\input{tasks/application-gpu}
\begin{task}[
  title=Demonstrator: Geosciences,
  id=geoscience,
  lead=UIO,
  PM=22,
  %wphases={0-36!.5},
  partners={QS,SRL}
]

% UPSud involvement: UPSud has a geoscience group (GEOPS) and will be
% interested in using the tools developed here. No formal PM.

The aim of this task (see page \pageref{sec:concept-demonstrators-geo}) is to build on the Jupyter ecosystem to create a standardised and shareable computing, data analysis and visualisation framework for Geosciences. This task will focus on filling gaps that hinder open science and will include the following activities:

\emph{Visualisation}

\begin{compactitem}
  \item Improvement upon existing mapping tools for specialised
    visualisation of in-situ and model-generated data arising in
    specific use cases (Land, river-runoff, ocean, ice, wave and
    atmosphere models, particle dispersion models, oil spill models,
    etc.).

  \item Improvements of the tooling for 3-D visualisation of
    geographical data sets in the Jupyter notebook, for use cases such as
    displaying clouds, volcanic plumes, atmospheric rivers.
\end{compactitem}

\emph{Collaboration with Jupyter with specialised tools for earth sciences}

\begin{compactitem}
  \item adding the ability to interactively integrate information or corrections
    observed during field trips, corresponding to specific geographical locations.

  \TODO{Concurrent editing links to real-time editing from the
    core WP2 - mention link?}

  \item adding the ability to deploy Jupyter-based applications together with
    the corresponding execution environment, both in the form of a runnable
    notebook with \emph{Binder} or as a read-only yet interactive \emph{Voila}
    dashboard.
\end{compactitem}

This work will be carried out in two stages with first local development and deployment of \TheProject EOSC services for Geosciences 
(such as a BinderHub for Big data geosciences and \emph{voila} innovative interactive \emph{App}) and then deployment of these services on EOSC (\localdelivref{demonstrators}).
\end{task}

\input{tasks/opendose-analysis}
\input{tasks/application-math}
% template for a task
% each task should be added to exactly one workpackage
% in the workpackage task list
\begin{task}[
  title=Demonstrator: Reproducible photon science workflows at European XFEL,
  id=reproducibility-xfel,
  lead=MP,
  PM=35,
  wphases={6-36},
  partners={SRL}
  ]

  This task (see page \pageref{sec:concept-demonstrator-photonscience}
  for context) includes the following activities:
  \begin{compactitem}
  \item Use the software archive for reproducible computation
    (as co-developed in \taskref{ecosystem}{reproducibility}), with
    the aim to provide reproducible computation environments for data analysis at
    European XFEL that remains executable for the same duration as the
    data is kept (currently aiming at 10+ years, at least 5 years).

    As is common in computational science, software used at XFEL often
    relies on specific combinations of libraries, in many cases with
    particular version requirements. Thus we will need a dedicated
    software archive that holds all relevant packages and source codes
    that are required to build the required computational environments
    (see \taskref{ecosystem}{reproducibility}) to ensure they are
    available even if an open source software provider decides to
    remove their repositories, or changes the API of a package, or
    GitHub decides to terminate their business.

    Applying the work from \taskref{ecosystem}{reproducibility} in the
    context of a production system will demonstrate its true utility,
    and provide important feedback for the design. There will be
    iterative feedback and refinement of the service.

  \item Extend the use of notebooks from \emph{interactive} data
    exploration and analysis at European XFEL to also provide
    computational work flows via (semi-)automatic execution of
    notebooks as described above. The work done in
    \taskref{core}{collaboration} will allow us to execute notebooks in
    the background, and to connect to the running notebook process to
    display or inspect progress, or to modify such a notebook if the
    science requires it.

    By doing so, we can make the standard analysis that is carried out
    by the facility available on EOSC as a service. By using one tool
    (the notebook) we simplify processes for users and for the research
    facility.

  \item Use the work from \taskref{ecosystem}{jupyter-widgets} on
    state-preserving widgets to provide GUI-like elements in notebook
    where interactive user input, data exploration or parameter
    modification is required.

  \item Explore use of the Voila capability to provide
    data exploration dash-boards to lower barriers of working with the
    data (will only be possible for somewhat standard experiments).

  \item Work with the PaNOSC project \cite{panosc} to evaluate and use
    these new and EOSC-enabled services for other Photon and Neutron
    Science research facilities.
  \item Develop a demonstrator (Deliverable \localdelivref{demonstrators}).

  \item Evaluate the chosen workflow design and experience from using
    it in a real-world context; make this available as a report and
    through presentations/workshops to interested organisations and
    facilities. (\localdelivref{applications-report}).




  \end{compactitem}

 \end{task}

\end{tasklist}



\begin{wpdelivs}
%\TODO{update due date and startup!}
%\TODO{update milestone!}

  \begin{wpdeliv}[due=12,miles=startup,id=codesign-support,dissem=PU,nature=R,lead=SRL]
    {Initial requirements for the demonstrators}
  \end{wpdeliv}

  \begin{wpdeliv}[due=24,miles=prototype,id=local-services,dissem=PU,nature=R,lead=EP]
    {Report on the developments of the demonstrator services deployed locally}
  \end{wpdeliv}

  \begin{wpdeliv}[due=36,miles=community,id=demonstrators,dissem=PU,nature=DEM,lead=EGI]
    {Demonstrators based on locally developed services made accessible
      through EOSC. Demonstrators may be developed further subsequently}
  \end{wpdeliv}

  \begin{wpdeliv}[due=48,miles=final,id=applications-report,dissem=PU,nature=R,lead=XFEL]
    {Evaluation of demonstrators and case studies. Report on
      feasibility and user feedback to guide EOSC service design}
  \end{wpdeliv}

\end{wpdelivs}
\end{workpackage}
%%% Local Variables:
%%% mode: latex
%%% TeX-master: "../proposal"
%%% End:

%  LocalWords:  workpackage wphases wpobjectives wpdescription pageref wpdelivs wpdeliv
%  LocalWords:  dissem mailinglists swrepository final-mgt-rep compactitem swsites ipr
%  LocalWords:  TOWRITE tasklist delivref
