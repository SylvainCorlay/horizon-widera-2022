\TOWRITE{ALL}{Proofread WP 1 Management pass 1}
\begin{draft}
\TOWRITE{PS (Work Package Lead)}{For WP leaders, please check the following (remove items
once completed)}
\begin{verbatim}
- [ ] have all the tasks in this Work Package a lead institution?
- [ ] have all deliverables in the WP a lead institution?
- [ ] do all tasks list all sites involved in them?
- [ ] does the table of sites and their PM efforts match lists of sites for each task?
      (each site from the table is listed in all relevant tasks, and no site is listed
      only in the table or only at some task)
\end{verbatim}
\end{draft}

\begin{workpackage}[
  id=applications,
  wphases=0-36!0.89,
  swsites,
  title=Applications and use cases,
  short=Applications,
  lead=MP,
  MPRM=10,
  SRLRM=3,
  UIORM=8,
  IFRRM=11
]

\begin{wpobjectives}
  The objectives of this work package are
 \begin{compactitem}
 \item to guide the development of core tools in \WPref{reproducibility} and \WPref{impact}
   by simultaneously attempting to use  them in diverse fields and use cases
 \item to do this together with active scientists from these fields to ensuring we develop tools
   which can cater for a broad European and global research community
 \item to demonstrate how the tools we develop can support more reproducible and
   re-usable science
   \end{compactitem}
\end{wpobjectives}

\begin{wpdescription}

  Whilst the components issued from work packages  \WPref{reproducibility} and \WPref{impact} will be
  made available as generic tools for reproducible open science,
  this work package is focused on using (and if required tailoring them further)
  to make them suitable for specific real-world cases.

  The use-cases we anticipate are
  \begin{compactitem}
  \item \localtaskref{binder-at-home} Ability to recreate software environments
    and re-produce results using local compute resources (such as the Desktop of
    a researcher)
  \item \localtaskref{data-publishing} Use of Binder to facilitate access to
    large data sets where the published resource does not only include the data
    itself, but also software to access and read the data.
  \item \localtaskref{binder-at-hpc} Use of Binder at High Performance Computing
    facilities to re-produce computationally more demanding results
    \item \localtaskref{demos} Applications and best practice examples that
    demonstrate use of Binder for more reproducible and re-usable science.
  \end{compactitem}

  \TOWRITE{}{Edit this paragraph:}
  The context and vision for each of the demonstrators is described in
  section \ref{sec:science-demonstrators-in-concept} on page
  \pageref{sec:science-demonstrators-in-concept}.

  Working collaboratively with core Binder developers and research active
  scientists, we merge state-of-the-art knowledge on what is technically
  possible with the understanding of the scientists what reproducibility
  features would significantly improve their workflow. We expect that -- in
  addition to iteratively refining the features of Binder -- we will also
  inspire each other to find out-of-the box solutions that each group on
  their own may not come to think about.

  \medskip

  The particular workflows, data infrastructures, data policies, and data access
  protocols for
  FAIR\footnote{Findable, Accessible, Interoperable and Reusable} sharing of data vary from one community and use-case to
  the other, and may not be fully defined yet. Therefore, this proposal
  does not enforce a specific way of handling data. Instead we
  will explore in the demonstrator tasks how existing data policies,
  infrastructure and workflows can be respected and integrated with
  authentication and authorisation, data management, and
  Binder services. EGI is a partner in this project \TODO{Is ``partner'' the
    right term? We will have a letter of support, but they are not formally a
    partner site in the submission.}
  for all the tasks in this work package and will work with us to find the
  best integration solutions in the evolving EOSC infrastructure.

  \medskip

  The tasks in this work package are progressed throughout the runtime of the
  project, serving both as requirements capture (and so to inform and guide the
  work in \WPref{reproducibility} and \WPref{impact}) and evaluation of the
  technological advances of the Binder tools.

  We will use the regular technical reports to update on progress. A final
  report will summarise the results (\localdelivref{report-use-cases}).
\end{wpdescription}

\begin{tasklist}

% template for a task
% each task should be added to exactly one workpackage
% in the workpackage task list
\begin{task}[
  title=Science demonstrators,
  % task id for references
  id=demos,
  % lead institution ID
  lead=MP,
  PM=8,
  wphases={0-36},
  % partner institution ID(s)
  % don't include lead here
  partners={IFR,UIO}
]

  In this task, we want to demonstrate the value and usefulness of \WPref{reproducibility} and
  \WPref{impact} with real scientific use cases from the research communities involved in \TheProject.
  The demonstrators are designed to exploit the solutions developed within \TheProject (Binder@Home, Binder@HPC, data publishing)
  and leverage existing institutional and/or national e-infrastructures as well as core EOSC services.

  \begin{compactitem}
  \item FAIR Nordic Earth System Modelling: this science demonstrator leverages Binder@HOME (model development, education, single column or very simple model configuration), Binder@HPC (operational runs at scale including on EuroHPC), data publishing (publication of simulation results from blue-sky research);
     % deliverable will be defined in the appropriate WorkPackage.tex
    % (\localdelivref{deliv-id})
  \end{compactitem}
\end{task}

%% template for a task
% each task should be added to exactly one workpackage
% in the workpackage task list
\begin{task}[
  title=Prototype Policy,
  % task id for references
  id=policy,
  % lead institution ID
  lead=SRL,
  PM=1,
  wphases={0-36},
  % partner institution ID(s)
  % don't include lead here
  partners={XXX}
]
  Develop a prototype policy for reproducible science using \TheProject tools.

  \begin{compactitem}
  \item ...
     % deliverable will be defined in the appropriate WorkPackage.tex
    % (\localdelivref{deliv-id})
  \end{compactitem}
\end{task}

% template for a task
% each task should be added to exactly one workpackage
% in the workpackage task list
\begin{task}[
  title=Binder@Home,
  % task id for references
  id=binder-at-home,
  % lead institution ID
  lead=SRL,
  PM=7,
  %wphases={0-36},
  % partner institution ID(s)
  % don't include lead here
  partners={MP,UIO}
]
In this task we target the compute power of the desktops of individual researchers and
users to recreate software environments in which computational results can be
reproduced and re-used: We envision to provide an experience identical or similar
to that of using BinderHub, but only employing the local computer available to the user anyway.

\paragraph*{Context:} Reproducibility services using Binder currently rely on hosted Binder instances.
The BinderHub Federation provides such a service global service at mybinder.org
which is free at
the point of use. It delegates the building of the software environment and
re-execution of the code to a small number of computer centres that have
volunteered to contribute compute resources.
It may hence be possible to overload the system \TODO{Add link to general discussion
  that mentions how many thousand builds take place per months etc}.
To allow the up-scaling of good reproducibility practice, it is 
desirable not to depend exclusively on such a single hosted service (or even multiple similar hosted services).

Moreover, the compute resources typically offered through mybinder are modest, namely at most 2 GB of RAM
and a single CPU core.
In contrast most laptops and Desktops have similar or better hardware capabilities than
the free mybinder cloud-computing resources currently offer, making it highly desirable to leverage these local resources.
% not sure how/if the following two sentences fit here, maybe better elsewhere
For some research areas, it is essential to access big data sets to reproduce the scientific computing workflow.
For such cases, it is crucial to reproduce the work using infrastructure that has optimised access to the data,
so that one does not consume unnessesary computing and network resources.  

\paragraph*{Task activity:} Based on the preparations in \WPref{reproducibility} and
\WPref{impact}, we intend to extend Binder such that users of the service can
carry out the building of the environment, and -- if desired -- the launching of a
notebook server \emph{on their own hardware}, e.g. their laptop or on-premise or cloud infrastructure.
The working title for such
functionality is ``Binder@Home'' (as a reference to the crowd-based SETI@home search for
extraterrestrial intelligence at home.\footnote{https://setiathome.berkeley.edu})

We will design, implement, and test the 'Binder@Home' functionality, and make it
available as part of the Binder software. That new 'Binder@Home' software component will
essentially complement BinderHub with an easy-to-use local single-user case, it will trigger
the local build of the software environment, the start of the Jupyter notebook
server, and the opening of the relevant local URL and port in a browser.
This effort will bridge the gap from mybinder.org to ``Home'';
by giving the freedom and digital sovereignty
to the researchers to chose where they execute their computational experiments in a simple manner,
without sacrificing the convenience of the mybinder.org service.

%  The task includes the following activities
%  \begin{compactitem}
%  \item ...
%     % deliverable will be defined in the appropriate WorkPackage.tex
%    % (\localdelivref{deliv-id})
%  \end{compactitem}
\end{task}

% template for a task
% each task should be added to exactly one workpackage
% in the workpackage task list
\begin{task}[
  title=Data publishing,
  % task id for references
  id=data-publishing,
  % lead institution ID
  lead=MP,
  PM=8,
  wphases={0-36},
  % partner institution ID(s)
  % don't include lead here
  partners={IFR,UIO}
  ]
  The task focuses on the use of reproducible software environments within
  Binder to provide working and interactive code that provides access to a large
  or complex data set.

  \paragraph*{Context:} Scientists would like to publish their data. Such a data
  publication must include data set itself, and metadata that explains how to
  interpret the data. In addition to such information, it can improve the
  quality of the data set if \emph{computer executable libraries or commands}
  are provided, which simplify the reading of the actual data files. Such
  routines encapsulates meta information about the data file (format and
  structure) in a machine-readable format.

  A binder-enabled repository can provide the access to such data sets by
  containing the specification of a software environment and hosting of the
  file-reading routines together. Such an approach significantly simplifies the
  re-use of the data (or reproduction of existing study) because the data
  reading routines do not need to be re-implemented.

  \paragraph*{Task activity:}
  We will design and implement functionality that allows such data publishing
  based on Binder tools.

  A major challenge is the link to the data: ideally, data sets are hosted on a
  separate infrastructure (such as archives, or files published together with a
  publication - for example on Zenodo). It will thus be necessary to reference
  the data this data-holding archive and the data location within that resource
  somehow. This data location will need to be used from the notebooks to access
  the data. %We are not aware of a common standard that could be adapted here.

  Some authentication for data access may be required: either because the data
  is not meant to be fully public, or because access to the data creates
  significant cost for the hosting party. Such authentication
  information/credentials from (the Binderhub) login must be passed to the point
  where the data-holding medium is mounted in a container.

  We will work very closely with the Max Planck Compute and Data Facility
  (MPCDF) to prototype such functionality. We will allow to access data sets
  that the MPCDF hosts themselves.

  An important outcome of this task is an evaluation of the chosen design and
  implementation, to propose a more generic model for the next feature extension
  of the Binder tools.
\end{task}

% template for a task
% each task should be added to exactly one workpackage
% in the workpackage task list
\begin{task}[
  title=Binder at HPC facilities,
  % task id for references
  id=binder-at-hpc,
  % lead institution ID
  lead=MP,
  PM=10,
  wphases={0-36},
  % partner institution ID(s)
  % don't include lead here
  partners={IFR, UIO}
  ]
In this task, we want to broaden the applicability of the Binder tools to become
useful in High Performance Computing (HPC) environments.

\paragraph*{Context:}
Reproducibility of data created with HPC resources is difficult. In addition to
a specification and access to the actual (simulation) software and dependencies,
one may also need specialised HPC hardware to be able to execute the software.

The notebook interface -- which works well for many examples in computational and
data science -- may not be appropriate for HPC applications, where jobs of
substantial run-time are typically submitted to a queueing system, which will
trigger execution of the (parallel) job, once the required resources have become
available.

It is outside the scope of this proposal to find and implement a generic
reproducibility approach for HPC use cases. Nevertheless, we propose to explore
some aspects of HPC-reproducibility already to influence the development of Binder,
and start the - probably iterative - process of finding a good solution.

\paragraph*{Strategy:}
We focus on reproducible execution of an HPC application to compute data as the step of
novelty here. This may well be without using notebooks, but consist of the
building of the software environment, and submission of a job making use of this
environment to the HPC system's queue.

We assume for this to work that the user (who wants to reproduce some results)
needs to logon to the HPC resource of their choice, and uses repo2docker to
create a suitable software container, and starts execution of those containers
'manually', for example through submission of a compute job to the Slurm
queueing system. We also assume that the hardware required for this reproduction
is available on the HPC system.

\paragraph*{Activity}: We will use repo2docker to automate creation of
reproducible software environment on an HPC system. If no parallelism is
required, this is similar to the Binder@home scenario
(\localtaskref{binder-at-home}). Shared memory parallelisation, using for
example OpenMP, should work well with approach.

A main task here will be to explore the feasibility of using distributed
parallelisation (for example through MPI) where for the execution of an
MPI-program the processes on the nodes run in containers but communicate via MPI
as usual. We evaluate the situation here with software that allows
MPI-parallelisation (such as Octopus). Subsequently, we will share our experience with
reproducible software environments in HPC contexts. \TODO{Add link to report
  deliverable}.

\paragraph*{Challenges}: We expect that we need to use a container technology
that is widely accepted at HPC sites (such as for example Singularity): Docker
on HPC systems is difficult due to the root-user requirements.

Another challenge is that of using accelerators such as GPU cards, which are
installed on the host, but need to be instructed from the software running
inside the container. \TODO{Add links here to relevant papers.}

For HPC systems, there are sometimes specialised drivers for high-performance
network cards: can these be used from the container environment, and if not what
is the performance impact?

Should hardware requirements be archived in the reproducible repository? If so,
what specification should we use?

The existing buildpacks that repo2docker supports \TODO{Add link to overview of
  supported languages etc}, may need to be extended for HPC specific software
provision (such as Spack or Easybuild).

% The re-execution of parallel software may create significant costs, and thus
% authentication may be necessary.

%   The task includes the following activities
%   \begin{compactitem}
%   \item ...
%      % deliverable will be defined in the appropriate WorkPackage.tex
%     % (\localdelivref{deliv-id})
%   \end{compactitem}
\end{task}

\end{tasklist}


\begin{wpdelivs}
\begin{wpdeliv}[
    % id for linking with \delivref or \localdelivref
  id=report-use-cases,
    % lead institution
    lead=MP,
    % month when deliverable is due (max 36)
    due=36,
    % associated milestone id (see milestones.tex)
    miles=final,
    % ~always PU, DEC
    dissem=PU,
    nature=DEC,
]
  {
  Report on real world use cases of Binder for reproducible and re-usable science
  }
\end{wpdeliv}

\begin{wpdeliv}[
  % id for linking with \delivref or \localdelivref
  id=deliv,
  % lead institution
  lead=UIO,
  % month when deliverable is due (max 36)
  due=36,
  % associated milestone id (see milestones.tex)
  miles=final,
  % ~always PU, DEC
  dissem=PU,
  nature=DEC,
  ]
  {
    Documentation of reproducibility examples and best practice with Binder (to
    be come part of the Binder software documentation)
  }
\end{wpdeliv}

\end{wpdelivs}
\end{workpackage}
%%% Local Variables:
%%% mode: latex
%%% TeX-master: "../proposal"
%%% End:

%  LocalWords:  workpackage wphases wpobjectives wpdescription pageref wpdelivs wpdeliv
%  LocalWords:  dissem mailinglists swrepository final-mgt-rep compactitem swsites ipr
%  LocalWords:  TOWRITE tasklist delivref
