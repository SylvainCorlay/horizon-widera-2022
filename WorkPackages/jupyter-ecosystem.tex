\TOWRITE{ALL}{Proofread WP 1 Management pass 1}
\begin{draft}
\TOWRITE{PS (Work Package Lead)}{For WP leaders, please check the following (remove items
once completed)}
\begin{verbatim}
- [ ] have all the tasks in this Work Package a lead institution?
- [ ] have all deliverables in the WP a lead institution?
- [ ] do all tasks list all sites involved in them?
- [ ] does the table of sites and their PM efforts match lists of sites for each task?
      (each site from the table is listed in all relevant tasks, and no site is listed
      only in the table or only at some task)
\end{verbatim}
\end{draft}

\begin{workpackage}[id=ecosystem,wphases=0-48,swsites,
  title=Developing the Jupyter Ecosystem,
  short=Ecosystem,
  lead=QS,
  % EGIRM=4,
  % INSERMRM=4,
  % EPRM=21,
  % QSRM=20,
  % SILRM=4,
  SRLRM=30,
  % % UIORM=4,
  % UPSUDRM=2,
  % WTTRM=6,
  % XFELRM=54,
  % EPRM=20,
]
\begin{wpobjectives}
 \begin{compactitem}
   \item develop projects for creating open science services built out of Jupyter components and exploring new models for such services
   \item develop tools for interactive visualization in Jupyter
   \item develop workflows for data science using Jupyter software
 \end{compactitem}
\end{wpobjectives}

\begin{wpdescription}

Open source software in general, and Jupyter in particular,
is developed not as a monolithic application,
but rather as a collection of related components,
which can be assembled in numerous combinations to meet diverse needs.
The Jupyter community is no different.
Jupyter itself is composed of several projects,
but there are even more projects that build on top of Jupyter to create
things like cloud services or data pipelines.
The goal of \TheProject is to facilitate open science through Jupyter,
and this includes working with projects all around the Jupyter ecosystem.
We will focus this work package on developing
Jupyter ecosystem projects with an emphasis on open science.

repo2docker is a project for creating
reproducible environments in which Jupyter notebooks (and other user interfaces) can be run.
It reads a number of common formats to list required software packages,
and prepares a Docker container with those packages installed.
BinderHub is software for operating a web service using repo2docker,
which enables sharing of interactive and reproducible Jupyter (and Rstudio) environments on the web with a single link.
We will develop repo2docker and BinderHub further to meet the needs of the open science community.

Widgets are an extension to Jupyter, which can define new kinds of interactivity.
3D visualisation of data is important to many kinds of science,
and there is lots of room for development of the 3D visualisation landscape in Jupyter.

In addition to the interactive aspects of Jupyter,
notebooks can be used in a "workflows" style,
where job systems run analyses and produce reports,
either on a scheduled basis or triggered by events.
There is a great deal of interest in using notebooks in this way,
and much room for development of tools supporting workflows in data-driven open science.

\end{wpdescription}

\begin{tasklist}
% add tasks from task directory here
% % template for a task
% each task should be added to exactly one workpackage
% in the workpackage task list
\begin{task}[
  title=Sample Task,
  % task id for references
  id=task-id,
  % lead institution ID
  lead=SRL,
  PM=1,
  wphases={0-36},
  % partner institution ID(s)
  % don't include lead here
  partners={XXX}
]
  The task includes the following activities
  \begin{compactitem}
  \item ...
     % deliverable will be defined in the appropriate WorkPackage.tex
    % (\localdelivref{deliv-id})
  \end{compactitem}
\end{task}

\begin{task}[
  title=Further development of repo2docker and Binder,
  id=r2d-and-binder,
  lead=SRL,
  PM=36,
  wphases={0-36!.75},
  partners={QS,MP}
]
  Running someone else's analyses is a particularly difficult problem.
  There are differences between operating systems, versions of installed software and access to the required data sets.
  These challenges mean that is currently considered to be beyond the scope of an expert peer reviewer to verify data science analysis codes before publication.
  BinderHub, part of Project Jupyter, enables one-click running of git repositories.
  BinderHub provides a web interface to the repo2docker tool.

  The task includes the following activities
  \begin{compactitem}
  \item extend repo2docker with support for execution on cloud resources
  \item extend repo2docker with support for execution on HPC resources with Docker support
  \item improved "first use" experience of running repo2docker locally
  \item add support for using archives such as Zenodo as source for repo2docker and BinderHub
  \item define procedures and recommendations for long term reproducibility and sustainability of repo2docker compatible repositories
  \item create educational material describing repo2docker and its benefits to researchers
  \item Enable Openshift based deployments of BinderHub
  \item User surveys about pain points using BinderHub
  \item User authentication in BinderHub
  \end{compactitem}
\end{task}

\input{tasks/xeus}
\input{tasks/widgets}
% template for a task
% each task should be added to exactly one workpackage
% in the workpackage task list
\begin{task}[
  title=Improving robustness of reproducibility in repo2docker,
  id=reproducibility,
  lead=MP,
  PM=42,
  wphases={0-36},
  partners={SRL,QS,UIO}
]

  Reproducible research can inspire greater confidence in scientific results,
  and make it easier for future research to build on those results.

  Reproducibility is seen as an essential pillar of scientific truth,
  nevertheless there is a real shortcoming of truly reproducible
  research in the areas of computational and data science. In part,
  this is a cultural matter. However, there is also a lack of
  computational e-infrastructure supporting reproducibility.
  \medskip

  Jupyter Notebooks combine explanation with code and output and are
  thus valuable tools for making scientific computing more
  reproducible. However, the code in a notebook invariably relies on
  external code and hidden dependencies: libraries and programs which are not saved as part of
  the notebook.

  This task concerns ways to record the versions of these tools in use, and to
  make them available for practical reproduction of the computation.

  Binder and its tool repo2docker are a first step in the right
  direction: given the description of a computational environment,
  they allow to create that computational environment as a docker
  container automatically on demand, which in turn allows to execute a
  given notebook within this container environment. By archiving the
  notebooks together with the environment specification, the container
  computation environment can be created on demand. We see a number of
  publications being complemented by such Git repositories that allow
  reproducing figures and results from papers by re-executing
  notebooks; often archiving these repositories via the Zenodo
  service (for example publication \cite{CortsOrtuo2018}) with GitHub
  repository \cite{GitHubRepoExampleCortes2018}).

  Container technologies, such as Docker, offer exciting possibilities
  for capturing a computational environment, but much of the
  development of these tools is focused on short-term operational
  uses, not long-term preservation.

  There are currently at least two limitations in the existing
  repo2docker approach:
\begin{enumerate}
\item the environment specifications need to be written carefully and
  need to explicitly define particular version numbers of operating
  systems, libraries, and software to be used in the
  environment. While there are no guidelines (yet) for best practice
  in writing such specifications, in principle users can do this
  correctly.
\item when repo2docker builds a container environment, it relies on
  the required software being available on the Internet: Commands that
  clone software from GitHub assume that the software is actually
  available on GitHub. If a relevant repository disappears (or GitHub
  disappears), it will be impossible to clone that software from
  there, and this will break the binder execution and thus
  reproducibility. Some environments are specified through
  Dockerfiles, and often start from an Ubuntu Linux distribution
  container, followed by an \texttt{apt-get update} command. This
  command will fail once the age of the specified distribution exceeds
  the support period, and similarly subsequent \texttt{apt-get
    install} commands will fail. As the Binder service matures, we
  start to see such failures occur.
\end{enumerate}

The task addresses these problems and includes the following activities:
\begin{compactitem}

\item Literature review and technology exploration: research Binder
  model and horizon scan for related technology to support long term
  reproducibility.

\item Establish and document best practice for Binder use: Create
  public guidelines for building containers for scientific computing
  purposes so that they remain useful in the longer term, building on
  existing technology such as repo2docker.
   % ($\rightarrow$\localdelivref{binder-guidelines})

\item Facilitating reproducible creation and long-term archiving of
  container images for reproducibility: Develop new software to
  provide long-term executable computational environments that support
  the Binder model.

  There is a trade-off between preserving binary container images, and
  preserving the source code and instructions to build a container.
  Preserving sources is more transparent, and makes it easier to
  modify the code to explore a result, but without special care, the
  instructions may not continue to work in the future, or may not
  build an equivalent container.  We will explore both approaches,
  with a particular interest in how to make build instructions that
  can still work many years in the future.
   % ($\rightarrow$\localdelivref{jupyter-archive})

  It may be necessary for the build process to make use of alternative
  sources for source code and packages to install, such as snapshots
  of GitHub repositories that are available on Zenodo, or dedicated
  software archive servers that provide selected pieces of software
  that are required to build particular containers.

  \end{compactitem}


%\begin{compactitem}
%  \item
%    (\localdelivref{deliv-id})
%  \end{compactitem}

This technology will be co-developed with a real scientific use case at
European XFEL (see \taskref{applications}{reproducibility-xfel} in
\WPref{applications}).

  We note that there is a wide variety of use cases for this kind of
  technology and advances towards long term reproducibility, which we
  expect to grow in importance over time and with the increasing
  acceptance of open science: journals do increasingly (and rightly
  so) request from authors that they can reproduce their studies, or
  even submit corresponding code with the submission of the papers. As
  Binder and notebooks are a popular way of achieving this, the long
  term executability becomes a challenge. The same is true for
  universities and other academic institutions that take FAIR data
  seriously, and want to support the reproducibility and re-usability
  aspect of data comprehensively through providing data analysis
  software that allows access to the meaning of the data.
\end{task}

\begin{task}[
  title={Teaching tools, infrastructure, and best practices},
  id=teaching-tools,
  lead=EP,
  PM=21, % EP: 19PM, UPSud: 2PM
  wphases={0-36!.7},
  partners={UPSUD}
  ]

  This task is devoted to improving the Jupyter ecosystem for
  education. See page \pageref{sec:concept-demonstrator-teaching} for
  context and a list of other tasks that will contribute to better
  teaching.

  Setting up a comfortable working environment for both teachers and
  students requires tools for easy sharing, collecting, self
  assessment, and semi-automatic grading of course material, class
  management, and integration with the local e-learning infrastructure
  such as, e.g., Moodle or OpenEDX.

  We will \textbf{review the state of the art}: existing tools,
  within the Jupyter ecosystem (e.g. nbgrader \cite{Hamrick2016} or OK\cite{OKpy}) and outside;
  course services (e.g. Gryd\cite{Gryd} or CoCalc\cite{Cocalc}); course infrastructure that
  have been designed and deployed at many institutions (Berkeley,
  École Polytechnique, Université Paris Sud), etc.

  To build and share a better vision of the needs, we will \textbf{conduct a
  survey in the education community about the usage of Jupyter}. In
  particular we will seek feedback from Jupyter-based MOOCs (Massively Open Online Courses),
  e.g. on
  Coursera\footnote{\url{https://www.coursera.org/courses?query=jupyter}},
  and
  Fun\footnote{\url{https://www.fun-mooc.fr/courses/course-v1:inria+41016+session01bis/about}}.

  The collected requirements will be exposed in a first report
  \delivref{ecosystem}{teaching-report} and largely disseminated.

  The core of the task will then be to \textbf{further develop
    teaching tools, infrastructure, and course templates to contribute
    to the emergence of versatile solutions and best practices around
    them}.

  The outcome will be put into production by the participants of the
  \TheProject project (and beyond!) who will deliver a large number of
  courses using Jupyter technology (see \taskref{applications}{teaching}). The variety of
  use cases and infrastructure will provide a rich test bed and
  immediate feedback at each iteration, ensuring that the developments
  are informed and steered by demand (co-design), and battle field
  tested.

  At this stage, we already envision specific development in the
  following directions:
  \begin{compactitem}
  % \item Review and follow up on related efforts: gryd.us, cocalc, Coursera/Fun,
  %   Berkeley, Ecole polytechnique, University of Paris Sud, ...
  % \item Survey of the needs in the education community
  \item Collaborative grade management
  \item Insulation through container of the automatic grading
  \item Integration with e-learning platforms (e.g. Moodle, OpenEDX
    (Coursera/Fun)), through an LTI connector.
  \item Develop course templates for various use cases.
  \item Disseminate tutorials on all of the above.
  \end{compactitem}

  A second report \delivref{ecosystem}{nbgrader-like} will review the
  developments, our in-class experience with them, and best practices.
\end{task}

\end{tasklist}


\begin{wpdelivs}
\begin{wpdeliv}[due=12,miles=startup,id=binder-guidelines,dissem=PU,nature=DEC,lead=XFEL]
  {Guidelines for Binder use to improve reproducibility of
  environments, based on existing technology such as repo2docker}
\end{wpdeliv}
\begin{wpdeliv}[due=12,miles=startup,id=teaching-report,dissem=PU,nature=R,lead=EP]
  {Study of the practices of using Jupyter for teaching and the needs to
  effectively manage classes and associated courses in the education community}
\end{wpdeliv}
\begin{wpdeliv}[due=36,miles=community,id=nbgrader-like,dissem=PU,nature=OTHER,lead=EP]
  {Unified framework to effectively manage classes and associated courses
  using Jupyter technology}
\end{wpdeliv}
\begin{wpdeliv}[due=36,miles=community,id=jupyter-archive,dissem=PU,nature=OTHER,lead=XFEL]
  {Long-term reproducibility: Computational environment software
    archive system that extends lifetime of computational environments
  used in Binder service.}
\end{wpdeliv}
\begin{wpdeliv}[due=36,miles=community,id=k3d-jupyter,dissem=PU,nature=OTHER,lead=SIL]
  {Implement interoperable 3d visualization widget based on K3D-jupyter code}
\end{wpdeliv}


\end{wpdelivs}
\end{workpackage}
%%% Local Variables:
%%% mode: latex
%%% TeX-master: "../proposal"
%%% End:

%  LocalWords:  workpackage wphases wpobjectives wpdescription pageref wpdelivs wpdeliv
%  LocalWords:  dissem mailinglists swrepository final-mgt-rep compactitem swsites ipr
%  LocalWords:  TOWRITE tasklist delivref
