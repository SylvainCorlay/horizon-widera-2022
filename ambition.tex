\TOWRITE{ALL}{Proofread 1.4 Ambition pass 2}

\subsubsection{Ambition}

\TheProject's ambition is to \myemph{improve the reproducibility and accessibility} of scientific
tools, results, ideas, data and data analysis, and to 
\myemph{enable collaboration} among researchers and between researchers and the public.

We have these broad aims:
\begin{itemize}
\item Make it easier to perform computational research in a reproducible way
\item Educate researchers about reproducible practices
\end{itemize}

The world's computational resources are
constantly growing and science can produce ever-more interesting and useful 
data.  But \myemph{how do we enable the European and global
communities to make best use of that data?}  And \myemph{how do we enable the
public}, too?

Open science is a principle of making research
results as broadly accessible and useful as possible.  The first,
minimal step for this is making publications free to access.  The
second step for computational research is to make code and data
publicly accessible, enabling transparency and facilitating
\myemph{reproducibility} and verifiability of results.  But \myemph{merely making these
resources technically available is not the best we can do}.  There can
be many challenges with software, such as environment specifications
and resource requirements, that can be an impediment to the transition
from `technically available' to `practically useful.'  With the tools
of the open source open science community, we can do better.

\TOWRITE{}{add binder}

The Jupyter ecosystem consists of a \myemph{large, global community} of
developers and researchers producing software focused on interactive
computation and communication, and is \myemph{widely used by millions} of
individuals in numerous scientific fields, ranging from molecular
biology \cite{Wang2016} to materials science \cite{Hughes2014},
astronomy \cite{Baron2017} and climate science
\cite{Laken2015,Laken2015b}.  Jupyter software is permissively
licensed under the Berkeley Software Distribution (BSD) license,
allowing anyone to use Jupyter software for free, and even build
derivative commercial products, as has been done in the cases of
Google Colaboratory, Microsoft Azure Notebooks, IBM Watson Studio, and
others.  By contributing to the Jupyter ecosystem,
\TheProject maximises its impact, immediately benefiting the existing
large Jupyter community, and increasing the likelihood that
\TheProject's results will be maintained by the community after the
end of formal funding.

When it comes to Jupyter and Open Science for \myemph{reproducibility} we aim to improve the
\textit{status quo} by bringing the two together:

\begin{itemize}
\item Improve software in the Jupyter ecosystem to \myemph{improve reproducibility of environments}.
\item Enable researchers and the public to \myemph{better perform and benefit
  from Open and Reproducible Science}, through software, services, and education.
\end{itemize}

\TOWRITE{}{'practical reproducibility'}

Open Science that truly benefits society must be more than merely
technically accessible.  Individuals must be able to find the
resources they want and interact with them.  Ideally, they should be
able to ask new questions of models and data published by those that
came before.  This is where \TheProject fits in.  Excellent research
is being performed in all scientific fields, but making those results
practically accessible and engaging to others is a challenge.
Jupyter notebooks enable publishing code and data in a form that is
interactive, where readers can see code, run it, and see results.
They can then modify the code and produce new data and charts that the
first authors may not have considered.  Jupyter does not solve the
software installation problem, however, which can be significant for
scientific software.  For a publication to be truly interactive or
reproducible, it must include a computational environment (or a
sufficiently precise description of one such that it can be recreated)
in order to reliably be able to run for another individual.  Services
and tools such as Binder and repo2docker have started to demonstrate
that this can be facilitated: by
publishing a description of the requirements to run the software,
repo2docker is able to recreate a computational environment with
everything needed to run the software, including a Jupyter
environment for interactively exploring the resource.  Binder wraps
this in a web service, enabling immediate, free \myemph{reproduction and sharing} of
computational results on the web, with no requirements of readers
other than a web browser.
By developing these tools further, \TheProject enables researchers to
(i)~\myemph{make their results reproducible},
(ii)~allow other researchers to \myemph{reproduce or extend that research within minutes} or hours,
and (iii)~\myemph{enable the public to interact with the
science they are funding}.

By cooperating with specific applications from diverse domains in science and education,
we ensure and demonstrate that the work is valuable to a broad community of researchers, students, educators, and the public.
All together, Jupyter and Binder enable the migration of the open
science community from static publication to \myemph{truly interactive,
reproducible publications}.

The Jupyter notebook application is already TRL 8,
while the Binder software and service prototype at mybinder.org is TRL 6.
We will bring Binder to at least TRL 8 during the course of the project.


%%% Local Variables:
%%% mode: latex
%%% TeX-master: "proposal"
%%% End:

%  LocalWords:  eucommentary textsuperscript textregistered textsuperscript specialised
%  LocalWords:  textregistered recomputation textbf textbf rigourous centred flagshsip
%  LocalWords:  subsubsection realisation textit
