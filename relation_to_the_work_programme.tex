% Requirements to address here:
%
% Topic:
% This topic aims to fund activities to
% a) determine how increased reproducibility generates gains and savings in the R&I process and improve overall performance - alongside the demonstrated positive effects on their quality, integrity and trust-worthiness, and
% b) find, experiment and mainstream concrete solutions and best-practices to increase the reproducibility of research funded with European taxpayers money, including through the more systematic integration of sex and gender as variables whenever relevant.
% Consequently, actions should help understand and promote reproducibility by:
% 1) creating an open knowledge base of results, methodologies and interventions on the drivers and consequences of reproducibility for the R\&I system; and to fill the main gaps in such knowledge;
% 2) develop, validate, pilot and deploy practices and practical tools for funders, publishers and scientists;
% 3) promote uptake, greater collaboration, and increased alignment of the activities of stakeholders - scientific and technical communities, publishers and funders among others - to increase reproducibility.

\label{sect:workprogramme}
\subsubsection{Relation to the Work Programme}

\TOWRITE{}{old from BOSSEE}
The \TheProject project addresses the challenges of the
``Increasing the reproducibility of scientific results'' call (ID: HORIZON-WIDERA-2022-ERA-01-41).

Our strategy is based on taking the increasingly popular Jupyter
Notebook and Jupyter Ecosystem: we want to evolve and improve them
so that new innovative services based of the Jupyter tools can be developed for
the EOSC.
\medskip


There is evidence that the Jupyter Notebook is an e-infrastructure
that is useful across many domains: it is already widely adopted in
numerous communities and used by millions of researchers and educators worldwide
\cite{jupyter-grant}.

\begin{itemize}
\item \emph{Journalists} and practitioners of \emph{data-driven
    journalism} at the LA Times, BuzzFeed News, Columbia Journalism School \cite{latimes-datadesk} \cite{columbia-nytimes} \cite{data-journalism},
\item \emph{Research institutions} such as CERN, JRC, and many more,
  operating institution-wide Jupyter deployment,
\item \emph{Universities} using Jupyter as a teaching platform,
\item \emph{Large cloud providers} building commercial products on the
  top of Jupyter (Google DataLab and Colaboratory, Amazon Sagemaker, Microsoft Azure
  Notebooks),
\item \emph{Other EOSC projects}. Jupyter is already planned to become
  an important service on the European Open Science Cloud (for example
  the EOSC-04-funded PaNOSC project \cite{panosc}).
\item \emph{Data scientists}: some argue that the Jupyter Notebook is
  \emph{the} tool of choice for data scientists across domains
  \cite{Perkel2018}.
\item Over 3 million notebooks are deposited on GitHub \cite{notebookcount}.
\end{itemize}
%
\begin{figure}[tb]
  \centering\includegraphics[height=0.2\textheight]{images/jeodpp.png}
  \centering\includegraphics[height=0.2\textheight]{images/jeodpp-demo.jpg}
  \caption{\emph{Left}: The Joint Research Centre (JRC) Earth Observation
    Data and Processing Platform (JEODPP) is a heavy user of the
    Jupyter Notebook (source:
    \url{https://cidportal.jrc.ec.europa.eu/home/}), where it features
    at the top of the pyramid to help users with interactive data
    visualisation and analysis. \emph{Right}: An example
    service in which an interactive visualisation is provided through
    the Jupyter Notebook rendering of the density map of the ships
    detected from Sentinel-1 images over the Mediterranean sea during
    the period October 2014 to September 2016. \cite[Figure
    6]{Soille2018}. \label{fig:jeodpp}}
\end{figure}
%
A particular example is the Joint Research Centre Earth Observation
Data and Processing Platform (JEODPP) shown in Fig.~\ref{fig:jeodpp},
illustrating the interactive data exploration within an environment
that allows to save and communicate the data exploration
conveniently. These projects are building upon Jupyter as it is
available at the moment.
\bigskip

\eucommentary{
a) determine how increased reproducibility generates gains and savings in the R\&I process and improve overall performance - alongside the demonstrated positive effects on their quality, integrity and trust-worthiness, and
}

\eucommentary{
b) find, experiment and mainstream concrete solutions and best-practices to increase the reproducibility of research funded with European taxpayers' money, including through the more systematic integration of sex and gender as variables whenever relevant.
}

\eucommentary{
Consequently, actions should help understand and promote reproducibility by:
1) creating an open knowledge base of results, methodologies and interventions on the drivers and consequences of reproducibility for the R\&I system; and to fill the main gaps in such knowledge;
}

\eucommentary{
2) develop, validate, pilot and deploy practices and practical tools for funders, publishers and scientists;
}

\eucommentary{
3) promote uptake, greater collaboration, and increased alignment of the activities of stakeholders - scientific and technical communities, publishers and funders among others - to increase reproducibility.
}

%
% % Lots of EOSC and EU-funded projects are built upon jupyter
% %
% % Opendreamkit
% % PaNOSC
% % JEODPP https://www.sciencedirect.com/science/article/pii/S0167739X1730078X?via%3Dihub
% % EOSC-Pilot
% % EGI https://ec.europa.eu/info/funding-tenders/opportunities/portal/screen/opportunities/topic-details/infraeosc-02-2019;freeTextSearchKeyword=innovative;typeCodes=0,1;statusCodes=31094501,31094502;programCode=null;programDivisionCode=null;focusAreaCode=null;crossCuttingPriorityCode=null;callCode=Default;sortQuery=openingDate;orderBy=asc;onlyTenders=false
% %
% % Jupyter is a critical piece of European e-infrastructure; this project is important for sustainability, we need not just to build upon Jupyter but to consolidate the foundations.
% %
% % We also want to enable novel use cases to enable advances in European computational and data science activities that build on the Jupyter ecosystem.
% %
% % Who is better placed than the team who built Jupyter in the first place to move Jupyter forward?
% %
% % JEODPP Image (jeodpp_new_small_4.png)
% %
% %
% % The main challenge we need to address is “Develop an agile, fit-for-purpose and sustainable service offering accessible through the EOSC hub that can satisfy the evolving needs of the scientific community by stimulating the design and prototyping of novel innovative digital services. Innovative models of collaboration that genuinely include incentive mechanisms for a user oriented open science approach should be considered.” (from Specific challenge in: https://ec.europa.eu/info/funding-tenders/opportunities/portal/screen/opportunities/topic-details/infraeosc-02-2019;freeTextSearchKeyword=innovative;typeCodes=0,1;statusCodes=31094501,31094502;programCode=null;programDivisionCode=null;focusAreaCode=null;crossCuttingPriorityCode=null;callCode=Default;sortQuery=openingDate;orderBy=asc;onlyTenders=false)
% %
% % We need to make sure to either put this phrase in and respond to how we address it, or drop the right keywords.
% %
% %
% % \TODO{We should also go through the requirements from the call [1] and
% %   show how we address those [to provide easily accessible evidence
% %   that we are addressing the call].}
% %
