

\subsubsection{Objectives}\label{sect:objectives}

% \noindent The aims of \TheProject are to:

% \TOWRITE{remove aims}
% \begin{compactenum}[\myemph{Aim} 1:]
% \item Enable a
%   \myemph{sustainable}, \myemph{community-developed}, general purpose, \myemph{interoperable} toolbox for
%   interactive computing, data processing, and visualization
%   \myemph{that facilitates the entire life-cycle of Open Science},
%   from initial exploration to \myemph{reproducible publication}, research and development in
%   industry, teaching, and outreach.
%   This is by supporting and steering the Jupyter software ecosystem,
%   which exists to develop open source software,
%   open standards, and services for interactive computing across dozens of programming languages.
%
% \item Leverage this technology for all scientists, across borders,
%   domains, disciplines, and demographics, through
%   \myemph{free public distributed collaborative services} tightly integrated
%   into the European Open Science Cloud (EOSC),
%   in collaboration with a federation of related services
%   operated by the wider community.
%
% \item Demonstrate the value and versatility of such services through
%   \myemph{innovative co-designed tailored applications} in a variety of disciplines and
%   contexts.
%
% \item \myemph{Support Open Science} and maximize impact through development and
%   dissemination of best practices,
%   \myemph{training}, and \myemph{community building}
%   around the usage and development of the above toolbox,
%   with a focus on \myemph{interoperability}, \myemph{reproducibility}
%   and \myemph{reusability}.
% \end{compactenum}
% develop and support the Jupyter ecosystem in a direction that benefits and facilitates open science
%     make these tools accessible to as many people as possible via operation of free, public services
%     demonstrate and ensure that these developments are useful to real scientists and the public
%     foster open science through training of students and researchers in best practices using these tools


% \item \label{aim:facilitation}
%   Facilitate Open Science through the development
%   of tools enabling reproducibility, sharing, and collaboration.

% \item \label{aim:accessibility}
%   Maximise accessibility and interoperability of Open Science services and tools,
%   across domains, disciplines, and demographics.

% \item \label{aim:sustainability}
%   Maximise sustainability of software tools for Open Science
%   by developing the community and contributing
%   to and supporting community-led software efforts.



%   Support open source software for open science, and notably the
%   Jupyter ecosystem,



% \end{compactenum}

\medskip
\noindent We will achieve our ambition through the following objectives:

\TOWRITE{mention TRL goals}

\begin{compactenum}[\myemph{Objective} 1:]

\item \label{obj:reproducibility} \myemph{Facilitate Reproducibility and FAIR data} ---
  improving the \myemph{reproducibility of computational environments}
  used for science, and facilitating \myemph{FAIR data practices}.
  We will contribute to the recording and reproducibility
  of environments with repo2docker and Binder,
  and extend capabilities to better support FAIR
  data requirements. In particular, the archival of execution
  environments to support \myemph{reusability} of notebooks in the future
  needs attention. Such notebooks may, for example, be published alongside
  traditional publications to detail the computation of published data
  and figures, and address the Reusablity requirement of FAIR data.

\item \label{obj:broaden} broaden the impact of existing tools for reproducibility by expanding the applicable domains and use cases
\item \label{obj:demonstrators}
  \myemph{Demonstrators in science and education} ---
  We will demonstrate and ensure the versatility and value of the components and
  the services built from them,
  through applications to a number of
  domains in academic research, education, research infrastructures, SMEs, and for
  the public sector, driven through our project partners. In
  particular, we will contribute demonstrators in the following areas:
  % astronomy (\taskref{applications}{astro}), education
  % (\taskref{applications}{teaching}), fluid dynamics
  % (\taskref{applications}{application-gpu}), geosciences
  % (\taskref{applications}{geoscience}), health
  % (\taskref{applications}{opendose-analysis}), mathematics
  % (\taskref{applications}{math}),
  % and photon science (\taskref{applications}{reproducibility-xfel}),
  % involving universities, research infrastructure facilities, and SMEs.

\item \label{obj:education}
  \myemph{Outreach, engagement, and sustainability} ---
  Reach out to scientists and the wider research
  communities to encourage engagement
  and exploitation of existing tools for Reproducible and Open Science
  for their research domains and interests.
  Educate the research communities about reproducible practices,
  and available tools for reproducible publications and policies.
  Engaging a larger community will help \myemph{ensure the sustainability} of
  the services and underlying infrastructure by distributing its
  development, hosting, and maintenance over stakeholders from a
  variety of institutions and backgrounds,
  from the private sector to public research, education
  and open government.



\end{compactenum}

\begin{table}
  \label{tab:objectives-tasks}
  \caption{
  Each objective and the tasks which further that goal.}
  \begin{tabular}{|m{.3\textwidth}|m{.7\textwidth}|}

    \hline

    \myemph{Objective} & \myemph{Tasks}
    \\\hline

    \ref{obj:reproducibility} &

    % \longtaskref{core}{maintenance}
    % \longtaskref{core}{jh-bh-conv},

    \\\hline

    \ref{obj:broaden} &

    % \longtaskref{core}{accessibility},
    % \longtaskref{core}{collaboration},
    % \longtaskref{ecosystem}{xeus-cpp},
    % \longtaskref{ecosystem}{jupyter-widgets},
    % \longtaskref{ecosystem}{teaching-tools}

    \\\hline

    \ref{obj:demonstrators} &
    % \longtaskref{applications}{astro},
    % \longtaskref{applications}{teaching},
    % \longtaskref{applications}{application-gpu},
    % \longtaskref{applications}{geoscience},
    % \longtaskref{applications}{opendose-analysis},
    % \longtaskref{applications}{math},
    % \longtaskref{applications}{reproducibility-xfel}

    \\\hline

    \ref{obj:education} &

    % \longtaskref{education}{workshops},
    % \longtaskref{education}{online-resources},
    % \longtaskref{education}{helpdesk}

    \\\hline

  \end{tabular}
  \TODO{HF: Is this table compulsory?} 
\end{table}
