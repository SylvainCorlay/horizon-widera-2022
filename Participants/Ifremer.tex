\begin{sitedescription}{IFR}

\begin{center}
\includegraphics[height=3cm]{Participants/Logos/SRL.png}
\end{center}


Ifremer, French National Institute for Ocean Science, is a
French public scientific and technological institution that works for
exploring, understanding and predicting the ocean.
A pioneer in ocean science, IFREMER’s cutting-edge research is
grounded in sustainable development and open science.
Ifremer's vision is to advance science, expertise and innovation to:
1) Protect and restore the ocean
2) Sustainably use marine resources to benefit society
3) Create and share ocean data, information & knowledge.
With more than 1,500 personnel spread along the French coastline
in more than 20 sites, the institute explores the 3 great oceans:
the Indian, Atlantic and Pacific oceans. A leader in ocean science,
IFREMER is managing the French Oceanographic Fleet and its dedicated
scientists create ground-breaking technology to push the boundaries
of ocean exploration and knowledge, from the abyss to the atmosphere-ocean interface.

Well-established in the international scientific community, IFREMER's scientists,
engineers and technicians are committed to advance knowledge about our planet’s
last unexplored frontiers. They provide the science we need for informed
decision-making and public policy and they transfer this knowledge and technology
to businesses to fulfill public and private needs. IFREMER's core mission is also to
strengthen public awareness about the importance of understanding the ocean and
its resources, and empowering future generations of leaders through education
and outreach national campaigns.

Founded in 1984, IFREMER is a French public organization and its budget
approximates 240 million euros. It is operating under the joint authority of
the French Ministry for Higher Education, Research and Innovation,
the French Ministry for the Ecological and Solidary Transition, and the
French Ministry of Agriculture and Food.
 should i get no of phD studet, master, post doc, starat ups of Ifremer here?

% lab (UMR-LOPS) , department (IRSI) info to be added?

\subsubsection*{Curriculum vitae}
% Curriculum of the personnel at this institution

%\begin{participant}[gender=female]{Tina Odaka}

  \medskip PhD, is a highly experienced research engineer on big data analysis for oceanography with solid back ground on high performance computing architectures and softwares.  

  With a solid background in Computer Sciences, she worked in various application fields, including computational ocean modeling, chemistry, bio informatics and recently in biologging data analysis.

After obtaining her PhD from co-supervision between Germany and Japan in the field of theoretical chemistry, Tina made her post-doc in satellite data processing and parallelisation of ocean models.  Tina has been working since 2008 at IFREMER as a HPC architect and HPC research consultant at PCDM( Marine Data Infrastructure for data storage, processing and computation) and played a leading role as scientific computation experts. 
  Her current interest is optimisation of workflow for research and development for computationl oceanoography and next generation computational architecture for digital innovation of marine science.  
 She has been organaising multiple workshop and tutorial sessions for PCDM and keen for education of scientific computing for researchers and engineers in marine science.  
 She have initiated and co-lead Data-AI-Modelisation working group at Laboratory for Ocean Physics and Satellite remote sensing(LOPS) since 2019. In her activity she promote 
\href{https://pangeo.io/}{The Pangeo}  ecosystem, an interactive computing software stack for HPC and public cloud infrastructures, and study optimised interactive workflows for marine research which does not interrupts the natural, iterative nature of the scientific process of data exploration. 
She is also a PI of integration of Pangeo European at EOSC infrastructure with EGI-ACE, and integration of interactive pangeo based model data analysis on Fugaku.  

   
\end{participant}

%%% Local Variables:
%%% mode: latex
%%% TeX-master: "../proposal"
%%% End:

% We will have total 6 moonth (4 persons) from computing department (IRSI) involved to set up / feed back
% the binder@HPC center
% Also 3 month (1-2 person) from research lab (UMR-LOPS) who use cloud infrastructures to set up /feed back
% the binder@cloud


\begin{participant}[PM=72, type=R]{Research Engineer }

We will hire one research engineers to carry out the work
at Ifremer, under the leadership of and together with Dr. Odaka.  The
fellow will have a background in geoscience, combined with Xarray
and Jupyter Notebook experience, and past experience of software engineering.
An ideal candidate will also have good communication skills and team working
abilities, and in particular interest and skill in the development and
operation of software services to best support this part of the project.

\end{participant}


\subsubsection*{Publications, products, achievements}
\begin{compactenum}
\item Odaka Tina, Banihirwe Anderson, Eynard-Bontemps Guillaume, Ponte Aurelien,
Maze Guillaume, Paul Kevin, Baker Jared, Abernathey Ryan (2019).
The Pangeo Ecosystem: Interactive Computing Tools for the Geosciences:
Benchmarking on HPC.
Juckeland G., Chandrasekaran S. (eds) Tools and Techniques for High Performance Computing.
HUST 2019, SE-HER 2019, WIHPC 2019. Communications in Computer and Information Science,
vol 1190. Springer, Cham. Print ISBN 978-3-030-44727-4 Online
ISBN 978-3-030-44728-1. https://doi.org/10.1007/978-3-030-44728-1_12. pp.190-204 .

\item Eynad-Bontemps Guillaume, Odaka Tina (2020).
Analyse de données scientifiques à l’échelle sur HPC ou dans le Cloud avec Pangeo.
Teratec Digital Forum 2020. 13 & 14 octobre 2020, Online .

% pangeo@fugaku
% datarmor paper


\end{compactenum}

\subsubsection*{Relevant projects or activities}

\begin{compactenum}
\item Pangeo Community (European)
Pangeo (https://pangeo.io/), a world-wide community driven platform initially
developed for Geoscience  has a public pangeo deployments based at US.
This project  is to demonstrate how to deploy and use Pangeo on EOSC and
underlined the benefits for the European community. (P.I. Tina Odaka, on going use case at EGI-ACE)
\item PCDM: Marine Data Infrastructure for data storage, processing and computation
Technical Lead, Tina Odaka 2009-2019.


\end{compactenum}

\subsubsection*{Significant infrastructure}

\item PCDM: Marine Data Infrastructure for data storage, processing and computation
https://wwz.ifremer.fr/pcdm
Marine computing center located in IFREMER, hosting more than 12PB, more than 500 users,
multiple research institute for Marine studies in France. Pangeo platform already deployed
(Jupyterhub and Python environment over HPC resources).
\item

\end{sitedescription}

