\begin{sitedescription}{SRL}

\begin{center}
\includegraphics[height=3cm]{Participants/Logos/SRL.png}
\end{center}

Simula is an internationally-leading Norwegian research institute in the key
ICT areas: communication systems, scientific computing and software
engineering. Simula's research areas have been evaluated with the highest
score by international expert panels in several national evaluations.

Dedicated to tackling scientific challenges with long-term impact and of
genuine importance to real life, Simula offers an environment that emphasises
and promotes basic research. This translates into numerous projects funded by
the EU, Norwegian government or regional institutions, that Simula was
involved in. In 2017, it successfully concluded Norwegian Centre of Excellence
for Biomedical Computing and is currently hosting the Centre for
Research-based Innovation, Certus. In addition, Simula is deeply involved in
research education with 35 PhD students, 40 master's students, and 20
postdoctoral fellows supervised annually; and application-driven innovation
and commercialisation, where it owns parts of 16 start-up companies with 110
employees.

The Department for Numerical Analysis and Scientific Computing (SCAN) aims to
develop mathematical methods and scientific tools to reach new understanding
of complex physical processes. It targets fundamental medical and industrial
problems where new insights from mathematical modelling can advance today's
knowledge. The department has hosted a ten-year Norwegian Centre of Excellence
in Biomedical Computing (2007-2017), one of the most prestigious research
environments in Norway, targeting ambitious and groundbreaking research. The
department received top scores in all six evaluations carried out by the
Research Council of Norway and is running a multitude of national and
international research projects, including one ERC Starter Grant project.

\subsubsection*{Curriculum vitae}
% Curriculum of the personnel at this institution

\input{CVs/Min.RK.tex}

% \input{CVs/Katarina.Subakova.tex}

\begin{participant}[PM=72, type=R]{Research Engineer x2}

We will hire two postdoctoral-level research engineers to carry out the work
at Simula, under the leadership of and together with Dr. Ragan-Kelley.  The
fellow will have a background in computational science, combined with IPython
and Jupyter Notebook experience, and past experience of software engineering.
An ideal candidate will also have good communication skills and team working
abilities, and in particular interest and skill in the development and
operation of software services to best support this part of the project.

\end{participant}


\subsubsection*{Publications, products, achievements}

\begin{compactenum}
\item 2017 ACM Software System Award for Jupyter
\item M. Bussonier, J. Forde, J. Freeman, B. Granger, T. Head, C. Holdgraf, K.
  Kelley, G. Nalvarte, A. Osheroff, M. Pacer et al. Binder 2.0 - Reproducible,
  interactive, sharable environments for science at scale In Python in Science
  ConferenceProceedings of the 17th Python in Science Conference. Austin,
  Texas: SciPy, 2018.
\item J. Forde, T. Head, C. Holdgraf, Y. Panda, G. Nalvarte, M. Pacer, F.
  Perez, B. Ragan-Kelley and E. Sundell. Reproducible Research Environments
  with Repo2Docker In ICML 2018 Reproducible Machine Learning. ICML, 2018.
\item T. Kluyver, B. Ragan-Kelley, F. Perez, B. Granger, M. Bussonier, J.
  Frederic, K. Kelley, J. Hamrick, J. Grout, S. Corlay et al. Jupyter
  Notebooks: a publishing format for reproducible computational workflows In
  20th International Conference on Electronic Publishing. IOS Press, 2016.

\end{compactenum}

\subsubsection*{Relevant projects or activities}

\begin{compactenum}
\item OpenDreamKit (GA No. 676541) Open Digital Research Environment Toolkit for the Advancement of Mathematics, participant, Work Package lead
\item Jupyter - collaboration with UC Berkeley, Cal Poly, funded by Gordon \&
  Betty Moore Foundation, Alfred P. Sloan Foundation, and Helmsley Trust
\item Binder - collaboration with UC Berkeley, funded by Gordon \& Betty Moore
  Foundation
\end{compactenum}

\subsubsection*{Significant infrastructure}

The fully owned Simula subsidiary Simula Innovation handles pre-commercial
innovation projects, creation and follow-up of company spin-offs, and general
support for entrepreneurs.

\end{sitedescription}



%KEY-MORE-TODOS


%%% Local Variables:
%%% mode: latex
%%% TeX-master: "../proposal"
%%% End:

%  LocalWords:  sitedescription Simula Simula commercialisation Certus subsubsection Logg
%  LocalWords:  Mardal Funke Rognes Sci Comput Langtangen FEniCS Aln ae lgaard vspace
%  LocalWords:  TOWRITE emphasises organised
