\begin{participant}[type=leadPI,PM=5,gender=male]{Hans Fangohr}
  % type is one of:
  % - leadPI: leader of the participating institution
  % - PI: Principal Investigator
  % - R: researcher?
  % Who is the coordinator is specified elsewhere

  % PM=YYY:
  % A fair evaluation of the number of months you will be
  % spending on this specific project along the four years.
  % Typical numbers:
  % - full time hired personnel: 48 months
  % - lead PI or proposal coordinator: 8-12 months
  % - PI: 4-5 months
  % - participant: 2-6 months

  % salary=ZZZ:
  % Approximate monthly gross salary (in term of total cost for the
  % employer). This is optional. If you are uncomfortable having this
  % information in a public file, you can alternatively send the
  % information to Eugenia Shadlova, or to your institution
  % leader/manager if he is willing to fill in himself the budget
  % forms on the eu portal.

  % The above information is used to fill in various tables in the
  % proposal file, and to evaluate the cost of the project for the
  % institutions.

  % You may remove all those comments.

  % About half a page of free text; for whatever it's worth, you may see
  % Nicolas.Thiery.tex for an example.


  \medskip Hans Fangohr is leading the Scientific Support Unit for Computational
  Science at the Max Planck Institute for the Structure and Dynamics of matter
  since 2021, and is a an academic at the University of Southampton in the United Kingdom
  since 2002 (full professor since 2010). He was leading the data analysis group
  and services at European XFEL in Germany from 2017 to 2020.

  He has been a long term proponent of open science, and in particular involved
  with the the use and further development of the Jupyter Notebook to enable
  this. He has hosted Thomas Kluyver at the University of Southampton since 2015
  from where he contributed as a core developer of the Jupyter team. As a PI in
  the EC-funded e-INFRA OpenDreamKit project (2015-2019), he has pushed forward
  the use of Jupyter Notebooks for reproducible computational science, and
  started the notebook validation tool (NBVAL). He made use of the Jupyter
  Ecosystem for research and education at graduate and postgraduate level at the
  University of Southampton, and shared resources widely, including a text book
  provided through Jupyter Notebooks, which can be executed interactively
  online. He has been awarded 4 prizes for the quality of his teaching.

  From 2017 to 2020, he was designing data analysis services and
  infrastructure at the European XFEL research facility. European XFEL
  is using Python and the Jupyter Notebook as core utilities in their
  large scale experiment control, data capture and data
  analysis. 

  Hans has explored, pioneered and advocated the use data repositories as part
  of scientific publications, which contain essential pieces of software and
  data which allow to reproduce key insights of the related publication.
  \TODO{Add examples? First one is from 2016.}

  In this project (\TheProject), where new capabilities for the Binder tools
  are being developed, Hans' wide experience and
  interaction with different science groups will be beneficial to
  ensure the outcome is of value to open science in many domains. This
  experience includes him chairing the interdisciplinary computational modelling
  group at the University of Southampton (200 academics, 2008-2017),
  chairing the national EPSRC scientific advisory committee on High
  Performance Computing in the UK (2014-2017) and interacting with a
  large variety of science users at European XFEL in his role
  to lead the data analysis service provision.

  Relevant publications:

  [1] Marijan Beg, Juliette Belin, Thomas Kluyver, Alexander Konovalov, Min Ragan-Kelley, Nicolas Thiery, Hans Fangohr
  Using Jupyter for reproducible scientific workflows 
  Computing in Science & Engineering 23, 36-46 (2021)

  [2] H. Fangohr et al, 
  Data exploration and analysis with Jupyter notebooks 
  Proceedings of the 17th International Conference on Accelerator and Large Experimental Physics Control Systems ICALEPCS2019, TUCPR02 (2020)

  [3] A Goetz, JF Perrin; H Fangohr; D Salvat, F Gliksohn, A Markvardsen, A McBirnie, A Gonzalez-Beltran, J Taylor
  PaNOSC FAIR research data policy framework 
  https://doi.org/10.5281/zenodo.3862701, (2020)

  [4] Hans Fangohr, Vidar Fauske, Thomas Kluyver, Maximilian Albert, Oliver Laslett, David Cortes-Ortuno, Marijan Beg, Min Ragan-Kelly
  Testing with Jupyter notebooks: NoteBook VALidation (nbval) plug-in for pytest 
  https://arxiv.org/abs/2001.04808, (2020)
\end{participant}

%%% Local Variables:
%%% mode: latex
%%% TeX-master: "../proposal"
%%% End:
