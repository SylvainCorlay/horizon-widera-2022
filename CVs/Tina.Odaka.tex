\begin{participant}[gender=female]{Tina Odaka}

  \medskip PhD, is a highly experienced research engineer on big data analysis for oceanography with solid back ground on high performance computing architectures and softwares.  

  With a solid background in Computer Sciences, she worked in various application fields, including computational ocean modeling, chemistry, bio informatics and recently in biologging data analysis.

After obtaining her PhD from co-supervision between Germany and Japan in the field of theoretical chemistry, Tina made her post-doc in satellite data processing and parallelisation of ocean models.  Tina has been working since 2008 at IFREMER as a HPC architect and HPC research consultant at PCDM( Marine Data Infrastructure for data storage, processing and computation) and played a leading role as scientific computation experts. 
  Her current interest is optimisation of workflow for research and development for computationl oceanoography and next generation computational architecture for digital innovation of marine science.  
 She has been organaising multiple workshop and tutorial sessions for PCDM and keen for education of scientific computing for researchers and engineers in marine science.  
 She have initiated and co-lead Data-AI-Modelisation working group at Laboratory for Ocean Physics and Satellite remote sensing(LOPS) since 2019. In her activity she promote 
\href{https://pangeo.io/}{The Pangeo}  ecosystem, an interactive computing software stack for HPC and public cloud infrastructures, and study optimised interactive workflows for marine research which does not interrupts the natural, iterative nature of the scientific process of data exploration. 
She is also a PI of integration of Pangeo European at EOSC infrastructure with EGI-ACE, and integration of interactive pangeo based model data analysis on Fugaku.  

   
\end{participant}

%%% Local Variables:
%%% mode: latex
%%% TeX-master: "../proposal"
%%% End:
