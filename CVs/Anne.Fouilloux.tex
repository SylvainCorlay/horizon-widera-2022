\begin{participant}[gender=female]{Anne Fouilloux}
  % type is one of:
  % - leadPI: leader of the participating institution
  % - PI: Principal Investigator
  % - R: researcher?
  % Who is the coordinator is specified elsewhere

  % PM=YYY:
  % A fair evaluation of the number of months you will be
  % spending on this specific project along the four years.
  % Typical numbers:
  % - full time hired personnel: 48 months
  % - lead PI or proposal coordinator: 8-12 months
  % - PI: 4-5 months
  % - participant: 2-6 months

  % salary=ZZZ:
  % Approximate monthly gross salary (in term of total cost for the
  % employer). This is optional. If you are uncomfortable having this
  % information in a public file, you can alternatively send the
  % information to Eugenia Shadlova, or to your institution
  % leader/manager if he is willing to fill in himself the budget
  % forms on the eu portal.

  % The above information is used to fill in various tables in the
  % proposal file, and to evaluate the cost of the project for the
  % institutions.

  % You may remove all those comments.

  % About half a page of free text; for whatever it's worth, you may see
  % Nicolas.Thiery.tex for an example.

  \medskip PhD, is a highly experienced Research Software Engineer dedicated to supporting
  researchers towards the adoption of Open Science best practices.

  With a solid background in Computer Sciences, she worked in various application fields, including environmental sciences, Intelligent Transport Systems, High-Performance computing, bio-informatics, meteorology and Geosciences.

  She is currently working for the \href{https://neic.no}{Nordic e-Infrastructure Collaboration} (NeIC) where she is leading the second phase of the \href{https://neic.no/nicest2/}{Nordic Collaboration on e-Infrastructures for Earth System Modeling Tools} project (NICEST2) and actively supports the \href{https://pangeo.io/}{Pangeo} community platform for Big Data geoscience. In 2021, she has been selected as a member of the Strategic Advisory Board of the \href{https://stories.ecmwf.int/destination-earth/index.html}{Destination Earth initiative}. 

She is also involved in a number of projects such as MAchine learning, Surface mass balance of glaciers, Snow cover, In-situ data, Volume change, Earth observation (MASSIVE, Research Council of Norway), \href{https://www.eosc-nordic.eu/}{EOSC-Nordic} and \href{https://www.reliance-project.eu/}{Research Lifecycle Management technologies for Earth Science Communities and Copernicus users in EOSC (RELIANCE, EU-funded, Grant number 101017501).
   Since 2015, Anne Fouilloux has been very active with \href{https://carpentries.org}{The Carpentries}, a diverse and global community of volunteers and she teaches foundational coding and data science skills to students and young researchers. She is a certified \href{https://carpentries.org/instructors/}{Carpentries instructor}, \href{https://carpentries.org/trainers/}{instructor trainer} and \href{https://carpentries.org/maintainers/}{maintainer}. 
   
\end{participant}

%%% Local Variables:
%%% mode: latex
%%% TeX-master: "../proposal"
%%% End:
