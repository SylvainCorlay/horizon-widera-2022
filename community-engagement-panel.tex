We have already secured agreement from the following to be part of the panel,
and will extend this if funded:
\begin{itemize}
\item Suzanne Dumouchel, Head of European Cooperation at TGIR Huma-Num CNRS unit, a large infrastructure for 
digital humanities and member of the EOSC Association Board of Directors. She is the partnerships coordinator 
of OPERAS Research Infrastructure, devoted to scholarly communication in Social Sciences and Humanities and is a member of DARIAH ERIC Coordination Office, 
dedicated to Digital Arts and Humanities. She is also the scientific coordinator of TRIPLE, H2020 project 
(INFRAEOSC2). Strongly committed to the Open Science movement and to the promotion of research in Social Sciences 
and Humanities (SSH), she is particularly active in the field of research infrastructures.
\item Andy Götz, Software Group Leader at European Radiation Synchrotron
  Facility (ESRF), coordinator of the EOSC project PaNOSC for making data from
  photon and neutron facilities FAIR, and chairman of the IT working Group of
  the ``League of European Accelerator-based Photon Sources'' (LEAPS). The LEAPS
  facilities wish to enable their users to create reproducible publications
  based on large data sets captured at the light sources.
\item Paula Andrea Martinez, Project Coordinator - Software Program, \href{https://ardc.edu.au/}{Australian Research Data Commons} (ARDC).
She is also the co-chair of the \href{https://www.rd-alliance.org/groups/fair-research-software-fair4rs-wg}{FAIR4RS RDA Working Group} and
Community Manager at \href{https://www.researchsoft.org/}{Research Software Alliance} (ReSA), and actively contributing
to increase the visibility of research software.
\item Aleksandra Nenadic, Training Lead of the Software Sustainability Institute, 
based at the University of Manchester (UK). She is also an active member and 
promoter of the Carpentries community and involved as an instructor, 
instructor trainer, mentor, workshop organiser and regional coordinator 
for the UK, driving and supporting new material creation using the 
Carpentries collaborative and pedagogical lesson development principles.
\item Gergely Sipos, head of services, solutions and support department at the
  EGI Foundation. He is representing EGI, ``Advanced Computing for EOSC''
  (EGI-ACE) and the EOSC Compute Platform, which are working on a large-scale
  deployment of BinderHub as part of their services for researchers in Europe
  and beyond.
\item Violaine Louvet, head of GRICAD (\href{Grenoble Alpe Research -
 Scientific Computing and Data Infrastructure}
 {https://gricad.univ-grenoble-alpes.fr/index_en.html}), supported by
 CNRS, Grenoble Alpes University and INRIA. GRICAD is a Tier 2
 infrastructure and provide data and computing resources to all the
 science communities in Grenoble. In particular, GRICAD provide HPC,
 HTC, cloud and storage resources for all the disciplinary fields,
 from computer sciences to human sciences and health. The structure
 also offer a JupyterHub and a BinderHub platform. We are also very
 involved in helping the scientific communities and in training.
 \item Rollin Thomas, Big Data Architect at HPC expert at the National Energy
  Research Scientific Computing Center at Lawrence Berkeley National Laboratory
  (US). He represents the HPC community, and focuses on interactivity,
  real-time, and reproducibility in supercomputing for science.
\item Andreas Zeller, Professor of Software Engineering. He uses Notebooks 
  to provide open-source text books to his students and the world-wide
  community of readers. He will represent Binder users in academia, who use it 
  to deliver zero-install computational environments for educational
  purposes.
\end{itemize}
