% template for a task
% each task should be added to exactly one workpackage
% in the workpackage task list
\begin{task}[
  title=Training Workshops for more reproducible science,
  id=workshops,
  lead=UIO,
  PM=9,
  wphases={12-36!.25},
  partners={SRL,MP,IFR}
]
This task is focused on taking the content from the
\taskref{education}{online-resources} (Best practice for
reproducible science guidelines) and disseminating it through various channels and to different target audiences.

% \begin{compactitem}

%    \item Defining and implementing a strategy to enable a shared vision of the Jupyter ecosystem across all the actors from developers, users to every stakeholder: the current misalignment hinders the full exploitation of Open Software practices where co-design is a de facto approach.
%
% For instance, the official Jupyter documentation (https://jupyter.org/documentation) solely reflects the view of developers where the Jupyter ecosystem is defined as a set of software packages (jupyter-core, jupyter-client, kernels, widgets (ipywidgets, ipyleaflet, etc.). The user vision is relegated to examplars (blogs, newsletters, etc.) which inevitably tend to be restrictive but often become de facto standards. This can lead to misconceptions and makes it more difficult for on-boarding novices and new communities.
%

% \item Triggering a cultural change to help under-represented groups to actively participate to the development of open source project to ensure the sustainability of the \TheProject services deployed on EOSC-HUB.
%

%\item Foster Open innovation by collaborating with others from different background and activities (school, universities, industries, journalists, artists, etc.)
%  \end{compactitem}

To achieve these goals, the following actions/activities will take place:

  \begin{compactitem}

  %\item co-design hackathons: the co-design efforts between domain scientists, \TheProject developers and service providers will be carried out at any point in time of the project and will be registered in a co-design register to help for future engagement with new communities of users. To be fully effective,  co-design hackathons will be organized to set the stage, define rules for co-design interactions and more importantly align all actors into a common user-centred vision of \TheProject services and associated development towards a successful EOSC deployment.

%    \item Workshops on Findable, Accessible, Interoperable and Reusable (FAIR)
%      software and data to facilitate the adoption of Open Science and Open
%      Scholarship best practices (transparent, sharable and collaborative
%      Science): this would not be restricted to the Jupyter ecosystem and will
%      teach users how to make data, lab notes and other research processes freely
%      available, under terms that enable reuse (licensing), redistribution and
%      reproducibility of methods and/or results.

   \item Delivery of workshops on (i) open science, (ii) reproducible computational
     science, and (iii) the use of Binder tools to support this.

     The content is focused on key insights and tools need for more reproducible
     science, but will be contextualised and delivered in the wider field of
     Findable, Accessible, Interoperable and Reusable (FAIR) software and data.

   % \item Trainings on how to use \TheProject software and services to fully
   %   exploit \TheProject developments for repoducible science: develop training
   %   materials and organize training events for researchers and the public to
   %   enable Open Science and maximise the usefulness of \TheProject
   %   developments.

   \item \TheProject Admin trainings: we will offer training events for learning on how to
     deploy \TheProject services such as BinderHub. This will be relevant for a
     very small (but important) group of users, i.e. those that want to host
     their own BinderHub instance. We know from multiple research organisations
     that this desire exists.

   \item Open call for open innovation mini-projects: mentored by \TheProject
     staff and targeting SMEs, municipalities, journalists, artists, etc.

   \item Where possible, we will schedule dissemination events to take place
     during conferences and community events, such as PyData, EuroSciPy,
     Supercomputing meetings.

   \item We will archive recordings of the training events to support the
     increasing desire of learners to make use of online streaming services
     (such as YouTube) to work through a learning programme at their own time
     and pace.

   \item We will offer in-person and remote training.

   \item The work will be done in collaboration with
     \href{https://coderefinery.org}{CodeRefinery} project which strongly
     support \TheProject proposal and will make available its network of instructors and helpers
     to co-organize, advertise and run online workshops on Open Science best practices. \TODO{This
       is great - can we get a letter of support?}

  \item We will detail our executed activities through the reporting at the end of
    each reporting period.
  \end{compactitem}
\end{task}
