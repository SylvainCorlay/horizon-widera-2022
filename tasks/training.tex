% template for a task
% each task should be added to exactly one workpackage
% in the workpackage task list
\begin{task}[
  title=Training Workshops and community building,
  id=workshops,
  lead=UIO,
  PM=36,
  wphases={0-36!.75},
  partners={SRL}
]
This task will be in charge of:

 
  \begin{compactitem}

   \item Defining and implementing a strategy to enable a shared vision of the Jupyter ecosystem across all the actors from developers, users to every stakeholder: the current misalignment hinders the full exploitation of Open Software practices where co-design is a de facto approach.

For instance, the official Jupyter documentation (https://jupyter.org/documentation) solely reflects the view of developers where the Jupyter ecosystem is defined as a set of software packages (jupyter-core, jupyter-client, kernels, widgets (ipywidgets, ipyleaflet, etc.). The user vision is relegated to examplars (blogs, newsletters, etc.) which inevitably tend to be restrictive but often become de facto standards. This can lead to misconceptions and makes it more difficult for on-boarding novices and new communities.


\item Triggering a cultural change to help under-represented groups to actively participate to the development of open source project to ensure the sustainability of the \TheProject services deployed on EOSC-HUB. 
 

\item Foster Open innovation by collaborating with others from different background and activities (school, universities, industries, journalists, artists, etc.)

  \end{compactitem}
 

To achieve these goals, the following actions/activities will take place:


  \begin{compactitem}
   \item co-design hackathons: the co-design efforts between domain scientists, \TheProject developers and service providers will be carried out at any point in time of the project and will be registered in a co-design register to help for future engagement with new communities of users. To be fully effective,  co-design hackathons will be organized to set the stage, define rules for co-design interactions and more importantly align all actors into a common user-centred vision of \TheProject services and associated development towards a successful EOSC deployment. 


   \item Workshops on Findable, Accessible, Interoperable and Reusable (FAIR) software and data to facilitate the adoption of Open Science and Open Scholarship best practices (transparent, sharable and collaborative Science): this would not be restricted to the Jupyter ecosystem and will teach users how to make data, lab notes and other research processes freely available, under terms that enable reuse (licensing), redistribution and reproducibility of methods and/or results.

   \item Trainings on how to use \TheProject software and services to fully exploit \TheProject developments for EOSC: develop training materials and organize training events for researchers and the public to enable Open Science and maximise the usefulness of \TheProject developments.

   \item \TheProject Admin trainings: training event for learning on how to deploy \TheProject services such as BinderHub.


   \item Open call for open innovation mini-projects: mentored by \TheProject staff and targeting SMEs, municipalities, journalists, artists, etc.

   \item Dissemination during conferences, such as EWASS (European Week of 
         Astronomy and Space Science), ADASS (Astronomical Data Analysis and 
         Software and Systems), and IVOA (International Virtual Observatory
         Alliance) meetings (for the astronomy demonstrator 
         \taskref{applications}{astro}).

  \end{compactitem}
 The work will be done in collaboration with \href{https://coderefinery.org}{CodeRefinery} project which strongly support \TheProject proposal and will be committing staff time for organizing and running workshops on Open Science best practices. 
\end{task}
