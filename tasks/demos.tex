% template for a task
% each task should be added to exactly one workpackage
% in the workpackage task list
\begin{task}[
  title=Science demonstrators,
  % task id for references
  id=demos,
  % lead institution ID
  lead=MP,
  PM=8,
  wphases={0-36},
  % partner institution ID(s)
  % don't include lead here
  partners={IFR,UIO}
]

  In this task, we want to demonstrate the value and usefulness of \WPref{reproducibility} and
  \WPref{impact} with real scientific use cases from the research communities involved in \TheProject.
  The demonstrators are designed to exploit the solutions developed within \TheProject (Binder@Home, Binder@HPC, data publishing)
  and leverage existing institutional and/or national e-infrastructures as well as core EOSC services.

  \begin{compactitem}
  \item FAIR Nordic Earth System Modelling: this science demonstrator leverages Binder@HOME (model development, education, single column or very simple model configuration), Binder@HPC (operational runs at scale including on EuroHPC), data publishing (publication of simulation results from blue-sky research);
  \item Marine physical and fish habitats modelling and analysis: In marine research there are well established reproducible research examples such as Argopy \TODO{ref to https://dx.doi.org/10.21105/joss.02425}  or pangeo ecosystems (http://gallery.pangeo.io/repos/pangeo-gallery/physical-oceanography/). Still in Ifremer, we have number of researchers who depends on commercial software for their data analysis and does not have access to publish reproducible research workflows.  We demonstrate these research use caseses and study their barriers and propose better accesible optimised research workflow that can benefit researchers themselves, but also include reproducible aspects.   The techinical tool kit for this activity will be co-developped with WP2 and WP3. We'll demonstrate these use cases on Binder@home and Binder@HPC. Ifremer is in charge of managing numerous marine databases and information systems and the HPC system have direct access to these datasets.  We will optimise the access to these datasets, and will study the possibility to extend that with the usage from cloud infrastructure, in connected efforts with Binder running on Clouds, such as EOSC. 
This effort and outcome of it will connect to the Digital Twin Ocean project to allow ocean data and models relevant to biodiversity to be re-used by researchers and engineers. This will provide a concept demonstration of ingesting ocean data and model output that can be reproduced through the existing ocean research infrastructures.; 

     % deliverable will be defined in the appropriate WorkPackage.tex
    % (\localdelivref{deliv-id})
  \end{compactitem}
\end{task}
