% template for a task
% each task should be added to exactly one workpackage
% in the workpackage task list
\begin{task}[
  title=Science demonstrators,
  % task id for references
  id=demos,
  % lead institution ID
  lead=MP,
  PM=8,
  %wphases={0-36},
  % partner institution ID(s)
  % don't include lead here
  partners={IFR,UIO}
]

  In this task, we want to demonstrate the value and usefulness of \WPref{reproducibility} and
  \WPref{impact} with real scientific use cases from the research communities involved in \TheProject.
  The demonstrators are designed to exploit the solutions developed within \TheProject (Binder@Home, Binder@HPC, data publishing) 
  and leverage existing institutional and/or national e-infrastructures as well as core EOSC services.
  Synergies between the different science applications and communities will be ensured through the technical tasks (Binder@Home, Binder@HPC, data publishing).
  e.g. 

\paragraph*{Context:}  For example, in marine research field there are reproducible research examples such as Argopy~\cite{maze2020},  Pangeo ecosystems (http://gallery.pangeo.io/repos/pangeo-gallery/physical-oceanography/). But even within the same research lab, we have number of researchers who depends on commercial software for their data analysis and does not have access to publish reproducible research workflows.   


\paragraph*{Task activity:}
 
Within the actual Binder capability, we will demonstrate following two reproducible research configurations.  We demonstrate these research use caseses and show barriers that has been preventing these workflow to be pulished as reproducible science.   

%These information will give first feed back to \WPref{reproducibility}, \WPref{impact} 
%ll enrich the process of propose better accesible optimised 
%research workflow that can benefit researchers themselves, but also include reproducible aspects. 


  \begin{compactitem}
  \item FAIR Nordic Earth System Modelling: this science demonstrator leverages Binder@HOME (model development, education, single column or very simple model configuration), Binder@HPC (operational runs at scale including on EuroHPC), data publishing (publication of simulation results from blue-sky research);
  \item Demonstration of marine physics and fish habitats modelling and analysis using Pangeo ecosystem.  
  
\TODO{Tina, move this to section 1? but where??
This effort and outcome of it will connect to the Digital Twin Ocean project to allow ocean data and models relevant to biodiversity to be re-used by researchers and engineers. This will provide a concept demonstration of ingesting ocean data and model output that can be reproduced through the existing ocean research infrastructures
should be able to add 'interdisciplinary' part in chap1  as it bridges biology and physics..}

     % deliverable will be defined in the appropriate WorkPackage.tex
    % (\localdelivref{deliv-id})
  \end{compactitem}
\end{task}
