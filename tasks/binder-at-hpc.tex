% template for a task
% each task should be added to exactly one workpackage
% in the workpackage task list
\begin{task}[
  title=Binder at HPC facilities,
  % task id for references
  id=binder-at-hpc,
  % lead institution ID
  lead=MP,
  PM=17,
  wphases={0-36},
  % partner institution ID(s)
  % don't include lead here
  partners={IFR,UIO}
]
In this task, we want to broaden the applicability of the Binder tools to become
useful in High Performance Computing (HPC) environments. In particular, this
requires parallel execution using the reproduced computational environments.

\paragraph*{Context:}
Reproducibility of data created with HPC resources is difficult. In addition to
a specification and access to the actual (simulation) software and dependencies,
one may also need specialised HPC hardware to be able to execute the software.

The notebook interface -- which works well for many examples in computational and
data science -- may not be appropriate for HPC applications, where jobs of
substantial run-time are typically submitted to a queueing system, which will
trigger execution of the (parallel) job, once the required resources have become
available.

It is outside the scope of this proposal to find and implement a generic
reproducibility approach for HPC use cases. Nevertheless, we propose to explore
some aspects of HPC-reproducibility already to influence the development of Binder,
and start the - probably iterative - process of finding a good solution.

\paragraph*{Strategy:}
We focus on reproducible execution of an HPC application to compute data as the step of
novelty here. This may well be without using notebooks, but consist of the
building of the software environment, and submission of a job making use of this
environment to the HPC system's queue.

We assume for this to work that the user (who wants to reproduce some results)
needs to logon to the HPC resource of their choice, and uses repo2docker to
create a suitable software container, and starts execution of those containers
'manually', for example through submission of a compute job to the Slurm
queueing system. We also assume that the hardware required for this reproduction
is available on the HPC system.

\paragraph*{Activity}: We will use repo2docker to automate creation of
reproducible software environment on an HPC system. If no parallelism is
required, this is similar to the Binder@home scenario
(\localtaskref{binder-at-home}). Shared memory parallelisation, using for
example OpenMP, should work well with approach.

A main task here will be to explore the feasibility of using distributed
parallelisation (for example through MPI) where for the execution of an
MPI-program the processes on the nodes run in containers but communicate via MPI
as usual. We evaluate the situation here with software that allows
MPI-parallelisation (such as Octopus). Subsequently, we will share our experience with
reproducible software environments in HPC contexts. \TODO{Add link to report
  deliverable}.

\paragraph*{Challenges}: We expect that we need to use a container technology
that is widely accepted at HPC sites (such as for example Singularity): Docker
on HPC systems is difficult due to the root-user requirements~\cite{Gerhardt_2017}.

Another challenge is that of using accelerators such as GPU cards, which are
installed on the host, but need to be instructed from the software running
inside the container.

For HPC systems, there are sometimes specialised drivers for high-performance
network cards: can these be used from the container environment, and if not what
is the performance impact?  \cite{Liu2021}

Should hardware requirements be archived in the reproducible repository? If so,
what specification should we use?

The existing buildpacks that repo2docker supports
\ref{repo2docker-supported-software-specifications} may need to be extended for
HPC specific software provision (such as Spack or Easybuild).

% The re-execution of parallel software may create significant costs, and thus
% authentication may be necessary.

%   The task includes the following activities
%   \begin{compactitem}
%   \item ...
%      % deliverable will be defined in the appropriate WorkPackage.tex
%     % (\localdelivref{deliv-id})
%   \end{compactitem}
\end{task}





%%% Local Variables:
%%% mode: latex
%%% TeX-master: "../proposal"
%%% End:
