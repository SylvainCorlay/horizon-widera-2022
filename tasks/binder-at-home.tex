% template for a task
% each task should be added to exactly one workpackage
% in the workpackage task list
\begin{task}[
  title=Binder@Home,
  % task id for references
  id=binder-at-home,
  % lead institution ID
  lead=SRL,
  PM=7,
  %wphases={0-36},
  % partner institution ID(s)
  % don't include lead here
  partners={MP,UIO,IFR}
]
In this task we target the compute power of the desktops of individual researchers and
users to recreate software environments in which computational results can be
reproduced and re-used: We envision to provide an experience identical or similar
to that of using BinderHub, but only employing the local computer available to the user anyway.

\paragraph*{Context:} Reproducibility services using Binder currently rely on hosted Binder instances.
The BinderHub Federation provides such a service global service at mybinder.org
which is free at
the point of use. It delegates the building of the software environment and
re-execution of the code to a small number of computer centres that have
volunteered to contribute compute resources.
It may hence be possible to overload the system \TODO{Add link to general discussion
  that mentions how many thousand builds take place per months etc}.
To allow the up-scaling of good reproducibility practice, it is 
desirable not to depend exclusively on such a single hosted service (or even multiple similar hosted services).

Moreover, the compute resources typically offered through mybinder are modest, namely at most 2 GB of RAM
and a single CPU core.
In contrast most laptops and Desktops have similar or better hardware capabilities than
the free mybinder cloud-computing resources currently offer, making it highly desirable to leverage these local resources.
% not sure how/if the following two sentences fit here, maybe better elsewhere
For some research areas, it is essential to access big datasets to reproduce the scientific computing workflow.
For such cases, it is crucial to reproduce the work using infrastructure that has optimised access to the data,
so that one does not consume unnecessary computing and network resources.  

\paragraph*{Task activity:} Based on the preparations in \WPref{reproducibility} and
\WPref{impact}, we intend to extend Binder such that users of the service can
carry out the building of the environment, and -- if desired -- the launching of a
notebook server \emph{on their own hardware}, e.g. their laptop or on-premise or cloud infrastructure.
The working title for such
functionality is ``Binder@Home'' (as a reference to the crowd-based SETI@home search for
extraterrestrial intelligence at home.\footnote{https://setiathome.berkeley.edu})

We will design, implement, and test the 'Binder@Home' functionality, and make it
available as part of the Binder software. That new 'Binder@Home' software component will
essentially complement BinderHub with an easy-to-use local single-user case, it will trigger
the local build of the software environment, the start of the Jupyter notebook
server, and the opening of the relevant local URL and port in a browser.
This effort will bridge the gap from mybinder.org to ``Home'';
by giving the freedom and digital sovereignty
to the researchers to chose where they execute their computational experiments in a simple manner,
without sacrificing the convenience of the mybinder.org service.

%  The task includes the following activities
%  \begin{compactitem}
%  \item ...
%     % deliverable will be defined in the appropriate WorkPackage.tex
%    % (\localdelivref{deliv-id})
%  \end{compactitem}
\end{task}
