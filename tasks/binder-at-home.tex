% template for a task
% each task should be added to exactly one workpackage
% in the workpackage task list
\begin{task}[
  title=Binder@Home,
  % task id for references
  id=binder-at-home,
  % lead institution ID
  lead=SRL,
  PM=7,
  %wphases={0-36},
  % partner institution ID(s)
  % don't include lead here
  partners={MP,UIO}
]
In this task, we want to use the compute power of the Desktops of individual researchers and
users, to recreate software environments in which computational results can be
reproduced, and re-used: We want to provide an experience identical or similar
to that of using BinderHub, but using only the local computer.

\paragraph*{Context:} Reproducibility services using Binder currently rely on hosted Binder instances.
The BinderHub Federation provides such a service global service at mybinder.org
which is free at
the point of use, and delegates the building of the software environment and
re-execution of the code to a small number of computer centres that have
volunteered to contribute compute resources.

It thus possible to overload the system \TODO{Add link to general discussion
  that mentions how many thousand builds take place per months etc}.

To allow the up-scaling of good reproducibility practice, it would be
desirable not to depend on such a single (or even multiple hosted services).

The compute resources typically offered through mybinder are modest: at most 2 GB of RAM
and a single CPU core.
Most laptops and Desktops have similar or much better hardware capabilities than
the mybinder cloud-computing resources currently offer.

For some research areas, it is essential to access big data sets to reproduce the scientific computing workflow.  For such cases, it is essential to reproduce the work using infrastructure that has optimised access to the data, so that one does not consume unnessesary computing and network resources.  

\paragraph*{Task activity:} Based on the preparations in \WPref{reproducibility} and
\WPref{impact}, extend Binder so that users of the service can
carry out the building of the environment, and -- if desired -- launching of a
notebook server \emph{on their own hardware}, such as their laptop or on-premise or cloud infrastructures.   The working title for such
functionality is ``binder@home'' (as a reference to the crowd-based SETI@home search for
extraterrestrial intelligence at home.\footnote{https://setiathome.berkeley.edu}

We will design, implement, and test the 'binder@home' functionality, and make it
available as part of binder software. We will need a utility that takes on
the responsibility of BinderHub for a single-user use case, and which triggers
the local build of the software environment, start of Jupyter notebook
server and opening of the relevant local URL and port an a browser.
This effort will bridge the gap from mybinder.org to ``Home'';
by giving the freedom and digital sovereignty
to the researchers to chose where they execute their computational experiments in a simple manner,
without sacrificing the convenience of the mybinder.org service.


%  The task includes the following activities
%  \begin{compactitem}
%  \item ...
%     % deliverable will be defined in the appropriate WorkPackage.tex
%    % (\localdelivref{deliv-id})
%  \end{compactitem}
\end{task}
