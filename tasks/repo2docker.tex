\begin{task}[
  title=repo2docker development,
  id=repo2docker-timemachine,
  lead=SRL,
  PM=12,
  wphases={0-24!0.5},
  partners={QS}
]

This tasks improves the robustness of \repotodocker{}. We illustrate this with one specific example: 
Often, a repository of scientific results may
specify which software library is required (such as the Python library
\softwarename{pandas}), but not which version.

A software environment creation tool -- such as \repotodocker{} -- can then
attempt to install the most recent version of \softwarename{pandas}. This is
usually the intention of the authors, and was correct at the time the repository
was created. However, as time moves on, the interface, behaviour and dependence
on other packages of \softwarename{pandas} will change, and at some point an
automatic build of the software for the whole repository may fail because of
conflicting dependencies.

We have found through anecdotal evidence that these problems can be
overcome if a \softwarename{pandas} version can be chosen that was the
most recent at the time when the repository was created. A related
issue is that the Python version itself (such as 3.8, 3.9 or 3.10) may
not be specified at all.

We will teach \repotodocker{} to establish the date of publication
(or last modification) of the repository, to determine the appropriate version
of software libraries from that time, and to select libraries with those
versions if no specific version is specified.

In the context of Python packages, we can use the
\softwarename{pypi-timemachine}
package.\footnote{\url{https://github.com/astrofrog/pypi-timemachine}}.

\TODO{@QS, @min - Comment on other software sources - conda, ... ?}
\end{task}
