% template for a task
% each task should be added to exactly one workpackage
% in the workpackage task list
\begin{task}[
  title=Best practice guidelines for reproducible science,
  id=online-resources,
  lead=UIO,
  PM=13,
  %wphases={6-36!0.36},
  partners={SRL,MP,IFR}
]
The aims of this task are to (i) provide online resources for researchers on more reproducible
and re-usable science, and (ii) support delivery of our workshops
(\taskref{education}{workshops}).

This task includes the following activities:
  \begin{compactitem}
  \item Collect and compose best practice guidelines for reproducible and
    re-usable science (\delivref{education}{best-practice-guide}).
  \item Split the content into multiple topic areas and target audiences so learners
    with different prior knowledge and needs can be directed to the most relevant content.
  \item Develop lesson materials on \emph{open science} best practices (version
    control, testing, automation of all steps, collaboration and peer review,
    documentation, software licensing and open source, use of Jupyter
    notebooks).
  \item Develop lesson materials on \emph{reproducible computational science},
    which focuses on combining the open science tools for reproducible science.
  \item Develop materials on \emph{using Binder tools to make science more
      reproducible and re-usable}. This includes addressing and describing the
    use cases from \WPref{applications}.
  \item Collaboration with the \href{https://coderefinery.org}{CodeRefinery}
    project and \href{https://carpentries.org/}{The Carpentries} (Carpentries incubator and Carpentries Lab)
    for the development and maintainance of the online lesson materials and delivery of workshops.
  \item The training material will also be referenced on the Binder tools webpage.
  \end{compactitem}
  All material will be licensed under an open license such as
  \href{https://creativecommons.org/licenses/by/4.0/}{CC 4.0}
  (\delivref{education}{best-practice-guide}).
\end{task}
