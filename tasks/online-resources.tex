% template for a task
% each task should be added to exactly one workpackage
% in the workpackage task list
\begin{task}[
  title=Online resources for open science,
  id=online-resources,
  lead=UIO,
  PM=14,
  wphases={0-36!.3},
  partners={SRL,MP,UIO}
]
  The aim of this task is to provide communities with online resources for Open Science and support \taskref{education}{workshops}.
  This task includes the following activities:
  \begin{compactitem}
  \item Collaboration with the \href{https://coderefinery.org}{CodeRefinery} project for the development and maintainance of the \href{https://coderefinery.org/lessons/}{online lesson materials} on open science best practices (JupyterLab, version control, collaboration and peer review, documentation, testing, software licensing, and reproducible research).
    Following CodeRefinery's tradition, the aim will be to contribute the
    lessons to \href{https://software-carpentry.org/}{Software Carpentry}
    and \href{https://data-carpentry.org/}{Data Carpentry}.
\item Development of an interactive book on applied stochastic
  processes in Physics (\localdelivref{sde-book}). Unlike classical
  books in this subject, it will be supplemented by interactive
  numerical examples solving real life problems. This will be the
  occasion to emphasize the role of high performance computing in
  solving problems modeled by stochastic differential equations (SDE).
  The book will be available via the EOSC hub, and showcase the use of
  HPC hardware (GPU).
  \item Production of a video tutorial for the astronomy application on the 
      CDS YouTube channel (\url{https://www.youtube.com/channel/UCUESQl7rNupLlV_VcceE0Ng}).
  \end{compactitem}
  All material will be licensed under an open license such as
  \href{https://creativecommons.org/licenses/by-sa/4.0/}{CC BY-SA 4.0}.
\end{task}
