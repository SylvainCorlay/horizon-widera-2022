% template for a task
% each task should be added to exactly one workpackage
% in the workpackage task list
\begin{task}[
  title=Online resources for reproducible science,
  id=online-resources,
  lead=UIO,
  PM=14,
  wphases={0-36!.3},
  partners={SRL,MP,UIO}
]
  The aims of this task are to (i) provide online resources for Open Science and
  (ii) support \taskref{education}{workshops}.
  
  This task includes the following activities:
  \begin{compactitem}
  \item Collaboration with the \href{https://coderefinery.org}{CodeRefinery}
    project for the development and maintainance of the
    \href{https://coderefinery.org/lessons/}{online lesson materials}.
  \item Develop lesson materials on \emph{open science} best practices (version control,
    testing, automation of all steps, collaboration and peer review,
    documentation, software licensing and open source, use of Jupyter notebooks).
  \item Develop lesson materials on \emph{reproducible computational science},
    which focuses on combining the open science tools for reproducible science.
  \item Develop materials on \emph{using Binder tools to make science more
      reproducible and re-usable}. This includes addressing and describing the
    use cases from \WPref{applications}.
  \item Following CodeRefinery's tradition, the aim will be
    to contribute the lessons to \href{https://software-carpentry.org/}{Software
      Carpentry} and \href{https://data-carpentry.org/}{Data Carpentry}.
  \item The training material will also be referenced on the Binder tools webpage.
  \end{compactitem}
  All material will be licensed under an open license such as
  \href{https://creativecommons.org/licenses/by-sa/4.0/}{CC BY-SA 4.0}
  (\delivref{education}{education-materials1}, \delivref{education}{education-materials1}).
\end{task}
