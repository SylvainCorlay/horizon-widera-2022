% template for a task
% each task should be added to exactly one workpackage
% in the workpackage task list
\begin{task}[
  title=Towards quantifiable progress for reproducible software environments,
  % task id for references
  id=repo2docker-checker,
  % lead institution ID
  lead=SRL,
  PM=10,
  % wphases={0-24!0.42},
  % partner institution ID(s)
  % don't include lead here
  partners={MP}
  ]
  The \repotodocker{} tool is a key component of the Binder software for
  reproducibility (see \ref{binder-how-does-it-work}). It can be used to create
  a software environment based on software dependency specification standards
  (see \ref{sec:repo2docker}) that are widely used.

  If the required software is specified -- for example through a
  \texttt{requirements.txt} file for Python dependencies -- then \repotodocker{}
  can create the software environment (currently limited to such environments in
  Docker images), within in which the main computation or data analysis can be
  reproduced.

  In this task, we will develop a tool -- with working name
  \softwarename{repo2docker-checker} -- that allows us to \emph{automatically}
  assess the reproducibility of software environments for software that is
  publicly available on GitHub, Bitbucket or GitLab repositories.

  For every repository, the \softwarename{repo2docker-checker} tool will report if an
  appropriate software could be compiled, or if a problem occurred. Software
  environments in repositories may be reproducible because the authors already
  use Binder to offer their repository in an interactive Binder environment. Or
  the software environment may be reproducible because the authors have followed
  standard conventions
  and \repotodocker{} understands these conventions.

The task includes the following activities:
\begin{compactitem}
  \item Through manual inspection of selected repositories, identify common
    failure modes of building of the software environment (such as for example
    not specifying the Python version to use).
  \item Design and develop the \softwarename{repo2docker-checker}. A prototype
    exists.\footnote{https://github.com/minrk/repo2docker-checker}
  \item Where possible, identify for what reason the software build has failed.
  \item Develop a strategy and heuristic to evaluate success of the build
    process.
  \item Identify suitable software repositories for the study.
  \item Automate the software reproduction process for the available
    repositories.
  \item Automate the analysis of the results, so the study can be repeated later.
  \item Carry out the study to estimate the fraction of reproducible repositories.
  \item Repeat the study after the robustness of \repotodocker{} has
    been improved (\localtaskref{repo2docker-timemachine} to evaluate
    progress. \TODO{Point to KPI}
  \end{compactitem}

  The tool will be made available as open source
  (\localdelivref{deliv-id-repo2docker-checker-software}).

  % Some of the findings
  % here will contribute to the \TODO{deliverable XXX in WP5 - best practice for
  %   reproducible repositories with Binder.} At the end of the project, we will
  % provide a summary of improvements in reproducibility we have achieved through
  % changes in the Binder tools.
\end{task}
