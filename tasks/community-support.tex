\begin{task}[
  title=Community support and engagement,
  id=community-support,
  lead=SRL,
  PM=13,
  %wphases={12-36!.5},
  partners={MP,QS,UIO,IFR}
]
A project such as \TheProject{} has the ambition to develop a small set of tools
that will \emph{impact many researchers} and have the potential to be useful
\emph{across all scientific domains that need electronic data processing} as part of their
scientific research and publication process.

As such, we expect that the demand through support queries, documentation
clarification questions, and helpful feedback will be substantial. With this
task, we explicitly reserve some time for such activities.

We have an opportunity here to address multiple aims simultaneously.

This task complements the Community Engagement Panel and has several aims: 
\begin{compactitem}
\item to engage with community members (and potentially their computing support
  staff) to help them make best use of the Binder tools. This can range from
  helping to configure a BinderHub installation, to address usage questions of
  tools such as \repotodocker{} in domain-specific contexts;
\item to engage with community members to better understand diverse
  requirements, and use this information to make the Binder tools and
  reproducibility guidelines more useful for a wider diversity of scientific
  domains;
\item to engage with community members to train researchers and research
  software engineers in reproducibility practices and tools (to address a
  shortage of staff with such skills)
\item to engage with community members to invite them to contribute to the
  binder tools, the reproducibility guidelines and policy development, and other
  open source tools.
\end{compactitem}

We will achieve those aims through listening to feedback, queries and requests
for help from the community, and reserve time to respond. Depending on the
complexity of an issue, guidance by email, chat, video meeting or even an
in-person visit may be appropriate. (When demand exceeds the time budget, we
will need to prioritise which issues we can deal with first.)

We know from our experience with running and contributing to open source
projects that such engagement activities are effective in training interested
and often highly skilled scientists and research software engineers to become
contributors to open source projects. While they may have a primary interest in
improving an open source tool to suit their needs, this will likely benefit
others as well. Once somebody has contributed to a particular open source
software tool, they are more likely to make follow-up contributions - for
example to improve documentation.

\end{task}
