% ---------------------------------------------------------------------------
%  Section 2: Impact
% ---------------------------------------------------------------------------


\section{Impact}
\label{sec:impact}

\subsection{Pathway toward Impact}

\eucommentary{
e.g. 4 pages
Impact: Logical steps towards the achievement of the expected impacts of the project over time, in particular beyond the duration of a project. A pathway begins with the projects' results, to their dissemination, exploitation and communication, contributing to the expected outcomes in the work programme topic, and ultimately to the wider scientific, economic and societal impacts of the work programme destination.
}

\eucommentary{
Provide a narrative explaining how the project’s results are expected to make a difference in terms of impact, beyond the immediate scope and duration of the project. The narrative should include the components below, tailored to your project.
(a)	Describe the unique contribution your project results would make towards (1) the outcomes specified in this topic, and (2) the wider impacts, in the longer term, specified in the respective destinations in the work programme.
	Be specific, referring to the effects of your project, and not R\&I in general in this field.
	State the target groups that would benefit. Even if target groups are mentioned in general terms in the work programme, you should be specific here, breaking target groups into particular interest groups or segments of society relevant to this project.
	The outcomes and impacts of your project may:
•	Scientific, e.g. contributing to specific scientific advances, across and within disciplines, creating new knowledge, reinforcing scientific equipment and instruments,  computing systems (i.e. research infrastructures);
•	Economic/technological, e.g. bringing new products, services, business processes to the market, increasing efficiency, decreasing costs, increasing profits, contributing to standards’ setting,  etc.
•	Societal , e.g. decreasing CO2 emissions, decreasing avoidable mortality, improving policies and decision making, raising consumer awareness.
Only include such outcomes and impacts where your project would make a significant and direct contribution. Avoid describing very tenuous links to wider impacts. However, include any potential negative environmental outcome or impact of the project including when expected results are brought at scale (such as at commercial level). Where relevant, explain how the potential harm can be managed.
(b)	Describe any requirements and potential barriers - arising from factors beyond the scope and duration of the project - that may determine whether the desired outcomes and impacts are achieved. These may include, for example, other R\&I work within and beyond Horizon Europe; regulatory environment; targeted markets; user behaviour. Indicate if these factors might evolve over time. Describe any mitigating measures you propose, within or beyond your project, that could be needed should your assumptions prove to be wrong, or to address identified barriers.
	Note that this does not include the critical risks inherent to the management of the project itself , which should be described below under ‘Implementation’.
(c)	 Give an indication of the scale and significance of the project’s contribution to the expected outcomes and impacts, should the project be successful.  Provide quantified estimates where possible and meaningful.
	‘Scale’ refers to how widespread the outcomes and impacts are likely to be. For example, in terms of the size of the target group, or the proportion of that group, that should benefit over time; ‘Significance’ refers to the importance, or value, of those benefits. For example, number of additional healthy life years; efficiency savings in energy supply.
	Explain your baselines, benchmarks and assumptions used for those estimates. Wherever possible, quantify your estimation of the effects that you expect from your project. Explain assumptions that you make, referring for example to any relevant studies or statistics. Where appropriate, try to use only one methodology for calculating your estimates: not different methodologies for each partner, region or country (the extrapolation should preferably be prepared by one partner).
	Your estimate must relate to this project only - the effect of other initiatives should not be taken into account.
}


The expected impact of \TheProject with respect to the
work program is detailed in the table below.

\begin{center}
\begin{tabular}{|m{.3\textwidth}|m{.7\textwidth}|}\hline
  Expected impact & \\\hline


  Integrating co-design into research and
  development of tools to better support scientific, industrial, and
  societal applications benefiting from a strong user orientation &
  The Jupyter and Binder tools have always been driven by a close connection to users; since
  the project began as IPython in 2001, many of the developers have been
  scientific researchers using the tools as they developed them. More recently,
  when Jupyter has benefited from dedicated developer time, developers have
  remained in academic institutions, in the role now referred to as
  'research software engineers,' allowing day-to-day interactions with
  researchers using these tools in a wide range of fields.

  By supporting developers in various research institutions where the improvements
  will be used as they are developed, \TheProject will continue this invaluable
  collaboration.

  The impact of this approach for enabling reproducible computational science is expected
  to be very high because it is a direct response to the strong demand from
  scientists for improving the productivity and reproducibility of their work.
  The notebook approach central to the Jupyter community is being embraced in many scientific disciplines, so the
  proposed services to be developed in \TheProject are strongly oriented to the user needs.
  Further, the environment-reproducibility focus of the Binder project is useful well beyond
  the notebook focus of the Jupyter community.

  \\\hline

  Supporting the objectives of Open and Reproducible Science by
  improving access to content and resources, and facilitating interdisciplinary
  collaborations &
  Binder and repo2docker have seen rapid uptake in many kinds of research,
  because they bring together the essential elements of the modern scientific
  computational workflow (from data collection to publication and open sharing)
  in the familiar format of a scientific notebook, with powerful functionality
  for access to scientific content for analysis and visualisation.
  Notebooks also embody the core concepts of open science by providing a
  mechanism to reproduce results in publications, and collaborative
  sharing of not just scientific results, but of the code that produced them.


  We expect the use of notebooks in EOSC to improve access to scientific code:
  digital documents and notebooks encourage publishing workflows, whereas code in
  scripts or manual interactive workflows are often kept by the researchers who
  performed them. The focus on clarity and reproducibility also helps to ensure
  that data is meaningfully accessible, by preserving essential understanding to
  make sense of the raw data.

  We have already seen a good example of the Jupyter ecosystem facilitating an
  interdisciplinary collaboration: the LIGO scientific collaboration shared
  notebooks detailing the data processing steps which led to the discovery of
  gravitational waves, using the Binder service to allow anyone to re-compute
  the published plots. Scientists with no background in gravitational waves
  studied these notebooks and improved the signal processing.
  In this proposal, we want to provide this ability to a wider audience through
  improved tools and documentation,
  including for disciplines which rely on processing much larger volumes of
  data \cite{ligo-open-science}.

    By connecting new notebook capabilities to existing and highly used services,
  we expect to have impacts for the users and also for the service provider.
  The scientific users will have access to new capabilities,
  and we anticipate adoption of new innovative ways of using the data.
  We also expect an impact on the services themselves, in terms of usage,
  but also in terms of capturing precious information and feedback on
  how to evolve these services to best support open and collaborative use of
  the data and services.

  \\\hline

  Facilitating Open and Reproducible Science by automating existing practices &
  A key to our philosophy and success thus far has been automating
  what scientists already are (or should be) doing.
  The approaches and environment specifications used in Binder are
  not specific to Binder, and are already widely adopted.
  We only seek to automate this process,
  and implement and document as many standards as we can find in use by the community.
  By implementing what is already in use,
  we minimise "lock-in" and meet users,
  lowering the barrier to adoption relative to "bespoke" tools,
  which require a large change in tooling, and significant disruption to researchers'
  work.




  % \\\hline
  \TOWRITE{ALL}{More impacts for other applications/demonstrators ...}

  \\\hline


\end{tabular}
\end{center}


\subsubsection{Measuring impact}

As we are building tools and services for Open and Reproducible Science,
the best measure of our impact is in the adoption and use of these tools and services,
which can be observed qualitatively (positive anecdotal feedback) and quantitatively
(counting visitors to a service, for example).
Much of our work will be in the form of contributions to existing public projects,
such as Jupyter and Binder,
which can be measured in our participation in those projects,
such as code and documentation contributions,
bug reports, and roadmap contributions.

We can measure our progress toward aims and objectives in \ref{sect:objectives}
via the following
Key Performance Indicators (KPIs):

\begin{compactenum}[\textbf{KPI} 1:]
  \item \label{kpi:workshop-attendees}
    Attendees at Reproducible Science workshops organised by \TheProject participants.
  \item \label{kpi:binder-publications}
    Open publications for which the authors have made a reproducible repository available
    through \TheProject services.
    \TODO{HF: Services or tools? We are thinking here of the number of (git) repos
      that reproduce results from papers, or for which we can build the
      environment sucessfully with repo2docker?}
    \TODO{HF: Can/should we count existing repositories that may become reproducible
    (in this sense) through the time-machine (and other) improvements to repo2docker?}
  \item \label{kpi:binder-visits}
    Visitors to \TheProject services, engaging with open, interactive
    communications.
    \TODO{HF: Are these users of mybinder.org? Doesn't hit the projectly as we don't
    run the mybinder.org service. Maybe remove the KPI?}
  \item \label{kpi:dissemination}
    Publications and presentations by \TheProject documenting the use of \TheProject services for
    Open Science. \TODO{HF: So this includes papers, presentations, but also
      documentation and best practice guides?}
  \item \label{kpi:contributions}
    Contributions by \TheProject and the wider community to Jupyter software and others,
    including issues reported, bugs fixed, features added, and roadmaps developed.
\end{compactenum}

\subsubsection{Barriers, Obstacles and Framework conditions}

The \TheProject project will certainly face a number of challenges as it undertakes
the ambitious program of work described by this proposal.
We can identify a number of potential barriers and obstacles but overall
these are assessed to be minor and planning is in place to mitigate the
identified risks.

While a number of the partners have worked closely together in previous projects,
the integration of new partners from different disciplines will require
dedicated efforts for communication within the project.

A detailed assessment of risks and mitigations can be found in \ref{sec:risks}.

In addition, \TheProject will produce highly visible demonstrations of
tools for open and reproducible science in targeted scientific disciplines.
The result will be innovative new prototype services that will provide
direct benefits to the early adopters in various research fields.
Moreover it will serve as a demonstration of a strategy for open and reproducible science
that will be applicable across many domains.

\subsection{Measures to maximise impact - Dissemination, exploitation and communication}

\eucommentary{
e.g. 5 pages, including 2.3
}
\eucommentary{
Describe the planned measures to maximise the impact of your project by providing a first version of your ‘plan for the dissemination and exploitation including communication activities’. Describe the dissemination, exploitation and communication measures that are planned, and the target group(s) addressed (e.g. scientific community, end users, financial actors, public at large).
	Please remember that this plan is an admissibility condition, unless the work programme topic explicitly states otherwise. In case your proposal is selected for funding, a more detailed ‘plan for dissemination and exploitation including communication activities’ will need to be provided as a mandatory project deliverable within 6 months after signature date. This plan shall be periodically updated in alignment with the project’s progress.
	Communication measures should promote the project throughout the full lifespan of the project. The aim is to inform and reach out to society and show the activities performed, and the use and the benefits the project will have for citizens. Activities must be strategically planned, with clear objectives, start at the outset and continue through the lifetime of the project. The description of the communication activities needs to state the main messages as well as the tools and channels that will be used to reach out to each of the chosen target groups.
	All measures should be proportionate to the scale of the project, and should contain concrete actions to be implemented both during and after the end of the project, e.g. standardisation activities. Your plan should give due consideration to the possible follow-up of your project, once it is finished. In the justification, explain why each measure chosen is best suited to reach the target group addressed. Where relevant, and for innovation actions, in particular, describe the measures for a plausible path to commercialise the innovations.
	If exploitation is expected primarily in non-associated third countries, justify by explaining how that exploitation is still in the Union’s interest.
	Describe possible feedback to policy measures generated by the project that will contribute to designing, monitoring, reviewing and rectifying (if necessary) existing policy and programmatic measures or shaping and supporting the implementation of new policy initiatives and decisions.
•	Outline your strategy for the management of intellectual property, foreseen protection measures, such as patents, design rights, copyright, trade secrets, etc., and how these would be used to support exploitation.
	If your project is selected, you will need an appropriate consortium agreement to manage (amongst other things) the ownership and access to key knowledge (IPR, research data etc.). Where relevant, these will allow you, collectively and individually, to pursue market opportunities arising from the project.
	If your project is selected, you must indicate the owner(s) of the results (results ownership list) in the final periodic report.
}

\TheProject is contributing to tools for Open and Reproducible Science.
Tools only have impact if and when they are used,
so it is important that we disseminate our work
in order to reach and support user communities for our software and services. This section
outlines how the project will establish and organise the dissemination and communication
actions to promote the project and the adoption of its outcomes beyond the project's lifetime.

The dissemination and communication plan is outlined in the following sub-sections.
Therein we distinguish:
\begin{itemize}
\item Dissemination as the public disclosure of the results of the project through
a process of promotion and awareness-raising right from the beginning of a project.
It makes research results known to various stakeholder groups (like research peers, industry
and other commercial actors, professional organisations, policymakers) in a targeted
way, to enable them to use the results in their own work.
\item Communication as the strategic and targeted measures for promoting the project
and its results to a multitude of audiences, including the media and the public, and possibly
engaging in a two-way exchange. The aim is to reach out to society as a whole and
in particular to some specific audiences while demonstrating how EU funding contributes to tackling
societal challenges.
\end{itemize}

\subsubsection{Dissemination and exploitation of results}

\WPref{education} is focused on dissemination of \TheProject work.
Our goal is to facilitate Open and Reproducible Science through the development and use of open and freely available tools.
All \TheProject software will be made publicly and freely available under Open Source licenses, and
hosted on public code hosting sites such as GitHub.
Most \TheProject work will be in the form of
contributions to existing projects,
which will be governed by the licenses of those projects.
All Jupyter and Binder software is released under the permissive BSD license,
which specifically allows commercial exploitation,
as has proven successful in enabling collaborations with industrial partners
such as Google, Microsoft, IBM, and more.
This means that all \TheProject software will be available and accessible to all who find it,
at no cost to \TheProject,
enabling long-term access beyond the funding of \TheProject.
Similarly, non-code products such as dissemination works
(workshop materials, etc.)
will be made freely available under open Creative Commons licenses.

As a result, the primary dissemination effort is to:

\begin{enumerate}
  \item make sure that prospective users are \textbf{aware of the work}, and
  \item enable them to use the tools through \textbf{learning resources, training, and services}.
\end{enumerate}

Our focus for dissemination will be on
% \taskref{education}{workshops},
operating workshops, training various communities in the availability,
purpose, development, and use of \TheProject software and services.
We will make a particular effort to use these workshops as an opportunity
to \textbf{support diversity and inclusion in the Open Science community},
by running workshops for under-served and under-represented groups in the academic and
open source communities.
Additionally, for long-term resources available to the wider community
who will not be able to attend workshops,
we will produce \textbf{free, online materials for training} in the use of \TheProject
software and services.
These resources will be hosted on free, public hosting services,
such as GitHub Pages,
enabling long-term access to the work of \TheProject,
even after the end of funding.

The operation of prototype services in \WPref{applications} is also a dissemination activity,
as services like Binder not only enable Open and Reproducible Science by facilitating interactive publications,
they also enable \textbf{interactive demonstration of tools and functionality}
developed in \TheProject.

\TheProject, in collaboration with the operators of mybinder.org,
will explore sustainability plans for covering long-term costs of operating such services,
including institutional subscription models, donations, and others.

\medskip
\noindent \textbf{Data management plan}\label{sec:data-management-plan}\\
Except for the usage data described below,
\TheProject activities will not generate or collect data.
While we have many demonstrators that interact with data, they do not generate or collect that
data themselves, but rather provide analytical mechanisms or access to data governed by
existing data management plans and data policies of project partners at each site,
as well as publicly accessible open data.


\noindent \textbf{Service usage data} \\
Any data collected through the operation of public services
(e.g. popularity data for public open science repositories)
will be fully anonymised to the satisfaction of relevant best privacy practices and regulations, such as GDPR,
and made publicly available in the standard JSON Lines format,
as is done already for mybinder.org \cite{mybinder-archive}.
This is very small data and easily archived on free hosting services such as GitHub,
and will be made available under the Creative-Commons Universal Public Domain Dedication (CC0).
There is no cost to the project associated with archiving this data long-term.

\subsubsection{Communication activities}

The main remaining goal for dissemination is making sure that potential users
are aware of the tools and services developed by \TheProject.

In order to maximise this impact, it is vital to address the audience as one project
and ensure the immediate recognition of information stemming from it.
Together with all partners involved, \TheProject will therefore build \textbf{a strong project identity}.
The following design and communication elements will be used to strengthen the project
uniformity and identity and to deliver clear messages to our audience: \TheProject naming, logo,
presentations template, templates for reports and letters, project posters/leaflets etc.

In addition to \TheProject-organised workshops in
\taskref{education}{workshops},
the primary mechanism by which we will communicate our results is through publications and conferences.
All publications funded by \TheProject will be \textbf{Open Access},
and sites expecting publications have budgeted funds for paying Open Access fees.
We will identify and attend appropriate conferences for sharing our work,
including running tutorials at conferences in historically interested communities such as PyData and SciPy.
Also, we will identify and attend conferences from complementary communities such as ROpenSci,
Mozilla Science, and Julia
as well as domain specific conferences to maximise the impact of \TheProject and to broaden its
audience outside the
traditionally included communities.

We will operate a \textbf{website}
(\taskref{education}{website})
for collecting and sharing information about \TheProject and its progress.
It will provide a centralised way to access the various publicly available deliverables, publications
and articles related
to the project. The site will be regularly updated over the lifetime of the project
with the project publications and public materials, such as flyers, posters and
public deliverables, organized workshops, available services, news, etc.
Site analytics will be associated with the project website, in order
to provide useful insight on how to improve its impact. In addition, the project intends to
develop its presence on \textbf{the social and content
networks}, such as Twitter. The channels will be used for interaction
with the professional community as well as the general public
(differentiation on the content per channel based on the target group wishing to address).
As part of the project?s communication plan, \TheProject will develop a social media strategy
in order to increase outreach and social impact, which can be summarised as follows: (a) identifying target
audience and key stakeholders, (b) updating social media content and sparking
discussion in social media/tweeting, (c) measuring social impact and reassessing
social media strategy as required.

\clearpage
\subsection{Summary}

\eucommentary{
Provide a summary of this section by presenting in the canvas below the
key elements of your project impact pathway and of the measures to
maximise its impact.
}

\textbf{KEY ELEMENT OF THE IMPACT SECTION}

\newlength{\savedparindent}
\newcommand{\summarybox}[2]{
\begin{framed}

\centerline{\textbf{ #1}}
\setlength{\savedparindent}{\parindent}
\par
\setlength{\parindent}{0pt}
{#2}
\setlength{\parindent}{\savedparindent}
\end{framed}
}


\begin{multicols}{2}
\summarybox{SPECIFIC NEEDS}{
\eucommentary{What are the specific needs that triggered this project?}

Reproducibility is a widely recognized challenge in computational science.
There are many tools that solve \emph{part} of the problem,
or aim to solve the whole problem while requiring wholesale adoption of a specific tool,
which may not be practical or desirable across a variety of domains or communities.
Reliably reproducing software environments without requiring adoption of any single tool chain
allows for modular adoption, integrating into policies, practices, etc.
We want to lower the barrier to reproducibility, and thereby increase the adoption of reproducible practices.
}
\summarybox{D \& E \& C MEASURES}{
\eucommentary{What dissemination, exploitation and communication measures will you apply to the results?}

\textbf{Exploitation:} \TOWRITE{}{???}

\textbf{Dissemination:} \TOWRITE{}{Workshops}

\textbf{Communication:} \TOWRITE{}{Documentation \(or do I have DIS/Comm backwards?\)}
}

\summarybox{EXPECTED RESULTS}{
\eucommentary{What do you expect to generate by the end of the project? }

\textbf{Improved software tools for reproducing computational environments}:
by improving the Binder tools for reproducible environments,
it will be easier for researchers to produce and share results openly and reproducibly.

\textbf{Improved understanding of good practices for reproducibility}: study results, documentation, and workshops will aid researchers and policy makers in understanding the most appropriate practices to adopt in their pursuit of reproducible research.
}

\summarybox{TARGET GROUPS}{
\eucommentary{Who will use or further up-take the results of the project? Who will benefit from the results of the project?}

\textbf{Computational science practitioners}: researchers with an interest in reproducibility of their on work.

\textbf{Research institutions}: institutions facilitating or enforcing the reproducibility of their researchers.

\textbf{Policy makers}: Funders and institutions requiring their subjects to follow reproducible practices

\textbf{Publishers}: Requiring/enabling reproducible publications
}

\summarybox{OUTCOMES}{
\eucommentary{What change do you expect to see after successful dissemination and exploitation of project results to the target group(s)?}

\textbf{Adoption of Binder tools for reproducibility}:
practitioners will have access to Binder tools for reproducibility.
Having improved their usability and robustness

\textbf{Facilitating practical policies for reproducible publications}
}
\summarybox{IMPACTS}{
\eucommentary{
What are the expected wider scientific, economic and societal effects of the project contributing to the expected impacts outlined in the respective destination in the work programme?
}

\textbf{Improved reproducibility of scientific results}: Improving the \emph{ease of use} tools for reproducibility
lowers the barrier for adoption, and thereby increases the number, quality, and access of reproducible research outputs.
}
\end{multicols}

% \noindent
% \fbox{
% \begin{minipage}{
% % \dimexpr\linewidth-2\fboxrule-2\fboxsep
% 2in
% }
% \textbf{SPECIFIC NEEDS 2}
%
% xyz
% \begin{compactenum}
% \item x
% \item y
% \end{compactenum}
% \end{minipage}
% }
% {
% \hfill\makebox[0pt]{\fbox{
% \textbf{SPECIFIC NEEDS}}}
% \hfill
% }

% \textbf{SPECIFIC NEEDS}
%
% xyz
% \begin{compactenum}
% \item x
% \item y
% \end{compactenum}
