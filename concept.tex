\TOWRITE{NT/...}{Finalise}
\TOWRITE{ALL}{Proofread concept and approach pass 2}

\subsection{Concept and Methodology}\label{sec:concept_methodology}
\eucommentary{5-8 pages}
\eucommentary{
-- Describe and explain the overall concept underpinning the project.
Describe the main ideas, models or assumptions involved. Identify
any trans-disciplinary considerations;
-- Describe and explain the overall approach and methodology, distinguishing, as
appropriate, activities indicated in the relevant section of the work programme, e.g.
Networking Activities, Service Activities and Joint Research Activities, as detailed in
the Part E of the Specific features for Research Infrastructures of the Horizon 2020
European Research Infrastructures (including e-Infrastructures) Work Programme 2014-
2015;\\
-- Describe how the Networking Activities will foster a culture of co-operation between the
participants and other relevant stakeholders.\\
-- Describe how the Service activities will offer access to state-of-the-art infrastructures,
high quality services, and will enable users to conduct excellent research.\\
-- Describe how the Joint Research Activities will contribute to quantitative and qualitative
improvements of the services provided by the infrastructures.\\
-- As per Part E of the Work Programme, where relevant, describe how the project will
share and use existing basic operations services (e.g. authorisation and accounting
systems, service registry, etc.) with other e-infrastructure providers and justify why such
services should be (re)developed if they already exist in other e-infrastructures. Describe
how the developed services will be discoverable on-line.\\
-- Where relevant, describe how sex and/or gender analysis is taken into account in the
project's content.}


\subsubsection{Concept}\label{sec:concept}

Open Science is the principle that science, in order to be most
\textbf{impactful} and \textbf{socially responsible}, should be done \textbf{publicly}, with as
much of the scientific process and products \textbf{accessible, reviewable,
and reusable} by as many members of the global community as possible.
In the modern age of computational science, almost all academic
fields, from humanities to social sciences to biology and astronomy
are presented with exciting opportunities for Open Science.  As more and
more research takes the form of code and/or data, the opportunity to
share, reproduce, and reuse scientific work is greater than ever, even
enabling new forms of \textbf{interdisciplinary collaboration}.

At the same time as we share in these exciting opportunities, there
are corresponding challenges, technical and social, to making Open
Science a practical reality.  We face big questions: If a researcher
has code and/or data to publicise, how is that best done?  How do
researchers learn \textbf{Open Science best practices} in their field?  How do
previously disconnected fields benefit from each other's work as the
same computational challenges are faced again and again by different
communities?

These are the questions that guide \TheProject.
With so much research being done that wants to be Open,
how can we make Open Science

\begin{enumerate}
    \item as \textbf{easy} as possible to share?
    \item as \textbf{useful} as possible to other researchers and the public?
\end{enumerate}

\noindent Our plan for \textbf{improving access and effectiveness of Open Science} can be summarised as:

\begin{enumerate}
\item improve and maintain \textbf{common software infrastructure} used for
  Open Science,
\item develop the Jupyter ecosystem to improve capabilities to \textbf{better
  serve Open Science},
\item \textbf{guide, validate, and demonstrate} our developments through
  collaboration with a wide variety of application domains,
\item enable students and researchers to perform Open Science through
  \textbf{training and education}, and improving inclusiveness by focusing
  these on under-served and under-represented communities, and
\item operate services to facilitate Open Science collaborations with
  Jupyter software.
\end{enumerate}

\medskip

\subsubsection{Project Jupyter and the surrounding ecosystem}
\label{sec:project-jupyter}

\begin{figure}[htb]\centering
  \includegraphics[width=0.9\textwidth]{use-cases-binder-logbook-solution.png}
  \caption{A typical use case for Jupyter notebooks in research.
            Image by Juliette Belin for the OpenDreamKit project, used under
            CC-BY-SA.}\label{fig:use-cases-binder}
\end{figure}

\noindent\textbf{Jupyter ecosystem as the root of \TheProject}


\TheProject has chosen to centre its efforts on the Jupyter software
ecosystem. Figure~\ref{fig:use-cases-binder} summarises a typical use
case of Jupyter Notebook and Binder; both are described in more detail
below.

The Jupyter notebook and Jupyter ecosystem are of increasing
importance in computational science and data science, in academia,
industry, and services. In addition to supporting high productivity of
researchers, they have great potential to push Open Science forward:
the notebook provides a complete description of a computational and
data science study (Step 1 in figure~\ref{fig:use-cases-binder}), and the notebook can -- in principle -- be turned
into a publication, or can be used to provide the required computation
for a part of a publication, such as a figure
(Step 2 in figure~\ref{fig:use-cases-binder}). Once the researcher has
specified what software is required to execute the notebook (Step 3
in figure~\ref{fig:use-cases-binder}), the study is completely
reproducible by anyone (Step 4 in figure~\ref{fig:use-cases-binder}).

In this way, the notebook enables reproducibility of complex tasks
with hardly any additional effort on the user side.
The Binder project allows to execute such notebooks in
tailored computational environments; an aspect of reproducibility that
is not widely supported yet,
and a great opportunity for improving best practices in Open Science.

Furthermore, for users wanting to connect
to a local Jupyter notebook server on their machine, or to connect to
a server somewhere else on the Internet, the users only need a
web-browser to display and use the notebook regardless of the location
of the notebook server,
allowing computation to run anywhere from a local laptop to a remote supercomputer or in the cloud.
Because of these characteristics,
the Notebook is already planned to become an
important service on the European Open Science Cloud (EOSC) (for
example in \cite{panosc}),
and is an ideal component to use when building Open Science Services.

\medskip\noindent\textbf{Project Jupyter}

\emph{Project Jupyter} \cite{Jupyter}, which has grown increasingly popular in the scientific
computing community, has become the \emph{lingua franca} of interactive
computing in both academia and industry. The main goal of Project Jupyter
is to provide a consistent set of tools to improve researchers'
workflows from the exploratory phase of the analysis to the communication
of the results \cite{Kluyver2016}.

Split in 2014 from the \emph{IPython Project} \cite{IPython}, Jupyter has grown rapidly in
popularity and adoption both in the industry and academia. We estimate the user
base of the Jupyter notebook to be in the millions \cite{jupyter-grant}. Users range from data
scientists to researchers, educators, and students from many fields,
including journalists and librarians. In 2017, the Jupyter
team was awarded the \emph{ACM Software System Award}, an annual award that
honors people or an organization \emph{"for developing a software system that had a
lasting influence"}. Prior recipients include \emph{Unix}, \emph{TCP/IP}, and
the \emph{World Wide Web} \cite{acm-award}.

A large number of discrete software components make up Project Jupyter.
While these interact with one another, many can be installed separately
to serve various use cases. For this proposal, we loosely divide the
software involved into \emph{Jupyter core} developed under the guidance
of the developers who started the project, and the broader \emph{Jupyter
ecosystem} including software developed by third parties,
which may interact or build upon core Jupyter components.
Some of the components and concepts important to \TheProject are detailed below.

\begin{figure}[ht]\centering
  \centering
  \includegraphics[width=0.9\textwidth]{spectrogram_smaller.png}
  \caption{A notebook document in the Jupyter Notebook interface.}\label{fig:notebook-screenshot}
\end{figure}

\medskip\noindent\emph{Jupyter core}
\begin{itemize}
  \item The \textbf{Jupyter Notebook} is the flagship application of Project Jupyter.
  It allows the creation of notebook documents, containing a mixture of text and
  interactively executable code, along with rich output from running that code.
  Figure \ref{fig:notebook-screenshot} shows an open notebook including graphs
  from an audio processing example. Notebook documents are readily shareable,
  providing a popular way to describe and illustrate computational methods and
  tools.
  \textbf{JupyterLab} is the new, modular, extensible client application
  for Jupyter notebooks, but the document format, server, and user model are the same.

  \item \textbf{Jupyter kernels} are the backend software which allow Jupyter to execute
  code in many different programming languages. The \textbf{IPython} kernel is
  the reference kernel, supporting the Python programming language, and is
  developed by the Jupyter core team. Kernels for other languages are maintained
  by third parties

  \item \textbf{nbconvert} converts notebook files to a variety of other file
  formats, including HTML and PDF, so that the content of a notebook can easily
  be shared with people who don't have Jupyter software. nbconvert also powers
  \textbf{nbviewer}, a web service which provides static HTML views of publicly
  accessible notebooks.

  \item \textbf{JupyterHub} is a multi-user extension of the Jupyter Notebook.
  It runs on one or more notebook servers, for example at a research institution.
  Users can log in to author and run notebooks securely through their web
  browser, without needing to install any special software on their own
  computer.

\end{itemize}

\medskip\noindent\emph{Jupyter ecosystem}\label{jupyter-ecosystem}

While Jupyter is a large, distributed, coordinated project,
the wider community of Jupyter users develops a great deal of
software with Jupyter integration,
providing increased or domain-specific functionality,
building on top of Jupyter, or integrating core Jupyter components in some aspect.
We call this the \textbf{Jupyter ecosystem}.
The broader Jupyter ecosystem includes many more projects than we will describe
here, but a selection of projects which are relevant to
\TheProject includes:

\begin{itemize}
  \item \textbf{Binder} builds on JupyterHub to allow sharing executable
  environments along with data files and a description of the software components
  required to run the notebooks. When someone accesses a Binder repository,
  the service builds the computational environment on demand, allowing them to
  execute and modify a copy of the notebooks.
  \textbf{repo2docker} \cite{repo2docker} and \textbf{BinderHub} \cite{binder} are components of the Binder
  software.

  \item \textbf{nbsphinx} \cite{Nbsphinx} integrates notebooks with the \emph{Sphinx}
  documentation system, which is widely used for software documentation,
  especially but not only for software written in Python.
  This allows developers to write notebooks showing how to use their software,
  then seamlessly make those notebooks part of their main documentation.

  \item \textbf{nbval} \cite{nbval} is a plugin for the popular \emph{pytest} testing
  framework to automatically execute notebooks and optionally check that the
  output matches that saved in the file. While this is not a subsitute for a
  test suite, it's valuable for documentation with code examples in notebooks.
  If changes to the underlying tools mean the example no longer
  works, testing with nbval will quickly show this, so that either the software
  or the example can be corrected. This ensures that example code and
  documentation don't get outdated.

  \item \textbf{nbdime} \cite{nbdime} provides tools for comparing and merging notebooks.
  These integrate with version control systems such as \emph{git}, which
  are designed for plain text files and typically don't handle notebook files
  well.

  \item \textbf{Widgets} allow interactive output in the notebook which can
  communicate with the kernel, updating values in the kernel and updating the
  displayed output as code runs. \textbf{ipywidgets} \cite{ipywidgets} provides the main
  implementation for the IPython kernel, while other packages such as
  \textbf{bqplot} \cite{bqplot}, \textbf{ipyvolume} \cite{ipyvolume} and
  \textbf{K3D} \cite{K3D} extend the framework to provide 2D and 3D visualisations.
  Figure \ref{fig:ipywidgets-example} shows a simple example of interactive
  widgets in use.

  \item The \textbf{Voila} package \cite{Voila} enables the
  sharing of notebook-based interactive dashboards for non-technical users.

  \item The \textbf{Xeus} instrastructure \cite{Corlay2017} supports writing kernels
  in C++. \textbf{xeus-cling} is one such kernel, running user code in C++,
  and built upon CERN's C++ interpreter, "cling" \cite{Vassilev2012},
  which has significant adoption in the High Energy Physics community.
  xeus-cling is already in use for teaching the C++ programming language.
\end{itemize}

\begin{figure}[ht]\centering
  \includegraphics[width=0.5\textwidth]{ipywidgets_example.png}
  \caption{An example of using two simple slider widgets to explore the
  parameter space of a function. The \texttt{@interact} decorator creates
  the widgets and connects them to the function.}
  \label{fig:ipywidgets-example}
\end{figure}

\medskip
\noindent\textbf{Jupyter as a basis for web services}\\
Because the Jupyter notebook is a web-based application, it can be
deployed at computational facilities or in the cloud, and can function
as the basis for services exposing computational resources of all
kinds to researchers and the public.  Because Jupyter is
\textbf{interactive}, it enables making scientific results and
communications more interactive than static publications.  The
audience can follow their own initiative and ask their own questions
of published data without needing support from the publishing author,
greatly facilitating the \textbf{practicality of Open Science}.

\medskip
\noindent\textbf{Jupyter is generic}\\
\TheProject chose Jupyter because it is
Generic.  Jupyter makes no domain-specific or even language-specific
assumptions.  Any application where mixing description, code, and
results is valuable can make use of Jupyter.  This broad applicability
makes investment in the Jupyter ecosystem extremely effective, because
improvements to Jupyter can serve many communities simultaneously.

Jupyter is built from a collection of standard protocols and file
formats.  Jupyter is not just a single, monolithic piece of
software, but a description of how such software can be built.  The
result is the ability for a variety of communities and applications to
use components of Jupyter for their purposes, and/or reimplement pieces to
meet their needs.
%
For example:
\begin{enumerate}
\item The notebook file format is a well-specified JSON document,
  which can be interpreted by many systems.  This has facilitated the
  development of different services providing rendering of notebooks, e.g. the code
  hosting website GitHub, which renders notebooks for easy viewing by
  anyone, without Jupyter software.
\item The Jupyter protocol describes how execution is performed, which
  has enabled the development of over one hundred kernel
  implementations in dozens of languages\footnote{\url{https://github.com/jupyter/jupyter/wiki/Jupyter-kernels}}.
\item Output in the Jupyter protocol uses web-standard MIME types,
  enabling any possible format to be an output in a Jupyter notebook.
\item The JupyterLab extension system provides a system for building
  applications from Jupyter components and others.
\item The Jupyter Widgets provide a system for customizing and
  extending interactivity in Jupyter-based environments.
\end{enumerate}

The popularity of Jupyter, with millions of users and hundreds of open
source contributors, is an indicator of the value and impact of this approach.

\medskip
\noindent\textbf{Improvement to the Jupyter ecosystem}\\
The benefits of focusing our work on a mature system like Jupyter include:

\begin{itemize}
\item vibrant community ensures health and sustainability,
\item large existing user base maximises impact of contributions,
\item mature software ecosystem maintains quality software through
  industry standards such as version control, tests, continuous
  integration, stable release cycles, roadmaps, and user support.
\end{itemize}

The Jupyter community aims to be inclusive, and \TheProject fully
embraces and supports that approach.  Jupyter is inclusive across a number of axes.
By being applicable across numerous domains, Jupyter and \TheProject
encourage participation from individuals of various interests and
backgrounds, and has taken action to improve diversity in the project
by participating in ``Outreachy,'' a program of paid internships for
individuals from groups that face under-representation, systemic bias,
or discrimination.  Jupyter has also operated workshops focused on
training contributors from under-represented groups.  In being free,
public, open source software, Jupyter and \TheProject are accessible
to as many individuals as possible, and invites users and contributors
beyond origin, nationality, beliefs, orientation.  One area where
Jupyter has lacked in this regard is in the User Interface
accessibility, and we will help improve this in
% \taskref{core}{accessibility}
.  Additionally, the project will
focus some of its workshops in
% \taskref{education}{workshops}
on
under-represented communities.


\begin{figure}[ht!]\centering
  \includegraphics[width=0.6\textwidth]{images/notebook_components.png}
  \caption{The architecture of the Jupyter Notebook, kernels, and tools
        which operate on notebook files}
  \label{fig:notebook-architecture}
\end{figure}

\medskip
\noindent\textbf{Related projects}

EOSC-hub is a 33 million Euro H2020 project that started in January 2018 with
the involvement of over 100 institutes. In three years the project is
establishing the first elements of the European Open Science Cloud. EOSC-hub
defines, creates and operates the integration and management system of the
EOSC.  This integration and management system (the Hub) builds on mature
processes, policies and tools from the leading European e-infrastructures to
cover the whole life-cycle of services from planning to delivery. Through this
management system online and `human' services, software and data are delivered
towards researchers via a single EOSC Portal. The Marketplace already includes
nearly 50 services from EOSC-hub provided by 3 e-infrastructure communities
(EGI, EUDAT, INDIGO-DataCloud), and from 18 Research Infrastructures and
scientific service providers. The catalogue of services is expected to
radically grow in the next years through national, regional and EU
initiatives.

Integrating Jupyter-based services into EOSC provides an excellent opportunity
for facilitating interoperability of EOSC services,
bringing data and computation together in a flexible environment.

\subsubsection{Methodology}\label{sec:methodology}


\textbf{Proposed improvements to core components of Jupyter (\WPref{core})}\\
We plan to make technical changes to Jupyter software to better support
% real-time collaboration (\taskref{core}{collaboration}),
so that two or more people in different places or working on different
devices can work together
on the same notebook. This would significantly enhance the value of
notebooks for collaborative research.
We will also work on making Jupyter software accessible to as broad a
% range of users as possible (\taskref{core}{accessibility}).

Further work to bring the code behind JupyterHub and Binder closer together
% (\taskref{core}{jh-bh-conv})
will bring a range of benefits, allowing more
flexible sharing of notebooks along with access to remote computing resources
such as those available through EOSC.

Finally, we are explicitly allocating time in \WPref{core} for maintaining
Jupyter software, as well as new development
 % (\taskref{core}{maintenance}).
Maintenance is crucial to creating reliable, sustainable software,
but its cost is often swept under the rug in funding applications
because of the perceived pressure to focus on novelty.
Being up front and explicit about this cost is critical to the sustainability
of open source open science.

\medskip
\noindent\textbf{Proposed improvements to the Jupyter ecosystem (\WPref{ecosystem})}\\
We further propose improvements to the wider Jupyter ecosystem for
better scientific workflows. In particular, we have identified
possible improvements to:

\begin{itemize}
  \item Binder and its crucial software component \emph{repo2docker}
    % (\taskref{ecosystem}{r2d-and-binder}).

  \item Xeus, to better support the C++ programming language in notebooks
    % (\taskref{ecosystem}{xeus-cpp}).

  \item Interactive widgets, including tools for 3D visualisation to help
    people make sense of large amounts of data
    % (\taskref{ecosystem}{jupyter-widgets}).

  \item Archiving of computational environments to allow reproducible research
    with a focus on the long term
     % (\taskref{ecosystem}{reproducibility}).

\end{itemize}

We may create new open source software projects in these tasks,
but we will carefully review existing software, both in the
Jupyter ecosystem and beyond, to avoid unnecessary duplication of effort.

\medskip\noindent\textbf{Beyond the improvement to the Jupyter Project
  (\WPref{applications}, \WPref{eosc}, \WPref{education})}\\
Beyond the improvement to the Jupyter core and ecosystem software for EOSC, we plan on
\begin{itemize}
\item Design, implementation, application, demonstration and
  evaluation of new innovative EOSC services
  in multiple demonstrators, that cover research fields such as
  health, astrophysics, photon and neutron science, geosciences and
  mathematics, and also interests of participating SMEs (\WPref{applications}).
\item Operating a \emph{European Binder Service} on the EOSC-Hub and
  enabling provision of Jupyter Services through the EOSC-Hub (\WPref{eosc}).
\item Producing \emph{training and education material} to disseminate
  the ability to do reproducible computational science using the tools
  we develop, among others (\WPref{education}).
\end{itemize}

\medskip
\noindent
\textbf{The science
  demonstrators}\label{sec:science-demonstrators-in-concept}\\

We describe the context and challenges for each demonstrator in this
section. The particular planned activities are shown in the
corresponding tasks in \WPref{applications}.\\
