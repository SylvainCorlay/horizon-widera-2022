\subsection{Objectives and ambition}

\subsubsection{Ambition}

\TheProject's ambition is to \myemph{improve the global reproducibility of
  scientific results} with focus on those aspects of the research process that
are supported by computation and software, such as computer simulation, data
processing, data analysis and creation of figures and tables in publications.

We want to achieve this ambition through
\begin{compactitem}
\item educating researchers about good reproducible practices, and
\item make it easier to perform computational research in a reproducible way
  through improving and developing relevant software tools.
\end{compactitem}

\subsubsection{State of the Art}

To make computational research reproducible, we generally need to archive
(i)~required data, (ii)~the required software, (iii)~the protocol that explains
how to process the data to obtain the result that is to be reproduced. Using
services such as Zenodo, it is possible to deposit such archives with a DOI and
make reference to them in publications.

In order to reproduce the results, it should be possible for anybody to take such an archive of a research
output, and to carry out the two necessary steps:
\begin{compactitem}
\item Step 1 to install the required software environment, and
\item Step 2 to follow the protocol to reproduce the results from the archived data.
\end{compactitem}

It is of particular value if Steps 1 \emph{and} 2 can be \emph{carried out
  automatically} by executing some kind of script or program that is part of the
archive. First, if the automatic execution is possible, we know that there is a
complete description of the protocol included in the archive, and that no
mistake is made in trying to follow the protocol. Second, the automatic
execution saves time.

\emph{Given enough training or experience}, researchers can use existing technical
solutions to compose such archives, and to allow a completely automatic
reproduction of the necessary software environment, and then reproduction of the actual
result using that software. 

\medskip Researchers who use the Jupyter Notebook to orchestrate their
computational research (see Figure~\ref{fig:jeodpp}) can achieve this automatic
reproducibility with little additional effort: they use the Notebook document as
the protocol of their analysis (Step 2), which can be executed automatically.
They can make use of the Binder tool (Section~\ref{seq:project-binder}) and the
associated mybinder.org service that has
been designed by the Jupyter team to automatically create the appropriate
software environment (step 1) in which the notebook can be executed.

\begin{figure}[tb]
  \centering\includegraphics[height=0.2\textheight]{images/jeodpp.png}
  \centering\includegraphics[height=0.2\textheight]{images/jeodpp-demo.jpg}
  \caption{\emph{Left}: The Joint Research Centre (JRC) Earth Observation
    Data and Processing Platform (JEODPP) is a user of the
    Jupyter Notebook (source:
    \url{https://cidportal.jrc.ec.europa.eu/home/}), where it features
    at the top of the pyramid to help users orchestrate layers of data
    analysis software and hardware. \emph{Right}: An example
    service in which an interactive visualisation is provided through
    the Jupyter Notebook rendering of the density map of the ships
    detected from Sentinel-1 images over the Mediterranean sea during
    the period October 2014 to September 2016. \cite[Figure
    6]{Soille2018}. \label{fig:jeodpp}}
  \TODO{Do we want to keep the figure? The left side is useful, and it is good
    to have some images.}
\end{figure}

\subsubsection{Beyond state of the art}

In this project, we will focus on the \emph{reproduction of the
  software environment} (Step 1) which is a prerequisite for any attempt to
reproduce the actual research outputs. In particular, we want to make the
creation of this computational environment \emph{automatic}.

We will go beyond the current state of the art by
\begin{compactitem}
\item enhancing the existing Binder software for Jupyter notebooks,
  directly improving the reproducibility of research created by the substantial
  community of researchers using such Notebooks,
\item extending the capabilities of the Binder tools so that its capability to
  create arbitrary software environments automatically can be used outside the
  Jupyter ecosystem,
\item adding new capabilities to the Binder tools that enable new reproducibility
  use cases, such as those than need access to large data sets, access to restricted sets,
  and reproducibility for High Performance Computing,
\item enabling the Binder tools to run on the desktop computer of individual
  researchers (rather than having to rely on central or institutional
  installations such as mybinder.org).
\end{compactitem}

\subsubsection{Motivation - Why?}

We focus on the computational reproducibility because it is a real
obstacle for practical reproducibility. 

Firstly, it affects the majority of all researchers: there are estimates that over 92\%
of all researchers work with \emph{research software} and over 50\% develop
their own~\cite{Hettrick2014}. Where experiments drive the research, this is
often data processing, analysis and plotting. Each of those computational
research cases needs a software environment in which the actual processing can
be carried out. The software environment may consist of somewhat standard packages (for example
use of a Python, R or Julia plotting library), or it may include tailored
programs that have been developed especially for this study.

Secondly, software packaging and management is a technically challenging topic,
and we cannot expect 92\% of all researchers to master it -- so we believe there
is a clear need to support this with appropriate tooling.

The complexity arises in parts from the increasing age of archived studies, and
also in the often unusual combinations of research software and libraries that
need to be combined for a particular study. Other difficulties include that a
reproduction typically needs to be done on a different computer, perhaps even on
a different operating system. If, say, a plotting library is used, then it may
have changed its interface or behaviour over time, so it is important to install
exactly the right version of the plotting library, before a reproduction of
results is attempted using it.

Thirdly, being able to re-create the appropriate software environment is a
pre-requisite before any actual reproduction of results can be attempted: it
would be inefficient to educate researchers what data and programs to archive,
if in the future nobody (or only very few highly trained people) will be able to
execute those scripts.

Finally, we think that there are low hanging fruits: the work proposed here will
make it possible create computational environments automatically for
\emph{existing data archives}: the Binder philosophy is to understand existing standards for software
specification, and to build a software environment based on those standards.
Where researchers have used the existing software specification already, Binder
will work immediately on their archived files. This means that (i)~a researcher
putting together a well organised archive does not need to know about Binder,
yet the researcher who wants to reproduce the results later can use Binder to
automate the recreation of the software environment. This also means that
(ii)~improvements we propose in this work, will make some existing archives (that
have been created in the past) more easily reproducible.

%(It is part of our training programme to educate about the importance and methods
%for software specification.)

\subsubsection{Objectives}\label{sect:objectives}
\begin{compactenum}[\myemph{Objective} 1:]

\item \label{obj:reproducibility} \myemph{Facilitate better computational
    reproducibility and FAIR data} by 
  improving the \myemph{reproducibility of computational environments}
  used for science, and facilitating \myemph{FAIR data practices}.
  We will contribute to the recording and automatic reproducibility
  of environments with Binder,
  and extend capabilities to better support FAIR
  data requirements. In particular, the archival of execution
  environments to support \myemph{reusability} of notebooks in the future
  needs attention. Such notebooks may, for example, be published alongside
  traditional publications to detail the computation of published data
  and figures, and address the Reusablity requirement of FAIR data.
  \TODO{This needs editing. It is too long in comparison to the other
    objectives. It is not clear to me how the FAIR data comes in here.}

\item \label{obj:broaden} \myemph{To broaden the impact} of existing Binder tools
    for reproducibility by expanding the feature set, applicable domains and use
    cases. In particular, make it possible to create computational environments
    outside the Jupyter ecosystem.
    
  \item \label{obj:demonstrators} \myemph{Provide demonstrator use cases for
      reproducible science} to validate and demonstrate the value of the work. We will apply
    improved tools and guidelines to a number of reproducible use cases in
    academic research, education, research infrastructures and SMEs.

\item \label{obj:education} \myemph{Outreach and engagement} --- Develop best
  practice guidelines for reproducible science, and disseminate this by
  educating the research communities about reproducible practices, and available
  tools for reproducible publications and policies. Reach out to scientists, and
  the wider research communities and reproducibility stakeholders to encourage
  engagement with this project.
  % and exploitation of existing tools for Reproducible and Open Science
  % for their research domains and interests.
  % Engaging a larger community will help \myemph{ensure the sustainability} of
  % the services and underlying infrastructure by distributing its
  % development, hosting, and maintenance over stakeholders from a
  % variety of institutions and backgrounds,
  % from the private sector to public research, education
  % and open government.



\end{compactenum}

\begin{table}
  \label{tab:objectives-tasks}
  \caption{
  Each objective and the tasks which further that goal.}
  \begin{tabular}{|m{.3\textwidth}|m{.7\textwidth}|}

    \hline

    \myemph{Objective} & \myemph{Tasks}
    \\\hline

    \ref{obj:reproducibility} &

    % \longtaskref{core}{maintenance}
    % \longtaskref{core}{jh-bh-conv},

    \\\hline

    \ref{obj:broaden} &

    % \longtaskref{core}{accessibility},
    % \longtaskref{core}{collaboration},
    % \longtaskref{ecosystem}{xeus-cpp},
    % \longtaskref{ecosystem}{jupyter-widgets},
    % \longtaskref{ecosystem}{teaching-tools}

    \\\hline

    \ref{obj:demonstrators} &
    % \longtaskref{applications}{astro},
    % \longtaskref{applications}{teaching},
    % \longtaskref{applications}{application-gpu},
    % \longtaskref{applications}{geoscience},
    % \longtaskref{applications}{opendose-analysis},
    % \longtaskref{applications}{math},
    % \longtaskref{applications}{reproducibility-xfel}

    \\\hline

    \ref{obj:education} &

    % \longtaskref{education}{workshops},
    % \longtaskref{education}{online-resources},
    % \longtaskref{education}{helpdesk}

    \\\hline

  \end{tabular}
  \TODO{Complete table, or delete.}
\end{table}




\subsubsection{Excellence}

We have a diverse and interdisciplinary team driving this project, in which we
bring together research software developers, researchers, research support staff
and educators -- each world class in their domain -- with the common vision to
work towards better open source tools for better reproducibility in science.

Through our multiple and interdisciplinary roles, we see the same process of
generating reproducible research outputs from the perspective of different
stakeholders, and can propose and develop solutions that are \emph{useful and
  practical} in real-world research environments.

We extend our own experience through our wide network of collaborators and
colleagues and will -- as part of the execution of this project -- seek
constant exchange with and feedback from different additional stakeholders in
our \emph{Community Engagegment Panel} to shape the work of the
\TheProject project. As a team experienced in developing open source software,
we expect to be able to go beyond this and attract development contributions
from volunteers to support this project.

The existing Binder tools -- which are the baseline for this project --
originate from Project Jupyter. We have a core Jupyter and Binder developer in
our team, and thus direct access to developer expertise and experience.

\subsubsection{Impact}

We have already discussed above that computational reproducibility affects the
majority of researchers, including those who carry out work that is not
predominantly computational.

The impact of this project will be realised through (i) the training we develop
and disseminate, and (ii) the improved Binder tools. The improved tools will
directly impact the community of researchers using Jupyter notebooks. However,
the \emph{new functionality and applicability of the Binder tools outside Project
Jupyter} will benefit all researchers who need computational reproducibility, and
may be -- and in the long run -- more important.

\medskip

We can give some indication of the size and activity of the Jupyter notebook
community and use of the existing Binder tools: Jupyter Notebooks are used to
support research in numerous communities, including
%\begin{compactitem}
%\item
\noemph{Journalists} and practitioners of \noemph{data-driven
    journalism} at the LA Times, BuzzFeed News, Columbia Journalism School
  \cite{latimes-datadesk} \cite{columbia-nytimes} \cite{data-journalism};
  \noemph{Data scientists} in academia, industry and services \cite{Perkel2018};
  % \item
  \noemph{Research institutions} such as CERN, EuXFEL, JRC, and many more,
  operating institution-wide Jupyter deployment;
  % \item
  \noemph{Universities} using Jupyter as a teaching platform;
  % \item
  \noemph{Large cloud providers} building commercial products on the
  top of Jupyter (Google DataLab and Colaboratory, Amazon Sagemaker, Microsoft Azure
  Notebooks);
  % \item
  within the \noemph{European Open Science Cloud} Jupyter is used in many EOSC
  projects, and a JupyterHub service is provided by the EGI foundation.
  % \item
  This impact was recognised by the \emph{ACM Software System Award} that was
  awarded to the Jupyter team to honour \emph{"developing a software system that
    had a lasting influence"} in 2017. (Prior recipients include \emph{Unix},
  \emph{TCP/IP}, and the \emph{World Wide Web} \cite{acm-award}.)
  
  There are 8 million notebooks deposited on GitHub \cite{notebookcount}, and
  the size of the Notebook user base was estimated to be of the order of
  millions in 2015 \cite{jupyter-grant}. We know that the Binder service for
  Jupyter notebooks is used to create about 30,000 computational environments
  every day (Section~\ref{sec:mybinder}) to enable execution of notebooks within
  that computational environment.

  
%  research effectiveness for many of their millions of users. 
  
%   Our aim is to re-use the work that has gone into the development of the Binder
%   tools, and to make this functionality for automatic creation of computational
%   environments available to researchers outside the Jupyter user community to
%   improve reproducibility at a wider scale.




  \subsubsection{Technical Readiness Level (TRL)}

  The Binder software and service prototype at mybinder.org is TRL 6. We will
  bring Binder to at least TRL 8 during the course of the project.



  %%% Local Variables:
  %%% mode: latex
  %%% TeX-master: "proposal"
  %%% End:
