\subsection{Objectives and ambition}

\subsubsection{Ambition}

\TheProject's ambition is to \myemph{improve the global reproducibility of
  scientific results} with focus on those aspects of the research process that
are supported by computation and software, such as computer simulation, data
processing, analysis and creation of figures and tables in publications.

We want to achieve this ambition through
\begin{compactitem}
\item educating researchers about good reproducible practices, and
\item make it easier to perform computational research in a reproducible way
  through improving and developing relevant software tools.
\end{compactitem}

\subsubsection{State of the Art}

To make computational research reproducible, we generally need to archive
(i)~required data, (ii)~the required software, (iii)~the protocol that explains
how to process the data to obtain the result that is to be reproduced. Using
services such as Zenodo, it is possible to deposit such archives with a DOI and
make reference to them in publications.

Ideally, it should be possible for anybody to take such an archive of a research
output, and to carry out the two necessary steps:
\begin{compactitem}
\item Step 1 to install the required software environment, and
\item Step 2 to follow the protocol to reproduce the results from the archived data.
\end{compactitem}

It is of particular value if Steps 1 \emph{and} 2 can be \emph{carried out
  automatically} by executing some kind of script or program that is part of the
archive. First, if the automatic execution is possible, we know that there is a
complete description of the protocol included in the archive, and that no
mistake is made in trying to follow the protocol. Second, the automatic
execution saves time.

\emph{Given enough training or experience}, researchers can use existing technical
solutions to compose such archives, and to allow a completely automatic
reproduction of the necessary software, and then reproduction of the actual
result using that software. 

\medskip Researchers who use the Jupyter Notebook to orchestrate their
computational research can achieve this automatic reproducibility with little
additional effort: they use the Notebook document as the protocol of their
analysis (Step 2), which can be executed automatically. They can make use of the
Binder tool that has been designed by the Jupyter team to automatically create
the appropriate software environment (step 1) in notebooks can be executed.

\subsubsection{Beyond state of the art}

In this project, we will focus on the \emph{reproduction of the
  software environment} (Step 1) which is a prerequisite for any attempt to
reproduce the actual research outputs. In particular, we want to make the
creation of this computational environment automatic.

We will go beyond the current state of the art:
\begin{compactitem}
\item improvements to Binder software for Jupyter notebooks, directly improving
  the reproducibility of research created by the substantial community of
  researchers using such Notebooks
\item extending the capabilities of the Binder tool so that its capability to
  create arbitrary software environments automatically can be used outside the
  Jupyter Ecosystem.
\item add new capabilities to the Binder tools that enable new reproducibility
  use cases, such access to large data sets, restricted access to data,
  and reproducibility for High Performance Computing.
\end{compactitem}

\subsubsection{Motivation - Why?}

The reproduction of software environments is a real obstacle for reproducible computation.
- a real problem

- not much attention

- low hanging fruit - attach to existing standards



\subsubsection{Objectives}\label{sect:objectives}

% \noindent The aims of \TheProject are to:

% \TOWRITE{remove aims}
% \begin{compactenum}[\myemph{Aim} 1:]
% \item Enable a
%   \myemph{sustainable}, \myemph{community-developed}, general purpose, \myemph{interoperable} toolbox for
%   interactive computing, data processing, and visualization
%   \myemph{that facilitates the entire life-cycle of open science},
%   from initial exploration to \myemph{reproducible publication}, research and development in
%   industry, teaching, and outreach.
%   This is by supporting and steering the Jupyter software ecosystem,
%   which exists to develop open source software,
%   open standards, and services for interactive computing across dozens of programming languages.
%
% \item Leverage this technology for all scientists, across borders,
%   domains, disciplines, and demographics, through
%   \myemph{free public distributed collaborative services} tightly integrated
%   into the European Open Science Cloud (EOSC),
%   in collaboration with a federation of related services
%   operated by the wider community.
%
% \item Demonstrate the value and versatility of such services through
%   \myemph{innovative co-designed tailored applications} in a variety of disciplines and
%   contexts.
%
% \item \myemph{Support open science} and maximize impact through development and
%   dissemination of best practices,
%   \myemph{training}, and \myemph{community building}
%   around the usage and development of the above toolbox,
%   with a focus on \myemph{interoperability}, \myemph{reproducibility}
%   and \myemph{reusability}.
% \end{compactenum}
% develop and support the Jupyter ecosystem in a direction that benefits and facilitates open science
%     make these tools accessible to as many people as possible via operation of free, public services
%     demonstrate and ensure that these developments are useful to real scientists and the public
%     foster open science through training of students and researchers in best practices using these tools


% \item \label{aim:facilitation}
%   Facilitate open science through the development
%   of tools enabling reproducibility, sharing, and collaboration.

% \item \label{aim:accessibility}
%   Maximise accessibility and interoperability of open science services and tools,
%   across domains, disciplines, and demographics.

% \item \label{aim:sustainability}
%   Maximise sustainability of software tools for open science
%   by developing the community and contributing
%   to and supporting community-led software efforts.



%   Support open source software for open science, and notably the
%   Jupyter ecosystem,



% \end{compactenum}

\medskip
\noindent We will achieve our ambition through the following objectives:

\TOWRITE{mention TRL goals}

\begin{compactenum}[\myemph{Objective} 1:]

\item \label{obj:reproducibility} \myemph{Facilitate reproducibility and FAIR data} ---
  improving the \myemph{reproducibility of computational environments}
  used for science, and facilitating \myemph{FAIR data practices}.
  We will contribute to the recording and reproducibility
  of environments with repo2docker and Binder,
  and extend capabilities to better support FAIR
  data requirements. In particular, the archival of execution
  environments to support \myemph{reusability} of notebooks in the future
  needs attention. Such notebooks may, for example, be published alongside
  traditional publications to detail the computation of published data
  and figures, and address the Reusability requirement of FAIR data.

\item \label{obj:broaden} broaden the impact of existing tools for reproducibility by expanding the applicable domains and use cases
\item \label{obj:demonstrators}
  \myemph{Demonstrators in science and education} ---
  We will demonstrate and ensure the versatility and value of the components and
  the services built from them,
  through applications to a number of
  domains in academic research, education, research infrastructures, SMEs, and for
  the public sector, driven through our project partners. In
  particular, we will contribute demonstrators in the following areas:
  % astronomy (\taskref{applications}{astro}), education
  % (\taskref{applications}{teaching}), fluid dynamics
  % (\taskref{applications}{application-gpu}), geosciences
  % (\taskref{applications}{geoscience}), health
  % (\taskref{applications}{opendose-analysis}), mathematics
  % (\taskref{applications}{math}),
  % and photon science (\taskref{applications}{reproducibility-xfel}),
  % involving universities, research infrastructure facilities, and SMEs.

\item \label{obj:education}
  \myemph{Outreach, engagement, and sustainability} ---
  Reach out to scientists and the wider research
  communities to encourage engagement
  and exploitation of existing tools for reproducible and open science
  for their research domains and interests.
  Educate the research communities about reproducible practices,
  and available tools for reproducible publications and policies.
  Engaging a larger community will help \myemph{ensure the sustainability} of
  the services and underlying infrastructure by distributing its
  development, hosting, and maintenance over stakeholders from a
  variety of institutions and backgrounds,
  from the private sector to public research, education
  and open government.



\end{compactenum}

\begin{table}
  \label{tab:objectives-tasks}
  \caption{
  Each objective and the tasks which further that goal.}
  \begin{tabular}{|m{.3\textwidth}|m{.7\textwidth}|}

    \hline

    \myemph{Objective} & \myemph{Tasks}
    \\\hline

    \ref{obj:reproducibility} &

    % \longtaskref{core}{maintenance}
    % \longtaskref{core}{jh-bh-conv},

    \\\hline

    \ref{obj:broaden} &

    % \longtaskref{core}{accessibility},
    % \longtaskref{core}{collaboration},
    % \longtaskref{ecosystem}{xeus-cpp},
    % \longtaskref{ecosystem}{jupyter-widgets},
    % \longtaskref{ecosystem}{teaching-tools}

    \\\hline

    \ref{obj:demonstrators} &
    % \longtaskref{applications}{astro},
    % \longtaskref{applications}{teaching},
    % \longtaskref{applications}{application-gpu},
    % \longtaskref{applications}{geoscience},
    % \longtaskref{applications}{opendose-analysis},
    % \longtaskref{applications}{math},
    % \longtaskref{applications}{reproducibility-xfel}

    \\\hline

    \ref{obj:education} &

    % \longtaskref{education}{workshops},
    % \longtaskref{education}{online-resources},
    % \longtaskref{education}{helpdesk}

    \\\hline

  \end{tabular}
  \TODO{HF: Is this table compulsory?} 
\end{table}


\subsubsection{Excellence}

We have a diverse and interdisciplinary team driving this project, in which we
bring together research software developers, researcher, research support staff
and educators - each world class in their domain - with the common vision to
work towards better open source tools for better reproducibility in science.

As active researchers, we are aware of the need and difficulties in creating
reproducible research outputs. 

The existing Binder tools -- which are the baseline for this project --
originate from Project Jupyter. We have a core Jupyter and Binder developer in
our team, and thus direct access to developer expertise and experience.

We are well connected and will, as part of the execution of this project, seek
constant exchange with different stakeholders to shape the work of the
\TheProject project. As a team experienced in developing open source software,
we expect to be able to go beyond this and attract development contributions
from volunteers to support this project.

\subsubsection{Impact}

There are estimates that over 50\% of all researchers work with \emph{research software}
- everybody has computational research challenges
  - also experimental groups

  - Jupyter community big (notebooks, users)

  , which has changed the work, education and
  research effectiveness for many of their millions of users. This impact was
  recognised by the \emph{ACM Software System Award} that was awarded to the
  Jupyter team to honour \emph{"developing a software system that had a lasting
    influence"} in 2017.
  
  Our aim is to re-use the work that has gone into the development of the Binder
  tools, and to make this functionality for automatic creation of computational
  environments available to researchers outside the Jupyter user community to
  improve reproducibility at a wider scale.

- 


\subsubsection{TRL}